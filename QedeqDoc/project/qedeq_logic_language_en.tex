% -*- TeX:EN -*-
%%% ====================================================================
%%% @LaTeX-file  qedeq_logic_language_en.tex
%%% Generated at 2007-12-12 13:52:44,574
%%% ====================================================================

%%% Permission is granted to copy, distribute and/or modify this document
%%% under the terms of the GNU Free Documentation License, Version 1.2
%%% or any later version published by the Free Software Foundation;
%%% with no Invariant Sections, no Front-Cover Texts, and no Back-Cover Texts.

\documentclass[a4paper,german,10pt,twoside]{book}
\usepackage[english]{babel}
\usepackage{makeidx}
\usepackage{amsmath,amsthm,amssymb}
\usepackage{color}
\usepackage[bookmarks,bookmarksnumbered,bookmarksopen,
   colorlinks,linkcolor=webgreen,pagebackref]{hyperref}
\definecolor{webgreen}{rgb}{0,.5,0}
\usepackage{graphicx}
\usepackage{xr}
\usepackage{epsfig,longtable}
\usepackage{tabularx}

\newtheorem{thm}{Theorem}[chapter]
\newtheorem{cor}[thm]{Corollary}
\newtheorem{lem}[thm]{Lemma}
\newtheorem{prop}[thm]{Proposition}
\newtheorem{ax}{Axiom}
\newtheorem{rul}{Rule}

\theoremstyle{definition}
\newtheorem{defn}[thm]{Definition}
\newtheorem{idefn}[thm]{Initial Definition}

\theoremstyle{remark}
\newtheorem{rem}[thm]{Remark}
\newtheorem*{notation}{Notation}

\addtolength{\textheight}{7\baselineskip}
\addtolength{\topmargin}{-5\baselineskip}

\setlength{\parindent}{0pt}

\frenchspacing \sloppy

\makeindex


\title{\textbf{Hilbert~II} \\
\vspace*{1cm} 
Presentation of \\ 
Formal Correct \\
Mathematical Knowledge \\
\vspace*{1cm} Logical Language}
\author{
Michael Meyling
}

\begin{document}

\maketitle

\setlength{\parskip}{5pt plus 2pt minus 1pt}
\mbox{}
\vfill

\par
The source for this document can be found here:
\par
\url{http://qedeq.org/0_03_07/doc/project/qedeq_logic_language.xml}

\par
Copyright by the authors. All rights reserved.
\par
If you have any questions, suggestions or want to add something to the list of modules that use this one, please send an email to the address \url{mailto:mime@qedeq.org}

\setlength{\parskip}{0pt}
\tableofcontents

\setlength{\parskip}{5pt plus 2pt minus 1pt}

\chapter*{Description} \label{chapter0} \hypertarget{chapter0}{}
\addcontentsline{toc}{chapter}{Description}

The project \textbf{Hilbert~II} includes formal correct mathematical knowledge. Here we introduce the underlying formal language for the mathematical formulas. This is done in an informal way. Important theorems (e.g.: universal decomposition, and any proofs) are left out.

\par
All we will do is manipulate symbols. We build lists of symbol strings and use certain simple rules to get new lists. So by starting with a few basic lists we create a whole universe of derived symbol lists. 
It turns out that these lists could be interpreted as a view to the incredible world of mathematics.

%% end of chapter Description

\chapter{Entities} \label{chapter1} \hypertarget{chapter1}{}

To describe the logical language we firstly deal with a more basic notation. This notation enables us to formulate the syntax of formulas and terms later on.

\section{Elements, Atoms and Lists} \label{chapter1_section0} \hypertarget{chapter1_section0}{}
The basic structure we have to deal with is an element. An element is either an atom or a list. 

\par
An atom carries textual data, atoms are just strings.      

\par
Each list has an operator and can contain elements again. An operator is also nothing more than a simple string. A list has a size: the number of elements it contains.  Their elements can be accessed by their position number. An atom has no operator, no size and no subelements in the previous sense.

\section{List Notation} \label{chapter1_section1} \hypertarget{chapter1_section1}{}
Lists and atoms can be written in the following manner.
We write down string atoms quoted with {\tt "} and the lists as the contents of the operator string followed by {\tt (} and a comma separated list of elements and an closing {\tt )}.

\section{Examples} \label{chapter1_section2} \hypertarget{chapter1_section2}{}
In this syntax we can write down the following element examples.

\begin{verbatim}
"I am a string atom"

EMPTY_LIST()

THIS_LIST("contains", "three", "atoms")

OPERATOR("argument 1", "argument 2")

FUNCTION_A(FUNCTION_B("1", "2"), "3")
\end{verbatim}

In the last example we have a list that has the operator \verb#FUNCTION_A# and contains two elements. The first element is \verb#FUNCTION_B("1", "2")# which is a list too. The second element is the atom \verb#"3"#.


%% end of chapter Entities

\chapter{Logical Language} \label{chapter2} \hypertarget{chapter2}{}

There are different basic things we have to do with. These are predicates, functions, subject variables and logical connectives. In the following all of them are named and described.

\section{Logical Operator Overview} \label{chapter2_section0} \hypertarget{chapter2_section0}{}
Lists are categorized according to their operators. 
Before we introduce the formal language in detail the used operators are briefly listed. 

\par
\begin{tabularx}{\columnwidth}{lll}
  \multicolumn{2}{l@{}}{\qquad\emph{logical}} \vspace*{1mm} \\
  \emph{AND}  & logical conjunction operator    & $\land$ \\
  \emph{OR}   & logical disjunction operator    & $\lor$ \\
  \emph{IMPL} & logical implication operator    & $\rightarrow$ \\
  \emph{EQUI} & logical biconditional operator  & $\leftrightarrow$ \\
  \emph{NOT}  & logical negation operator       & $\neg$ \\
  \\
  \multicolumn{2}{l@{}}{\qquad\emph{logical quantifiers}} \vspace*{1mm} \\
  \emph{FORLL}   & universal quantifier           & $\forall$ \\
  \emph{EXISTS}  & existential quantifier         & $\exists$ \\
  \emph{EXISTSU} & unique existential quantifier  & $\exists !$ \\
  \\
  \multicolumn{2}{l@{}}{\qquad\emph{variables}} \vspace*{1mm} \\
  \emph{VAR}      & subject variables           & $x, y, z, \ldots$ \\
  \emph{PREDVAR}  & predicate variables         & $A, B, R, \ldots$ \\
  \emph{FUNCVAR}  & function variables          & $f, g, h, \ldots$ \\
  \\
  \multicolumn{2}{l@{}}{\qquad\emph{constants}} \vspace*{1mm} \\
  \emph{PREDCON}  & predicate constants         & $=, \in, \subseteq, \ldots$ \\
  \emph{FUNCCON}  & function constants          & $\emptyset, \mathfrak{P}, \ldots$  \\
  \emph{CLASS}    & class term                  & $ \{ x | \phi(x) \}$ 
\end{tabularx}

\section{Terms and Formulas} \label{chapter2_section1} \hypertarget{chapter2_section1}{}
Now we define recursivly our formal language. We call some elements \emph{subject variables}, \emph{terms} and some other \emph{formulas}. We also define the relations a subject variable \emph{is free in} and \emph{is bound in} a term or a formula. If something is not according to the formal rules errors occur. The error codes are also described.

\subsection{General Error Codes
}
The atoms and lists that build up a formula or term are subject to restrictions.  The following errors occur if an atom has no content or has content with length of $0$ or an list has no operator or one of its sub-elements does not exist.
These are mainly technical error codes, only the error code 30470 shows an semantical error.

\begin{tabularx}{\linewidth}{llX}
  30400  & no element         & an element doesn't exist - it is null \\
  30410  & no atom            & an atom doesn't exist - it is null \\
  30420  & no list            & a list doesn't exist - it is null \\
  30430  & no atom content    & an atom has no content - it is null \\
  30440  & atom content empty & an atom has content with $0$ length \\
  30450  & no operator        & a list has no operator - it is null \\
  30460  & operator empty     & a list has an operator with $0$ length \\
  30470  & list expected      & list element expected but not found
\end{tabularx}


\subsection{Subject Variable
}
We call an element \emph{subject variable} iff it has the operator \emph{VAR} and its list size is $1$ with an atom as its only argument.

\par
Each subject variable is also called a \emph{term}. Only the subject variable itself is free in itself. No subject variable is bound in a subject variable.

\begin{tabularx}{\linewidth}{llX}
  30710  & not exactly one argument & list has not exactly one element \\
  30730  & atom element expected    & the first and only list element must be an atom 
\end{tabularx}


\subsection{Function Term
}
If an element has the operator \emph{FUNVAR} or \emph{FUNCON} and its list size is greater than or equal to $1$ with an atom as its first argument and the remaining arguments are all terms then it is called a term too.

\par
Iff a subject variable is free in any sub-element it is also free in the new term. No other subject variables are free. Analogous for bound subject variables. 


\begin{tabularx}{\linewidth}{llX}
  30720  & argument(s) missing      & if operator is \emph{FUNCON} the list must have at least one element \\
  30730  & atom element expected    & the first list element must be an atom \\ 
  30740  & argument(s) missing      & if operator is \emph{FUNVAR} the list must have more than one element \\
  30770  & free bound mixed         & found a bound subject variable that is already free in a previous list element \\
  30780  & free bound mixed         & found a free subject variable that is already bound in a previous list element \\
  30690  & undefined constant       & the operator is \emph{FUNCON} and this function constant has not been defined for this argument number
\end{tabularx}

\par
Any other error for term checks may occur due to the fact that all (but the first) sub-elements must be terms too.


\subsection{Predicate Formula
}
If an element has the operator \emph{PREDVAR} or \emph{PREDCON} and its list size is greater than or equal to $1$ with an atom as its first argument and the remaining arguments are all terms and no errors occur then it is called a \emph{formula}.

\par
Iff a subject variable is free in any sub-element it is also free in the new formula. No other subject variables are free. Analogous for bound subject variables. 

\begin{tabularx}{\linewidth}{llX}
  30720  & argument(s) missing      & list must have at least one element \\
  30730  & atom element expected    & the first list element must be an atom \\ 
  30770  & free bound mixed         & found a bound subject variable that is already free in a previous list element \\
  30780  & free bound mixed         & found a free subject variable that is already bound in a previous list element \\
  30590  & undefined constant       & the operator is \emph{PREDCON} and this predicate constant has not been defined for this argument number
\end{tabularx}

\par
Any other error for formula checks may occur due to the fact that all (but the first) sub-elements must be terms.


\subsection{Logical Connectives
}
If an element has the operator \emph{AND}, \emph{OR}, \emph{IMPL} or \emph{EQUI} and its list size is greater than or equal to $2$ and the remaining arguments are all formulas and no errors occur then it is called a formula too.

\par
Iff a subject variable is free in any sub-element it is also free in the new formula. No other subject variables are free. Analogous for bound subject variables. 

\begin{tabularx}{\linewidth}{llX}
  30740  & argument(s) missing      & list must have more than one element \\
  30760  & exactly $2$ elements expected & the operator is \emph{IMPL} and this list size is not equal to $2$ \\
  30770  & free bound mixed         & found a bound subject variable that is already free in a previous list element \\
  30780  & free bound mixed         & found a free subject variable that is already bound in a previous list element
\end{tabularx}

\par
Any other error for formula checks may occur due to the fact that all sub-elements must be formulas.


\subsection{Negation
}
If an element has the operator \emph{NOT}, its list size is exactly $1$ and its  only sub-element arguments is a formula then it is called a formula too.

\par
Iff a subject variable is free in the sub-element it is also free in the new formula. No other subject variables are free. Analogous for bound subject variables. 

\begin{tabularx}{\linewidth}{llX}
  30710  & exactly $1$ argument expected & list must have exactly than one element \\
\end{tabularx}

\par
Any other error for formula checks may occur due to the fact that the sub-element must be a formula.


\subsection{Quantifiers
}
If an element has the operator \emph{FORALL}, \emph{EXISTS} or \emph{EXISTSU} its first sub-element is a subject variable and its second and perhaps its third sub-element is a formula then the element is called a \emph{formula} too.

\par
Iff a subject variable is free in the sub-element it is also free in the new formula. No other subject variables are free. Analogous for bound subject variables. 

\begin{tabularx}{\linewidth}{llX}
  30760  & $2$ or $3$ arguments expected & list must have exactly $2$ or $3$ elements \\
  30540  & subject variable expected & first sub-element must be a subject variable \\
  30550  & already bound             & subject variable already bound in second or third sub-element \\
  30770  & free bound mixed          & found a bound subject variable that is already free in a previous list element \\
  30780  & free bound mixed          & found a free subject variable that is already bound in a previous list element 
\end{tabularx}

\par
Any other error for formula checks may occur due to the fact that the sub-element must be a formula.


\subsection{Class Term
}
An list element with the operator \emph{CLASS}, containing an subject variable and an formula is a term.

\par
Iff a subject variable is free in the formula and is not equal to the first sub-element (which is a subject variable) it is also free in the new term. No other subject variables are free. If a subject variable is bound in the formula it is bound in the new term. Also the first sub-element is bound. No other subject variables are bound.


\begin{tabularx}{\linewidth}{llX}
  30760  & $2$ arguments expected & the list must contain exactly two arguments \\
  30540  & subject variable expected & the first sub-element must be a subject variable \\
  30550  & already bound & the subject variable is already bound in the formula \\
  30680  & undefined class operator & the class operator is still unknown
\end{tabularx}

\par
Any other error for formula checks may occur due to the fact that the second sub-element must be a formula.


\subsection{Term
}
When checking an element for beeing a term the element must have the operator for a \emph{Subject Variable}, \emph{Function Term} or \emph{Class Term}.

\begin{tabularx}{\linewidth}{llX}
  30620  & unknown term operator & element has no operator that is known as a term operator
\end{tabularx}

\par
Any other error for the accordant operator checks may occur.


\subsection{Formula
}
When checking an element for beeing a formul the element must have the operator for a \emph{Predicate Formula}, \emph{Logical Connective}, \emph{Negation} or \emph{Quantifier}.

\begin{tabularx}{\linewidth}{llX}
  30530  & unknown logical operator & element has no known logical operator
\end{tabularx}

\par
Any other error for the accordant operator checks may occur.



%% end of chapter Logical Language

\chapter{Representations} \label{chapter3} \hypertarget{chapter3}{}

The representation of elements differ according to the viewpoint. Lets take the following formula for example.

$$y \ =  \ \{ x \ | \ \phi(x) \} \ \leftrightarrow \ \forall z\ (z \in y\ \leftrightarrow \ z \in \{ x \ | \ \phi(x) \} )$$

The predicate constant $\in$ must have been defined in previous sections.

\section{List Notation} \label{chapter3_section0} \hypertarget{chapter3_section0}{}
In list notation (see \ref{chapter1_section1}) the above formula looks like the following.

\begin{verbatim}
EQUI(
  PREDCON(
    "equal",
    VAR("y"),
    CLASS(
      VAR("x"),
      PREDVAR(
        "\phi",
        VAR("x")
      )
    )
  ),
  FORALL(
    VAR("z"),
    EQUI(
      PREDCON(
        "in",
        VAR("z"),
        VAR("y")
      ),
      PREDCON(
        "in",
        VAR("z"),
        CLASS(
          VAR("x"),
          PREDVAR(
            "\phi",
            VAR("x")
          )
        )
      )
    )
  )
)
\end{verbatim}

Due to XSD restrictions for the XML document some error codes listed in Chapter~\ref{chapter0} will not occur. Instead the XML will be classified as invalid.

\section{Java} \label{chapter3_section1} \hypertarget{chapter3_section1}{}
The list notation leads directly to the following Java code.
\begin{verbatim}
Element el = new ElementListImpl("EQUI", new Element[] {
    new ElementListImpl("PREDCON", new Element[] {
        new AtomImpl("equal"), 
        new ElementListImpl("VAR", new Element[] {
            new AtomImpl("y"), 
        }),
        new ElementListImpl("CLASS", new Element[] {
            new ElementListImpl("VAR", new Element[] {
                new AtomImpl("x"), 
            }),
            new ElementListImpl("PREDVAR", new Element[] {
                new AtomImpl("\\phi"), 
                new ElementListImpl("VAR", new Element[] {
                    new AtomImpl("x"), 
                })
            })
        })
    }),
    new ElementListImpl("FORALL", new Element[] {
        new ElementListImpl("VAR", new Element[] {
            new AtomImpl("z"), 
        }),
        new ElementListImpl("EQUI", new Element[] {
            new ElementListImpl("PREDCON", new Element[] {
                new AtomImpl("in"), 
                new ElementListImpl("VAR", new Element[] {
                    new AtomImpl("z"), 
                }),
                new ElementListImpl("VAR", new Element[] {
                    new AtomImpl("y"), 
                })
            }),
            new ElementListImpl("PREDCON", new Element[] {
                new AtomImpl("in"), 
                new ElementListImpl("VAR", new Element[] {
                    new AtomImpl("z"), 
                }),
                new ElementListImpl("CLASS", new Element[] {
                    new ElementListImpl("VAR", new Element[] {
                        new AtomImpl("x"), 
                    }),
                    new ElementListImpl("PREDVAR", new Element[] {
                        new AtomImpl("\\phi"), 
                        new ElementListImpl("VAR", new Element[] {
                            new AtomImpl("x"), 
                        })
                    })
                })
            })
        })
    })
});

\end{verbatim}

\section{XML} \label{chapter3_section2} \hypertarget{chapter3_section2}{}
The XML representation within an QEDEQ module looks a little bit different. Here all first list atoms are represented as the attribute {\tt ref} or {\tt id}.
So the above formula may look like the following.

\begin{verbatim}
<EQUI>
  <PREDCON ref="equal">
    <VAR id="y"/>
    <CLASS>
      <VAR id="x"/>
      <PREDVAR id="\phi">
        <VAR id="x"/>
      </PREDVAR>
    </CLASS>
  </PREDCON>
  <FORALL>
    <VAR id="z"/>
    <EQUI>
      <PREDCON ref="in">
        <VAR id="z"/>
        <VAR id="y"/>
      </PREDCON>
      <PREDCON ref="in">
        <VAR id="z"/>
        <CLASS>
          <VAR id="x"/>
          <PREDVAR id="\phi">
            <VAR id="x"/>
          </PREDVAR>
        </CLASS>
      </PREDCON>
    </EQUI>
  </FORALL>
</EQUI>
\end{verbatim}

Due to XSD restrictions for the XML document some error codes listed in Chapter~\ref{chapter0} will not occur. Instead the XML will be classified as invalid.


%% end of chapter Representations

\backmatter

\addcontentsline{toc}{chapter}{\indexname} \printindex

\end{document}

