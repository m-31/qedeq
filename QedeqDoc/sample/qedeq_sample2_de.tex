% -*- TeX:DE -*-
%%% ====================================================================
%%% @LaTeX-file    qedeq_sample2
%%% Generated from http://www.qedeq.org/0_04_07/doc/sample/qedeq_sample2.xml
%%% Generated at   2013-05-20 16:27:22.644
%%% ====================================================================

%%% Permission is granted to copy, distribute and/or modify this document
%%% under the terms of the GNU Free Documentation License, Version 1.2
%%% or any later version published by the Free Software Foundation;
%%% with no Invariant Sections, no Front-Cover Texts, and no Back-Cover Texts.

\documentclass[a4paper,german,10pt,twoside]{book}
\usepackage[german]{babel}
\usepackage{makeidx}
\usepackage{amsmath,amsthm,amssymb}
\usepackage{color}
\usepackage{xr}
\usepackage{tabularx}
\usepackage[bookmarks=true,bookmarksnumbered,bookmarksopen,
   unicode=true,colorlinks=true,linkcolor=webgreen,
   pagebackref=true,pdfnewwindow=true,pdfstartview=FitH]{hyperref}
\definecolor{webgreen}{rgb}{0,.5,0}
\usepackage{epsfig,longtable}
\usepackage{graphicx}
\usepackage[all]{hypcap}

\newtheorem{thm}{Theorem}
\newtheorem{cor}[thm]{Korollar}
\newtheorem{lem}[thm]{Lemma}
\newtheorem{prop}[thm]{Proposition}
\newtheorem{ax}{Axiom}
\newtheorem{rul}{Regel}

\theoremstyle{definition}
\newtheorem{defn}{Definition}
\newtheorem{idefn}[defn]{Initiale Definition}

\theoremstyle{remark}
\newtheorem{rem}[thm]{Bemerkung}
\newtheorem*{notation}{Notation}


\addtolength{\textheight}{7\baselineskip}
\addtolength{\topmargin}{-5\baselineskip}

\setlength{\parindent}{0pt}

\frenchspacing \sloppy

\makeindex


\title{Mathematisches Beispielmodul}
\author{
Michael Meyling
}

\begin{document}

\maketitle

\setlength{\parskip}{5pt plus 2pt minus 1pt}
\mbox{}
\vfill

\par
Die Quelle f{"ur} dieses Dokument ist hier zu finden:
\par
\url{http://www.qedeq.org/0_04_07/doc/sample/qedeq_sample2.xml}

\par
Die vorliegende Publikation ist urheberrechtlich gesch{"u}tzt.
\par
Bei Fragen, Anregungen oder Bitte um Aufnahme in die Liste der abh{\"a}ngigen Module schicken Sie bitte eine EMail an die Adresse \href{mailto:mime@qedeq.org}{mime@qedeq.org}

\par
Die Autoren dieses Dokuments sind:
Michael Meyling \href{mailto:michael@meyling.com}{michael@meyling.com}



\par
Benutzte QEDEQ module:

\par


\par
\textbf{l} qedeq\_logic\_v1 \url{http://www.qedeq.org/0_04_07/doc/math/qedeq_logic_v1_de.pdf}

\par
\textbf{s} qedeq\_set\_theory\_v1 \url{http://www.qedeq.org/0_04_07/doc/math/qedeq_set_theory_v1_de.pdf}


\setlength{\parskip}{0pt}
\tableofcontents

\setlength{\parskip}{5pt plus 2pt minus 1pt}

\chapter{Beispiel} \label{chapter1} \hypertarget{chapter1}{}

Wir verwenden die beiden QEDEQ-Hauptmodule und notieren eine einfache Proposition.

\section{Erstes und einziges} \label{chapter1_section1} \hypertarget{chapter1_section1}{}
Hier starten wir.

\par
Wir schreiben unsere Proposition, welche Predicate benutzt, die in importierten QEDEQ-Modulen definiert sind.

\begin{prop}
\label{proposition:one} \hypertarget{proposition:one}{}
{\tt \tiny [\verb]proposition:one]]}
\mbox{}
\begin{longtable}{{@{\extracolsep{\fill}}p{\linewidth}}}
\centering $\forall \ \mathfrak{M}(z)\ \ (z \in x\ \leftrightarrow\ z \in y)\ \rightarrow\ x \ = \ y$
\end{longtable}

\end{prop}



%% end of chapter Beispiel

\addcontentsline{toc}{chapter}{\indexname} \printindex

\end{document}

