% -*- TeX:EN -*-
%%% ====================================================================
%%% @LaTeX-file    qedeq_sample2
%%% Generated from http://www.qedeq.org/0_04_07/doc/sample/qedeq_sample2.xml
%%% Generated at   2013-05-20 16:27:22.632
%%% ====================================================================

%%% Permission is granted to copy, distribute and/or modify this document
%%% under the terms of the GNU Free Documentation License, Version 1.2
%%% or any later version published by the Free Software Foundation;
%%% with no Invariant Sections, no Front-Cover Texts, and no Back-Cover Texts.

\documentclass[a4paper,german,10pt,twoside]{book}
\usepackage[english]{babel}
\usepackage{makeidx}
\usepackage{amsmath,amsthm,amssymb}
\usepackage{color}
\usepackage{xr}
\usepackage{tabularx}
\usepackage[bookmarks=true,bookmarksnumbered,bookmarksopen,
   unicode=true,colorlinks=true,linkcolor=webgreen,
   pagebackref=true,pdfnewwindow=true,pdfstartview=FitH]{hyperref}
\definecolor{webgreen}{rgb}{0,.5,0}
\usepackage{epsfig,longtable}
\usepackage{graphicx}
\usepackage[all]{hypcap}

\newtheorem{thm}{Theorem}
\newtheorem{cor}[thm]{Corollary}
\newtheorem{lem}[thm]{Lemma}
\newtheorem{prop}[thm]{Proposition}
\newtheorem{ax}{Axiom}
\newtheorem{rul}{Rule}

\theoremstyle{definition}
\newtheorem{defn}{Definition}
\newtheorem{idefn}[defn]{Initial Definition}

\theoremstyle{remark}
\newtheorem{rem}[thm]{Remark}
\newtheorem*{notation}{Notation}


\addtolength{\textheight}{7\baselineskip}
\addtolength{\topmargin}{-5\baselineskip}

\setlength{\parindent}{0pt}

\frenchspacing \sloppy

\makeindex


\title{Mathematical Example Module}
\author{
Michael Meyling
}

\begin{document}

\maketitle

\setlength{\parskip}{5pt plus 2pt minus 1pt}
\mbox{}
\vfill

\par
The source for this document can be found here:
\par
\url{http://www.qedeq.org/0_04_07/doc/sample/qedeq_sample2.xml}

\par
Copyright by the authors. All rights reserved.
\par
If you have any questions, suggestions or want to add something to the list of modules that use this one, please send an email to the address \href{mailto:mime@qedeq.org}{mime@qedeq.org}

\par
The authors of this document are:
Michael Meyling \href{mailto:michael@meyling.com}{michael@meyling.com}



\par
Used other QEDEQ modules:

\par


\par
\textbf{l} qedeq\_logic\_v1 \url{http://www.qedeq.org/0_04_07/doc/math/qedeq_logic_v1_en.pdf}

\par
\textbf{s} qedeq\_set\_theory\_v1 \url{http://www.qedeq.org/0_04_07/doc/math/qedeq_set_theory_v1_en.pdf}


\setlength{\parskip}{0pt}
\tableofcontents

\setlength{\parskip}{5pt plus 2pt minus 1pt}

\chapter{Sample} \label{chapter1} \hypertarget{chapter1}{}

We just use the two main QEDEQ modules and note a simple proposition.

\section{First and only} \label{chapter1_section1} \hypertarget{chapter1_section1}{}
Here we start.

\par
We state our proposition which uses predicates that are defined in imported QEDEQ modules.

\begin{prop}
\label{proposition:one} \hypertarget{proposition:one}{}
{\tt \tiny [\verb]proposition:one]]}
\mbox{}
\begin{longtable}{{@{\extracolsep{\fill}}p{\linewidth}}}
\centering $\forall \ \mathfrak{M}(z)\ \ (z \in x\ \leftrightarrow\ z \in y)\ \rightarrow\ x \ = \ y$
\end{longtable}

\end{prop}



%% end of chapter Sample

\addcontentsline{toc}{chapter}{\indexname} \printindex

\end{document}

