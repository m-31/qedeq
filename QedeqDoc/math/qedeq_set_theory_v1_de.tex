% -*- TeX:DE -*-
%%% ====================================================================
%%% @LaTeX-file  qedeq_set_theory_v1_de.tex
%%% Generated at 2007-12-12 13:36:44,155
%%% ====================================================================

%%% Permission is granted to copy, distribute and/or modify this document
%%% under the terms of the GNU Free Documentation License, Version 1.2
%%% or any later version published by the Free Software Foundation;
%%% with no Invariant Sections, no Front-Cover Texts, and no Back-Cover Texts.

\documentclass[a4paper,german,10pt,twoside]{book}
\usepackage[german]{babel}
\usepackage{makeidx}
\usepackage{amsmath,amsthm,amssymb}
\usepackage{color}
\usepackage[bookmarks,bookmarksnumbered,bookmarksopen,
   colorlinks,linkcolor=webgreen,pagebackref]{hyperref}
\definecolor{webgreen}{rgb}{0,.5,0}
\usepackage{graphicx}
\usepackage{xr}
\usepackage{epsfig,longtable}
\usepackage{tabularx}

\newtheorem{thm}{Theorem}[chapter]
\newtheorem{cor}[thm]{Korollar}
\newtheorem{lem}[thm]{Lemma}
\newtheorem{prop}[thm]{Proposition}
\newtheorem{ax}{Axiom}
\newtheorem{rul}{Regel}

\theoremstyle{definition}
\newtheorem{defn}[thm]{Definition}
\newtheorem{idefn}[thm]{Initiale Definition}

\theoremstyle{remark}
\newtheorem{rem}[thm]{Bemerkung}
\newtheorem*{notation}{Notation}

\addtolength{\textheight}{7\baselineskip}
\addtolength{\topmargin}{-5\baselineskip}

\setlength{\parindent}{0pt}

\frenchspacing \sloppy

\makeindex


\title{Axiomatische Mengenlehre}
\author{
Michael Meyling
}

\begin{document}

\maketitle

\setlength{\parskip}{5pt plus 2pt minus 1pt}
\mbox{}
\vfill

\par
Die Quelle f{"ur} dieses Dokument ist hier zu finden:
\par
\url{http://qedeq.org/0_03_07/doc/math/qedeq_set_theory_v1.xml}

\par
Die vorliegende Publikation ist urheberrechtlich gesch{"u}tzt.
\par
Bei Fragen, Anregungen oder Bitte um Aufnahme in die Liste der abh{"a}ngigen Module schicken Sie bitte eine EMail an die Addresse \url{mailto:mime@qedeq.org}

\setlength{\parskip}{0pt}
\tableofcontents

\setlength{\parskip}{5pt plus 2pt minus 1pt}

\chapter*{Vorwort\label{ch:preface}} \label{chapter0} \hypertarget{chapter0}{}
\addcontentsline{toc}{chapter}{Vorwort\label{ch:preface}}

Mathematik ist eine Wissenschaft mit einer Struktur, die im Laufe der Zeit riesige Dimensionen erreicht hat. Diese unglaublich hohe Burg besitzt nur ein ganz schmales Fundament und ihre Festigkeit gr{\"u}ndet sich auf einfachen pr{\"a}dikatenlogischen M{\"o}rtel. Im Prinzip kann der Aufbau von jeder Mathematikerin verstanden werden. Von dem neuesten Gipfel mathematischer Erkenntnis kann jeder Pfad logisch folgerichtig bis in die mengentheoretischen Wurzeln nachvollzogen werden.

\par
Bei diesem Unternehmen will dieses Dokument Hilfestellung geben. Ziel ist eine Pr{\"a}sentation der mengentheoretischen Wurzeln in verst{\"a}ndlicher Weise. Bei aller Verst{\"a}ndlichkeit soll es jedoch jederzeit m{\"o}glich sein, tief in die Details einzusteigen. Ja sogar bis auf die Ebene eines formal korrekten Beweises hinab. Dazu gibt es dieses Dokument in verschiedenen Detailierungen. F{\"u}r alle aber gilt, dass die Formeln in Axiomen, Definitionen und Propositionen in formal korrekter Form vorliegen.

\par
Wir wollen bei den Wurzeln anfangen\ldots

\par
Dieses Dokument ist noch im Entstehen und wird von Zeit zu Zeit aktualisiert. Insbesondere werden an den durch {\glqq+++\grqq} gekennzeichneten Stellen noch Erg{\"a}nzungen oder {\"A}nderungen vorgenommen.

\par
Besondere Dank geht an meine Frau \emph{Gesine~Dr{\"a}ger} und unseren Sohn \emph{Lennart} f{\"u}r ihre Unterst{\"u}tzung und ihr Verst{\"a}ndnis f{\"u}r ihnen fehlende Zeit.

\par
\vspace*{1cm} Hamburg, Oktober 2007 \\
\hspace*{\fill} Michael Meyling

%% end of chapter Vorwort\label{ch:preface}

\chapter*{Einleitung\label{ch:introduction}} \label{chapter1} \hypertarget{chapter1}{}
\addcontentsline{toc}{chapter}{Einleitung\label{ch:introduction}}

Nachdem durch die Logik die Art der mathematischen Argumentation vorgegeben wird, wird in der Mengenlehre ganz allgemein {\"u}ber Objekte und ihre Zusammenfassungen gesprochen. Besonders interessant ist die Mengenlehre dadurch, dass sie zum einen von eigentlich allen mathematischen Disziplinen verwendet wird. Zum anderen l{\"a}sst sich jede mathematische Disziplin innerhalb der Mengenlehre definieren. Zahlentheorie, Algebra, Analysis und alle weiteren Gebiete lassen sich darauf aufbauen.

  \par
  Dieses Dokument beschreibt die mathematischen Grundlagen der Mengenlehre. Ziel ist dabei die Bereitstellung von elementaren Ergebnissen der Mengenlehre, die in anderen mathematischen Disziplinen ben{\"o}tigt werden. Nach den Grundlagen wird die Boolsche Algebra der Klassen betrachtet. Es schliessen sich Betrachtungen {\"u}ber Relationen und Funktionen an. Ein weiteres wichtiges Ergebnis sind die Definition der nat{\"u}rlichen Zahlen und die Erf{\"u}llung der Peano-Axiome durch diese, auch auf den Begriff der Rekursion wird eingegangen.
 
  \par
  Die Darstellung erfolgt in axiomatischer Weise soll aber im Ergebnis der mathematischen Praxis entsprechen. Daher wird auch das Axiomensystem der Mengenlehre von \emph{A.~P.~Morse} und  \emph{J.~L.~Kelley} (MK\index{MK}) verwendet.

%% end of chapter Einleitung\label{ch:introduction}

\chapter{Anfangsgr{\"u}nde} \label{chapter2} \hypertarget{chapter2}{}

In diesem Kapitel beginnen wir mit den ganz elementaren Axiomen und Definitionen der Mengenlehre. Wir versuchen nicht eine formale Sprache einzuf{\"u}hren\footnote{Dessen ungeachtet sind die Formeln der Axiome, Definitionen und Propositionen in dem Ursprungstext dieses Dokuments in einer formalen Sprache notiert. Der Ursprungstext ist eine XML-Datei, deren Syntax mittels der XSD \url{http://www.qedeq.org/current/xml/qedeq.xsd} definiert wird.} und setzen das Wissen um den Gebrauch von logischen Symbolen voraus. Noch genauer formuliert: wir arbeiten mit einer Pr{\"a}dikatenlogik erster Stufe mit Gleichheit.

\par\index{Cantor}\index{Menge!Definition}
\emph{G.~Cantor}, der als Begr{\"u}nder der Mengenlehre gilt, hat in einer Ver{\"o}ffentlichung im Jahre 1895 eine Beschreibung des Begriffs \emph{Menge} gegeben.

\begin{quote}
 Unter einer {\glqq Menge\grqq} verstehen wir jede Zusammenfassung $M$ von bestimmten wohlunterscheidbaren Objekten $m$ unserer Anschauung oder unseres Denkens (welche die {\glqq Elemente\grqq} von $M$ genannt werden) zu einem Ganzen.
\end{quote}

\par
Diese Zusammenfassung kann {\"u}ber die Angabe einer Eigenschaft dieser Elemente erfolgen. Um 1900 wurden verschiedene Widerspr{\"u}che dieser naiven Mengenlehre entdeckt. Diese Widerspr{\"u}che lassen sich auf trickreich gew{\"a}hlte Eigenschaften zur{\"u}ckf{\"u}hren.

\par
Es gibt verschiedene M{\"o}glichkeiten diese Widerspr{\"u}che zu verhindern. In diesem Text schr{\"a}nken wir zwar die Angabe von Eigenschaften in keiner Weise ein, aber wir nennen das Ergebnis der Zusammenfassung zun{\"a}chst einmal \emph{Klasse}. Zus{\"a}tzliche Axiome regeln dann, wann eine bestimmte Klasse auch eine Menge ist. Alle Mengen sind Klassen, aber nicht alle Klassen sind Mengen. Eine Menge ist eine Klasse, die selbst Element einer anderen Klasse ist. Eine Klasse, die keine Menge ist, ist nicht Element irgend einer anderen Klasse.

\section{Klassen und Mengen} \label{chapter2_section0} \hypertarget{chapter2_section0}{}
Obgleich wir im Wesentlichen {\"u}ber \emph{Mengen}\index{Menge} sprechen wollen, haben wir am Anfang nur \emph{Klassen}\index{Klasse}. Dieser Begriff wird nicht formal definiert. Anschaulich gesprochen, ist eine Klasse eine Zusammenfassung von Objekten. Die beteiligten Objekte heissen auch \emph{Elemente}\index{Element} der Klasse.
Mengen werden dann als eine besondere Art von Klassen charakterisiert.

\par
Die folgenden Definitionen und Axiome folgen dem Aufbau einer vereinfachten Version der Mengenlehre nach \emph{von~Neumann-Bernays-G{\"o}del} (\emph{NBG}\index{NBG}). Die genaue Bezeichnung lautet \emph{MK}\index{MK} nach \emph{Morse-Kelley}.

\par
Die hier vorgestellte Mengenlehre hat als Ausgangsobjekte \emph{Klassen}.
Weiterhin wird nur ein einziges Symbol f{\"u}r eine bin{\"a}re Relation vorausgesetzt: der \emph{Enthaltenseinoperator}.

\begin{idefn}[Elementbeziehung]
\label{definition:in} \hypertarget{definition:in}{}
$$x \in y$$

\end{idefn}

Wir sagen auch $x$ \emph{ist Element von} $y$, $x$ \emph{geh{\"o}rt zu} $y$, $x$ \emph{liegt in} $y$, $x$ \emph{ist in} $y$.
Neben der Identit{\"a}t ist dies das einzige Pr{\"a}dikat welches wir zu Beginn haben. Alle anderen werden definiert.\footnote{Das Gleichheitspr{\"a}dikat k{\"o}nnte auch innerhalb der Mengenlehre definiert werden, aber dann wird auch ein weiters Axiom ben{\"o}tigt und es ergeben sich technischen Schwierigkeiten bei der Herleitung der Gleichheitsaxiome.} Zu Anfang haben wir auch noch keine Funktionskonstanten.


\par
Obgleich wir die Elementbeziehung einfach negieren k{\"o}nnen, m{\"o}chten wir daf{\"u}r eine Abk{\"u}rzung definieren.

\begin{defn}[Negation der Elementbeziehung]
\label{definition:notIn} \hypertarget{definition:notIn}{}
$$x \notin y\ :\leftrightarrow \ \neg x \in y$$

\end{defn}




\par
Unser erstes Axiom besagt, dass beliebige Klassen $x$ und $y$ genau dann gleich sind, wenn sie dieselben Elemente enthalten.\footnote{Falls wir das Gleichheitspr{\"a}dikat nicht als logisches Symbol voraussetzen w{\"u}rden, dann w{\"u}rden wir es hiermit definieren.}

\begin{ax}[Extensionalit{\"a}t\index{Extensionalit{\"a}tsaxiom}]
\label{axiom:extensionality} \hypertarget{axiom:extensionality}{}
\mbox{}
\begin{longtable}{{@{\extracolsep{\fill}}p{\linewidth}}}
\centering $\forall z\ (z \in x\ \leftrightarrow \ z \in y)\ \rightarrow \ x \ =  \ y$
\end{longtable}

\end{ax}

Die Klassen $x$ and $y$ k{\"o}nnen verschieden definiert sein, beispielsweise:
\par
\begin{tabularx}{\linewidth}{rcX}
  $x$ & = & Klasse aller nichtnegativen ganzen Zahlen, \\
  $y$ & = & Klasse aller ganzen Zahlen, die als Summe von vier Quadraten geschreiben werden k{\"o}nnen,
\end{tabularx}
\par
aber wenn sie dieselben Elemente besitzen, sind sie gleich.


\par
Jetzt legen wir fest, was eine \emph{Menge} ist.

\begin{defn}[Menge\index{Menge!Definition}]
\label{definition:isSet} \hypertarget{definition:isSet}{}
$$\mathfrak{M}(x)\ :\leftrightarrow \ \exists y\ x \in y$$

\end{defn}

Mengen sind also nichts anderes, als Klassen mit einer besonderen Eigenschaft. Eine Klasse ist genau dann eine Menge, wenn sie Element irgendeiner Klasse ist.


\par
Als erste Folgerung aus dem Extensionalit{\"a}tsaxiom erhalten wir das Folgende.\footnote{Es wird ein eingeschr{\"a}nkter Allquantor benutzt, $z$ l{\"a}uft nur {\"u}ber Mengen.}

\begin{prop}
\label{theorem:extensionalitySetRestricted} \hypertarget{theorem:extensionalitySetRestricted}{}
\mbox{}
\begin{longtable}{{@{\extracolsep{\fill}}p{\linewidth}}}
\centering $\forall \ \mathfrak{M}(z)\ \ (z \in x\ \leftrightarrow \ z \in y)\ \rightarrow \ x \ =  \ y$
\end{longtable}

\end{prop}
\begin{proof}
Angenommen es gelte $\forall \ \mathfrak{M}(z) \ ( z \in x \ \leftrightarrow \ z \in y)$. Sei $z$ eine beliebige Klasse. Falls $z \in x$ dann gilt $z$ ist eine Menge nach Definition~\ref{definition:isSet}, und daraus folgt mit der Annahme $z \in y$. Analog folgt $z \in y \ \rightarrow \ z \in x$. Da $z$ beliebig, haben wir $\forall z \ (z \in x \ \leftrightarrow \ z \in y)$. Und mit dem  Extensionalit{\"a}tsaxiom~\ref{axiom:extensionality} erhalten wir daraus $x = y$.
\end{proof}




\par
Weiterhin k{\"o}nnen wir in dem Extensionalit{\"a}tsaxiom die Implikation umkehren.

\begin{prop}
\label{theorem:extensonalityEquivalence} \hypertarget{theorem:extensonalityEquivalence}{}
\mbox{}
\begin{longtable}{{@{\extracolsep{\fill}}p{\linewidth}}}
\centering $x \ =  \ y\ \leftrightarrow \ \forall z\ (z \in x\ \leftrightarrow \ z \in y)$
\end{longtable}

\end{prop}
\begin{proof}
Dies ist eine einfache Anwendung des zweiten identit{\"a}tslogischen Axioms.
\end{proof}




\par
Unser n{\"a}chstes Axiom der Mengenlehre erm{\"o}glicht uns in simpler Art und Weise neue Klassen zu bilden. Eine Klasse wird ganz einfach durch die Angabe einer pr{\"a}dikatenlogischen Formel charakterisiert.

\begin{ax}[Komprehension]
\label{axiom:comprehension} \hypertarget{axiom:comprehension}{}
\mbox{}
\begin{longtable}{{@{\extracolsep{\fill}}p{\linewidth}}}
\centering $\exists x\ \forall y\ (y \in x\ \leftrightarrow \ (\mathfrak{M}(y)\ \land \ \phi(y)))$
\end{longtable}

\end{ax}

\index{NBG}Durch eine kleine {\"A}nderung dieses Axioms w{\"u}rden wir im Folgenden ein \emph{NBG}-Axiomensystem der Mengenlehre erhalten, welches auf \emph{John von Neumann}, \emph{Isaak Bernays} und \emph{Kurt G{\"o}del} zur{\"u}ckgeht.\index{pr{\"a}dikativ}\index{Formel, pr{\"a}dikative}
Dazu definieren wir: eine Formel, in der alle gebundenen Subjektvariablen auf Mengen restringiert sind wird \emph{pr{\"a}dikative Formel} genannt. Pr{\"a}dikative Formeln formalisieren also diejenigen Eigenschaften, die man als {\glqq Eigenschaften von Mengen\grqq} bezeichnen kann.\footnote{Noch etwas formaler: in einer pr{\"a}dikativen Formel laufen alle Quantorenvariablen nur {\"u}ber Mengen: $\forall \ \mathfrak{M}(x) \ \exists \ \mathfrak{M}(y) \ldots$}
Fordern wir nun also zus{\"a}tzlich, dass $\phi$ pr{\"a}dikativ sein muss, dann erhalten wir im zusammen mit den folgenden Axiomen ein NBG-System.


\par
Durch das Komprehensionsaxiom und die Extensionalit{\"a}t wird nun der Zusammenhang zwischen einer Aussage $\phi(y)$ und der durch sie definierten Klasse festgelegt. Dabei behauptet das Komprehensionsaxiom die Existenz mindestens einer Klasse, deren Elemente die Aussage $\mathfrak{M}(y) \land \phi(y)$ erf{\"u}llen. Das Extensionalit{\"a}tsaxiom und die Gleichheitsaxiome sichern ab, dass es h{\"o}chstens eine solche Klasse gibt: irgend zwei Klassen, welche dieselben Elemente besitzen, sind gleich im Sinne der Ersetzbarkeit in allen einschl{\"a}gigen Aussagen. Mit anderen Worten: es gibt nur genau eine solche Klasse.

\begin{prop}
\label{theorem:comprehension} \hypertarget{theorem:comprehension}{}
\mbox{}
\begin{longtable}{{@{\extracolsep{\fill}}p{\linewidth}}}
\centering $\exists! x\ \forall y\ (x \in y\ \leftrightarrow \ (\mathfrak{M}(y)\ \land \ \phi(y)))$
\end{longtable}

\end{prop}
\begin{proof}
Zu zeigen ist:
$$
\begin{array}{rl}
\exists x                      & \forall y \ (y \in x \leftrightarrow  \ \mathfrak{M}(y) \land \phi(y)) \\
\land \ \forall u \ \forall v  & (\forall y \ (y \in u \leftrightarrow \mathfrak{M}(y) \land \phi(y)) \ \land  \ \forall y \ ( y \in v \leftrightarrow \ \mathfrak{M}(y) \land \phi(y))) \\
       & \qquad \rightarrow u = v)
\end{array}
$$
Seien $u$ und $v$ beliebig. Es gelte weiterhin:

\par
$\forall y \ (y \in u \leftrightarrow \ \mathfrak{M}(y) \land \phi(y)) \land \ \forall y \ ( y \in v \leftrightarrow \mathfrak{M}(y)
\land \phi(y)))$

\par
Dann folgt mit \hyperref{http://www.qedeq.org/current/doc/math/qedeq_logic_v1_de.pdf}{}{allandpp}{Formel 3.37}: $\forall y \ ((y \in u \leftrightarrow \mathfrak{M}(y) \land \phi(y)) \land (y \in v \leftrightarrow \mathfrak{M}(y) \land \phi(y)))$

\par
Daraus erhalten wir mit \hyperref{http://www.qedeq.org/current/doc/math/qedeq_logic_v1_de.pdf}{}{andequi}{Formel 3.31}: $\forall y \ ((y \in u \leftrightarrow y \in v ))$. Und mit Proposition~\ref{theorem:extensonalityEquivalence} folgt nun $u = v$. Also haben wir gezeigt:

\par
$\forall u \ \forall v \ (\forall y \ (y \in u \leftrightarrow \mathfrak{M}(y) \land \phi(y)) \land \ \forall y \ (y \in v \leftrightarrow \mathfrak{M}(y) \land \phi(y))) \rightarrow u = v)$

\par
Zusammen mit Axiom~\ref{axiom:comprehension} folgt nun die Behauptung.
\end{proof}




\par
Ausgehend von \ref{theorem:comprehension} k{\"o}nnen wir die Sprachsyntax erweitern und eine neue abk{\"u}rzende Schreibweise einf{\"u}hren.

\begin{rul}[Klassenschreibweise]
\label{rule:classDefinition} \hypertarget{rule:classDefinition}{}
F{\"u}r jede Formel $\alpha(x)$ definieren wir den Termausdruck $\{ x \ | \ \alpha(x)\}$ durch 
$$
  \exists y \ (y = \{x \ | \ \alpha(x) \} \ \land \ \forall x \ x \in y \ \leftrightarrow \mathfrak{M}(x) \ \land \alpha(x))
$$

\par
Die freien Variablen von $\{ x \ | \ \alpha(x)\}$ sind die freien Variablen von $\alpha(x)$ vermindert um $\{ x \}$. Die gebunden Variablen entsprechen den gebunden Variablen von $\alpha(x)$. Alle Ableitungsregln werden entsprechend erweitert.
                
%old                
%                  Der Ausdruck $\{ x | \alpha(x)\}$ bezeichnet ebenfalls einen Term. Sei $\beta(y)$ eine Formel und $beta(\{x | \alpha(x)\}$ die Formel, die bei Ersetzung von y durch den angegebenen Term entstehende Formel. Diese Formel kann ersetzt werden durch $\beta(y) \ \land \ \forall x \ x \in y \leftrightarrow \mathfrak{M}(x) \ \land \ \alpha(y)$. 
%                  \par
%                  In dem Term $\{ x | \alpha(x)\}$ gilt $x$ als gebunden und der Term besitzt zus{\"a}tzlich die gebunden Variablen von $\alpha$ als gebundene Variablen. Die freien Variablen von $\alpha$ sind auch die freien Variablen des neuen Termausdrucks.
\end{rul}
Basierend auf: 
 \ref{theorem:comprehension}

Insbesondere m{\"u}ssen die Substitutionsregeln angepasst werden.\footnote{Weil nun ein Term auch gebundene Subjektvariablen besitzen kann. Gl{\"u}cklicherweise haben wir das jedoch schon bei der Formulierung unserer Substitutionsregeln ber{\"u}cksichtigt, so dass wir nicht s tun m{\"u}ssen.} Es handelt sich hierbei um eine \emph{konservative} Erweiterung\footnote{Unter einer konservativen Erweiterung  verstehen wir das Folgende: Sei $\mathfrak{L}$ eine Sprache und $\mathfrak{L'}$ eine Erweiterung von $\mathfrak{L}$. Da $\mathfrak{L'} \supset \mathfrak{L}$ gilt auch $\mbox{Formel}_\mathfrak{L'} \supset \mbox{Formel}_\mathfrak{L}$. Falls nun f{\"u}r jede Formelmenge $\Gamma \subseteq \mbox{Formel}_\mathfrak{L}$ und jede Formel $\alpha \in \mbox{Formel}_\mathfrak{L}$ gilt: $\Gamma \vdash_\mathfrak{L'} \alpha \ \Rightarrow \ \Gamma \vdash_\mathfrak{L} \alpha$, dann hei{\ss}t $\mathfrak{L'}$ eine \emph{konservative} Erweiterung von $\mathfrak{L}$.}.

%old                
%                { x | alpha (x)} = { x | alpha (x)}
%                y = y n All x x in y <-> M(x) n alpha(x)
                
                Im Folgenden wird auf diese Schreibweise zur{\"u}ckgegriffen.


\par
Die neue Schreibweise kann auch in einfacher Weise in die alte Syntax transformiert werden.
Die G{\"u}ltigkeit der Ausgangspr{\"a}dikate dr{\"u}ckt sich f{\"u}r diese neue Termart wie folgt aus.

\begin{prop}
\label{theorem:setNotation} \hypertarget{theorem:setNotation}{}
\mbox{}
\begin{longtable}{{@{\extracolsep{\fill}}p{0.9\linewidth}l}}
\centering $y \in \{ x \ | \ \phi(x) \} \ \leftrightarrow \ (\mathfrak{M}(y)\ \land \ \phi(y))$ & \label{theorem:setNotation:a} \hypertarget{theorem:setNotation:a}{} \mbox{\emph{(a)}} \\
\centering $y \ =  \ \{ x \ | \ \phi(x) \} \ \leftrightarrow \ \forall z\ (z \in y\ \leftrightarrow \ z \in \{ x \ | \ \phi(x) \} )$ & \label{theorem:setNotation:b} \hypertarget{theorem:setNotation:b}{} \mbox{\emph{(b)}} \\
\centering $\{ x \ | \ \phi(x) \}  \ =  \ \{ x \ | \ \psi(x) \} \ \leftrightarrow \ \forall z\ (z \in \{ x \ | \ \phi(x) \} \ \leftrightarrow \ z \in \{ x \ | \ \psi(x) \} )$ & \label{theorem:setNotation:c} \hypertarget{theorem:setNotation:c}{} \mbox{\emph{(c)}} \\
\centering $\{ x \ | \ \phi(x) \}  \in \{ x \ | \ \psi(x) \} \ \leftrightarrow \ \forall u\ \forall v\ ((u \ =  \ \{ x \ | \ \phi(x) \} \ \land \ v \ =  \ \{ x \ | \ \psi(x) \} )\ \rightarrow \ u \in v)$ & \label{theorem:setNotation:d} \hypertarget{theorem:setNotation:d}{} \mbox{\emph{(d)}} \\
\centering $\{ x \ | \ \phi(x) \}  \in y\ \leftrightarrow \ \forall u\ (u \ =  \ \{ x \ | \ \phi(x) \} \ \rightarrow \ u \in y)$ & \label{theorem:setNotation:e} \hypertarget{theorem:setNotation:e}{} \mbox{\emph{(e)}} 
\end{longtable}
+++ wenn diese Formel richtig gesetzt w{\"u}rde, sollte sie so aussehen:
\begin{align}
y \in \{ x~|~\phi(x) \} & \leftrightarrow  \mathfrak{M}(y) \land \phi(y) \tag{a} \\
y = \{ x~|~ \phi(x) \} & \leftrightarrow  \forall z \ (z \in y \leftrightarrow z \in \{ x~|~\phi(x) \}) \tag{b} \\
\{ x~|~\phi(x) \} = \{ x~|~\psi(x) \} & \leftrightarrow \forall z \ (z \in \{ x~|~\phi(x) \} \tag{c} \\
\begin{split}
  & \qquad \leftrightarrow z \in \{x~|~\psi(x) \}) \nonumber \\
\{ x~|~\phi(x) \} \in \{ x~|~\psi(x) \} & \leftrightarrow  \forall u \ \forall
v \ ((u  = \{ x~|~\phi(x) \} \\
  & \qquad \land \ v = \{ x~|~\psi(x) \}) \rightarrow u \in v) 
\end{split} \tag{d} \\
\{ x~|~\phi(x) \} \in y & \leftrightarrow  \forall u \ (u  = \{ x~|~\phi(x) \}  \rightarrow u \in y) \tag{e} 
\end{align}
\end{prop}
\begin{proof}
+++ fehlt noch.
\end{proof}

Durch sukzessives Anwenden des obigen Satzes kann also die neue Syntax in die alte {\"u}berf{\"u}hrt werden.


\par
Da durch die neue Schreibweise ein Term eindeutig festgelegt wird, muss nat{\"u}rlich auch das Folgende gelten.

\begin{prop}
\label{theorem:setDefinitionUnique} \hypertarget{theorem:setDefinitionUnique}{}
\mbox{}
\begin{longtable}{{@{\extracolsep{\fill}}p{\linewidth}}}
\centering $\exists! x\ x \ =  \ \{ y \ | \ \phi(y) \} $
\end{longtable}

\end{prop}




\par
Aus der {\"A}quivalenz von Aussageformen kann auf die Gleichheit der daraus gebildeten Klassen geschlossen werden.

\begin{prop}
\label{theorem:propositionEqualImplClassEqual} \hypertarget{theorem:propositionEqualImplClassEqual}{}
\mbox{}
\begin{longtable}{{@{\extracolsep{\fill}}p{\linewidth}}}
\centering $\forall x\ (\phi(x)\ \leftrightarrow \ \psi(x))\ \rightarrow \ \{ y \ | \ \phi(y) \}  \ =  \ \{ y \ | \ \psi(y) \} $
\end{longtable}

\end{prop}

Die Umkehrung gilt jedoch nicht.


\par
Jede Klasse l{\"a}sst sich durch eine Aussage beschreiben, indem auf ihre Elemente Bezug genommen wird.

\begin{prop}
\label{theorem:classDescriptionPossible} \hypertarget{theorem:classDescriptionPossible}{}
\mbox{}
\begin{longtable}{{@{\extracolsep{\fill}}p{\linewidth}}}
\centering $x \ =  \ \{ y \ | \ y \in x \} $
\end{longtable}

\end{prop}




\section{Spezielle Klassen} \label{chapter2_section1} \hypertarget{chapter2_section1}{}
In diesem Abschnitt definieren wir die ersten Klassen.

\par
Die Russellsche Klasse kann nun einfach definiert werden.

\begin{defn}[Russell-Klasse\index{Russell!-sche Klasse}\index{Klasse!Russellsche}]
\label{definition:RussellClass} \hypertarget{definition:RussellClass}{}
$$\mathfrak{Ru}\ := \ \{ x \ | \ x \notin x \} $$

\end{defn}




\par
Die Russellsche Klasse ist eine \emph{echte}\index{echt}\index{Klasse!echte} Klasse, d.~h. sie ist keine Menge.

\begin{prop}
\label{theorem:RussellClassIsClass} \hypertarget{theorem:RussellClassIsClass}{}
\mbox{}
\begin{longtable}{{@{\extracolsep{\fill}}p{\linewidth}}}
\centering $\neg \mathfrak{M}(\mathfrak{Ru})$
\end{longtable}

\end{prop}




\par
Die \emph{Allklasse} soll alles m{\"o}gliche umfassen.

\begin{defn}[Allklasse\index{Klasse!All-}\index{Allklasse}]
\label{definition:universalClass} \hypertarget{definition:universalClass}{}
$$\mathfrak{V}\ := \ \{ x \ | \ x \ =  \ x \} $$

\end{defn}




\par
Zumindest umfasst die Allklasse alle Mengen.

\begin{prop}
\label{theorem:universalClassContainsAllSets} \hypertarget{theorem:universalClassContainsAllSets}{}
\mbox{}
\begin{longtable}{{@{\extracolsep{\fill}}p{\linewidth}}}
\centering $\mathfrak{V} \ =  \ \{ x \ | \ \mathfrak{M}(x) \} $
\end{longtable}

\end{prop}




\par
Mitgliedschaft in der Allklasse ist daher gleichbedeutet mit der 
Eigenschaft eine Menge zu sein.

\begin{prop}
\label{theorem:isInUniversalClass} \hypertarget{theorem:isInUniversalClass}{}
\mbox{}
\begin{longtable}{{@{\extracolsep{\fill}}p{\linewidth}}}
\centering $x \in \mathfrak{V}\ \leftrightarrow \ \mathfrak{M}(x)$
\end{longtable}

\end{prop}




\par
Entsprechend definieren wir die \emph{leere Klasse}. Sp{\"a}ter werden wir feststellen, dass die leere Klasse
eine Menge ist. Dazu ben{\"o}tigen wir jedoch weitere Mengenaxiome. Wir bezeichnen diese Klasse jedoch
schon jetzt mit den Worten \emph{leere Menge}.

\begin{defn}[Leere Klasse\index{leere Klasse}\index{leere Menge}\index{Klasse!leere}\index{Menge!leere}]
\label{definition:emptySet} \hypertarget{definition:emptySet}{}
$$\emptyset\ := \ \{ x \ | \ x \ \neq  \ x \} $$

\end{defn}




\par
Keine Klasse ist Element der leeren Klasse.

\begin{prop}
\label{theorem:noClassIsMemberOfEmptySet} \hypertarget{theorem:noClassIsMemberOfEmptySet}{}
\mbox{}
\begin{longtable}{{@{\extracolsep{\fill}}p{\linewidth}}}
\centering $\forall z\ z \notin \emptyset$
\end{longtable}

\end{prop}




\par
Eine Klasse, welche keine Elemente besitzt, ist die leere Klasse.

\begin{prop}
\label{theorem:noClassIsMemberIsEmptySet} \hypertarget{theorem:noClassIsMemberIsEmptySet}{}
\mbox{}
\begin{longtable}{{@{\extracolsep{\fill}}p{\linewidth}}}
\centering $\forall z\ z \notin x\ \leftrightarrow \ x \ =  \ \emptyset$
\end{longtable}

\end{prop}





%% end of chapter Anfangsgr{\"u}nde

\chapter{Boolesche Algebra der Klassen\index{Boolesche Algebra!der Klassen}} \label{chapter3} \hypertarget{chapter3}{}

Die elementaren Operationen von Klassen und ihre Eigenschaften werden nun beschrieben.

Eine Boolesche Algebra, oft auch Boolescher Verband genannt, ist eine spezielle algebraische
Struktur, die die Eigenschaften der logischen Operatoren \emph{und}, \emph{oder}, \emph{nicht}
sowie die Eigenschaften der mengentheoretischen Verkn{\"u}pfungen \emph{Durchschnitt},
\emph{Vereinigung} und \emph{Komplement} abstrahiert.

\par
Sie ist benannt nach \emph{G. Boole}, der sie in der Mitte des 19. Jahrhunderts definierte, um
algebraische Methoden in der Aussagenlogik anwenden zu k{\"o}nnen.

\section{Boolesche Klassenoperatoren} \label{chapter3_section0} \hypertarget{chapter3_section0}{}
Die Schreibweise bzw. Regel~\ref{rule:classDefinition} erm{\"o}glicht die Definition von Klassenperatoren mithilfe
der logischen Verkn{\"u}pfungen.

\par
Die Vereinigung zweier Klassen besteht aus den Elementen beider Klassen.

\begin{defn}[Vereinigung\index{Vereinigung}\index{Klasse!Vereinigung}]
\label{definition:union} \hypertarget{definition:union}{}
$$(x \cup y)\ := \ \{ z \ | \ (z \in x\ \lor \ z \in y) \} $$

\end{defn}




\par
Entsprechend wird der Durchschnitt zweier Klassen definiert, als Klasse die aus den Elementen besteht, die in beiden Klassen vorhanden sind.

\begin{defn}[Durchschnitt\index{Durchschnitt}\index{Klasse!Durchschnitt}]
\label{definition:intersection} \hypertarget{definition:intersection}{}
$$(x \cap y)\ := \ \{ z \ | \ (z \in x\ \land \ z \in y) \} $$

\end{defn}




\par
Auch das Komplement einer Klasse kann einfach definiert werden.

\begin{defn}[Komplement\index{Komplement}\index{Klasse!Komplement}]
\label{definition:complement} \hypertarget{definition:complement}{}
$$\overline{x}\ := \ \{ z \ | \ z \notin x \} $$

\end{defn}




\par
Ob eine Menge ein Element der Vereinigung zweier Klassen ist, kann nat{\"u}rlich auch direkt angegeben werden.

\begin{prop}
\label{theorem:unionMember} \hypertarget{theorem:unionMember}{}
\mbox{}
\begin{longtable}{{@{\extracolsep{\fill}}p{\linewidth}}}
\centering $z \in (x \cup y)\ \leftrightarrow \ (z \in x\ \lor \ z \in y)$
\end{longtable}

\end{prop}




\par
Entsprechendes gilt f{\"u}r den Durchschnitt zweier Klassen.

\begin{prop}
\label{theorem:intersectionMember} \hypertarget{theorem:intersectionMember}{}
\mbox{}
\begin{longtable}{{@{\extracolsep{\fill}}p{\linewidth}}}
\centering $z \in (x \cap y)\ \leftrightarrow \ (z \in x\ \land \ z \in y)$
\end{longtable}

\end{prop}




\par
Analoges gilt f{\"u}r das Komplement, dort muss jedoch die Mengeneigenschaft explizit abgepr{\"u}ft werden.

\begin{prop}
\label{theorem:complementMember} \hypertarget{theorem:complementMember}{}
\mbox{}
\begin{longtable}{{@{\extracolsep{\fill}}p{\linewidth}}}
\centering $z \in \overline{x}\ \leftrightarrow \ (\mathfrak{M}(z)\ \land \ z \notin x)$
\end{longtable}

\end{prop}




\par
Die vorherigen S{\"a}tze zeigen die {\"U}bertragbarkeit der Eigenschaften der logischen Verkn{\"u}pfungen $\lor$, $\land$ und $\neg$ auf die Klassenverkn{\"u}pfungen $\cup$, $\cap$ und $\bar{~}$. Deshalb lassen sich die entsprechenden logischen Gesetzm{\"a}ssigkeiten direkt auf die Klassenverkn{\"u}pfungen {\"u}bertragen.

\begin{prop}
\label{theorem:unionIntersectionComplement} \hypertarget{theorem:unionIntersectionComplement}{}
\mbox{}
\begin{longtable}{{@{\extracolsep{\fill}}p{0.9\linewidth}l}}
\centering $(x \cup y) \ =  \ (y \cup x)$ & \label{theorem:unionIntersectionComplement:a} \hypertarget{theorem:unionIntersectionComplement:a}{} \mbox{\emph{(a)}} \\
\centering $(x \cap y) \ =  \ (y \cap x)$ & \label{theorem:unionIntersectionComplement:b} \hypertarget{theorem:unionIntersectionComplement:b}{} \mbox{\emph{(b)}} \\
\centering $((x \cup y) \cup z) \ =  \ (x \cup (y \cup z))$ & \label{theorem:unionIntersectionComplement:c} \hypertarget{theorem:unionIntersectionComplement:c}{} \mbox{\emph{(c)}} \\
\centering $((x \cap y) \cap z) \ =  \ (x \cap (y \cap z))$ & \label{theorem:unionIntersectionComplement:d} \hypertarget{theorem:unionIntersectionComplement:d}{} \mbox{\emph{(d)}} \\
\centering $x \ =  \ (x \cup x)$ & \label{theorem:unionIntersectionComplement:e} \hypertarget{theorem:unionIntersectionComplement:e}{} \mbox{\emph{(e)}} \\
\centering $x \ =  \ (x \cap x)$ & \label{theorem:unionIntersectionComplement:f} \hypertarget{theorem:unionIntersectionComplement:f}{} \mbox{\emph{(f)}} \\
\centering $\overline{\overline{x}} \ =  \ x$ & \label{theorem:unionIntersectionComplement:g} \hypertarget{theorem:unionIntersectionComplement:g}{} \mbox{\emph{(g)}} \\
\centering $\overline{(x \cup y)} \ =  \ (\overline{x} \cap \overline{y})$ & \label{theorem:unionIntersectionComplement:h} \hypertarget{theorem:unionIntersectionComplement:h}{} \mbox{\emph{(h)}} \\
\centering $\overline{(x \cap y)} \ =  \ (\overline{x} \cup \overline{y})$ & \label{theorem:unionIntersectionComplement:i} \hypertarget{theorem:unionIntersectionComplement:i}{} \mbox{\emph{(i)}} \\
\centering $(x \cup (y \cap z)) \ =  \ ((x \cup y) \cap (x \cup z))$ & \label{theorem:unionIntersectionComplement:j} \hypertarget{theorem:unionIntersectionComplement:j}{} \mbox{\emph{(j)}} \\
\centering $(x \cap (y \cup z)) \ =  \ ((x \cap y) \cup (x \cap z))$ & \label{theorem:unionIntersectionComplement:k} \hypertarget{theorem:unionIntersectionComplement:k}{} \mbox{\emph{(k)}} \\
\centering $\overline{\emptyset} \ =  \ \mathfrak{V}$ & \label{theorem:unionIntersectionComplement:l} \hypertarget{theorem:unionIntersectionComplement:l}{} \mbox{\emph{(l)}} \\
\centering $\overline{\mathfrak{V}} \ =  \ \emptyset$ & \label{theorem:unionIntersectionComplement:m} \hypertarget{theorem:unionIntersectionComplement:m}{} \mbox{\emph{(m)}} \\
\centering $(x \cap \mathfrak{V}) \ =  \ x$ & \label{theorem:unionIntersectionComplement:n} \hypertarget{theorem:unionIntersectionComplement:n}{} \mbox{\emph{(n)}} \\
\centering $(x \cap \emptyset) \ =  \ \emptyset$ & \label{theorem:unionIntersectionComplement:o} \hypertarget{theorem:unionIntersectionComplement:o}{} \mbox{\emph{(o)}} \\
\centering $(x \cup \mathfrak{V}) \ =  \ \mathfrak{V}$ & \label{theorem:unionIntersectionComplement:p} \hypertarget{theorem:unionIntersectionComplement:p}{} \mbox{\emph{(p)}} \\
\centering $(x \cup \emptyset) \ =  \ x$ & \label{theorem:unionIntersectionComplement:q} \hypertarget{theorem:unionIntersectionComplement:q}{} \mbox{\emph{(q)}} \\
\centering $(x \cup \overline{x}) \ =  \ \mathfrak{V}$ & \label{theorem:unionIntersectionComplement:r} \hypertarget{theorem:unionIntersectionComplement:r}{} \mbox{\emph{(r)}} \\
\centering $(x \cap \overline{x}) \ =  \ \emptyset$ & \label{theorem:unionIntersectionComplement:s} \hypertarget{theorem:unionIntersectionComplement:s}{} \mbox{\emph{(s)}} 
\end{longtable}

\end{prop}




\section{Boolsche Algebra\index{Boolesche Algebra}} \label{chapter3_section1} \hypertarget{chapter3_section1}{}
Die Klassen bilden mit den Operatoren $\cap$, $\cup$, $\bar{~}$ und den Konstanten $\emptyset$,
$\mathfrak{V}$ eine Boolesche Algebra\index{Boolesche Algebra}.
\par
+++ 
Referenzen zu Kommutativit{\"a}t, Assoziativit{\"a}t, Distributivit{\"a}t, Idempotenz, etc.

\section{Ordnung\index{Ordnung}} \label{chapter3_section2} \hypertarget{chapter3_section2}{}
F{\"u}r eine boolsche Algebra kann eine kanonische \emph{Teilordnung}\index{Teilordnung} definiert werden. Daher
k{\"o}nnen wir auch f{\"u}r die Klassenalgebra eine Teilordnung festlegen.

\par
Wir definieren die \emph{Teilklassenrelation} durch eine Schnittklassenbildung.

\begin{defn}[Teilklasse\index{Teilklasse}]
\label{definition:subclass} \hypertarget{definition:subclass}{}
$$x \ \subseteq \ y\ :\leftrightarrow \ (x \cap y) \ =  \ x$$

\end{defn}

Sind $x$ und $y$ Mengen sagen wir auch: $x$ ist \emph{Teilmenge}\index{Teilmenge} von $y$.


\par
Die {\"u}bliche Definition der Teilklassenrelation erhalten wir nun als Satz.

\begin{prop}
\label{theorem:subsetIfMemberschipImpl} \hypertarget{theorem:subsetIfMemberschipImpl}{}
\mbox{}
\begin{longtable}{{@{\extracolsep{\fill}}p{\linewidth}}}
\centering $x \ \subseteq \ y\ \leftrightarrow \ \forall z\ (z \in x\ \rightarrow \ z \in y)$
\end{longtable}

\end{prop}




\par
Diese Relation ist reflexiv, transitiv und antisymmetrisch, definiert also eine Teilordnung\index{Teilordnung} mit $\emptyset$ als kleinstem und $\mathfrak{V}$ als gr{\"o}{\ss}tem Element.

\begin{prop}
\label{theorem:subsetIsPartialOrdered} \hypertarget{theorem:subsetIsPartialOrdered}{}
\mbox{}
\begin{longtable}{{@{\extracolsep{\fill}}p{0.9\linewidth}l}}
\centering $x \ \subseteq \ x$ & \label{theorem:subsetIsPartialOrdered:a} \hypertarget{theorem:subsetIsPartialOrdered:a}{} \mbox{\emph{(a)}} \\
\centering $(x \ \subseteq \ y\ \land \ y \ \subseteq \ z)\ \rightarrow \ x \ \subseteq \ z$ & \label{theorem:subsetIsPartialOrdered:b} \hypertarget{theorem:subsetIsPartialOrdered:b}{} \mbox{\emph{(b)}} \\
\centering $(x \ \subseteq \ y\ \land \ y \ \subseteq \ x)\ \leftrightarrow \ x \ =  \ y$ & \label{theorem:subsetIsPartialOrdered:c} \hypertarget{theorem:subsetIsPartialOrdered:c}{} \mbox{\emph{(c)}} \\
\centering $\emptyset \ \subseteq \ x$ & \label{theorem:subsetIsPartialOrdered:d} \hypertarget{theorem:subsetIsPartialOrdered:d}{} \mbox{\emph{(d)}} \\
\centering $x \ \subseteq \ \mathfrak{V}$ & \label{theorem:subsetIsPartialOrdered:e} \hypertarget{theorem:subsetIsPartialOrdered:e}{} \mbox{\emph{(e)}} \\
\centering $x \ \subseteq \ \emptyset\ \rightarrow \ x \ =  \ \emptyset$ & \label{theorem:subsetIsPartialOrdered:f} \hypertarget{theorem:subsetIsPartialOrdered:f}{} \mbox{\emph{(f)}} \\
\centering $\mathfrak{V} \ \subseteq \ x\ \rightarrow \ x \ =  \ \mathfrak{V}$ & \label{theorem:subsetIsPartialOrdered:g} \hypertarget{theorem:subsetIsPartialOrdered:g}{} \mbox{\emph{(g)}} 
\end{longtable}

\end{prop}




\par
Eine Schnittklasse ist immer Teilmenge ihrer Ausgangsklassen.

\begin{prop}
\label{theorem:intersectionIsSubset} \hypertarget{theorem:intersectionIsSubset}{}
\mbox{}
\begin{longtable}{{@{\extracolsep{\fill}}p{0.9\linewidth}l}}
\centering $(x \cap y) \ \subseteq \ x$ & \label{theorem:intersectionIsSubset:a} \hypertarget{theorem:intersectionIsSubset:a}{} \mbox{\emph{(a)}} \\
\centering $(x \cap y) \ \subseteq \ y$ & \label{theorem:intersectionIsSubset:b} \hypertarget{theorem:intersectionIsSubset:b}{} \mbox{\emph{(b)}} 
\end{longtable}

\end{prop}




\par
Eine Vereinigungsklasse hat ihre Ausgangsklassen als Teilklassen.

\begin{prop}
\label{theorem:unionIsSuperset} \hypertarget{theorem:unionIsSuperset}{}
\mbox{}
\begin{longtable}{{@{\extracolsep{\fill}}p{0.9\linewidth}l}}
\centering $x \ \subseteq \ (x \cup y)$ & \label{theorem:unionIsSuperset:a} \hypertarget{theorem:unionIsSuperset:a}{} \mbox{\emph{(a)}} \\
\centering $y \ \subseteq \ (x \cup y)$ & \label{theorem:unionIsSuperset:b} \hypertarget{theorem:unionIsSuperset:b}{} \mbox{\emph{(b)}} 
\end{longtable}

\end{prop}




\par
F{\"u}r zwei Teilklassen ist auch die Vereinigungsklasse Teilklasse. Und falls eine Klasse Teilklasse von zwei Klassen ist, dann ist sie auch Teilklasse der Schnittklasse. Beide Beziehungen sind auch umkehrbar.

\begin{prop}
\label{theorem:subsetAndAddition} \hypertarget{theorem:subsetAndAddition}{}
\mbox{}
\begin{longtable}{{@{\extracolsep{\fill}}p{0.9\linewidth}l}}
\centering $(x \ \subseteq \ z\ \land \ y \ \subseteq \ z)\ \leftrightarrow \ (x \cup y) \ \subseteq \ z$ & \label{theorem:subsetAndAddition:a} \hypertarget{theorem:subsetAndAddition:a}{} \mbox{\emph{(a)}} \\
\centering $(z \ \subseteq \ x\ \land \ z \ \subseteq \ y)\ \leftrightarrow \ z \ \subseteq \ (x \cap y)$ & \label{theorem:subsetAndAddition:b} \hypertarget{theorem:subsetAndAddition:b}{} \mbox{\emph{(b)}} 
\end{longtable}

\end{prop}




\par
Bei Schnitt oder Vereinigung bleibt eine Teilklassenbeziehung erhalten.

\begin{prop}
\label{theorem:subsetAddition} \hypertarget{theorem:subsetAddition}{}
\mbox{}
\begin{longtable}{{@{\extracolsep{\fill}}p{0.9\linewidth}l}}
\centering $x \ \subseteq \ y\ \rightarrow \ (x \cup z) \ \subseteq \ (y \cup z)$ & \label{theorem:subsetAddition:a} \hypertarget{theorem:subsetAddition:a}{} \mbox{\emph{(a)}} \\
\centering $x \ \subseteq \ y\ \rightarrow \ (x \cap z) \ \subseteq \ (y \cap z)$ & \label{theorem:subsetAddition:b} \hypertarget{theorem:subsetAddition:b}{} \mbox{\emph{(b)}} 
\end{longtable}

\end{prop}




\par
Bei der Bildung des Komplements kehrt sich die Teilklassenbeziehung um.

\begin{prop}
\label{theorem:subsetComplement} \hypertarget{theorem:subsetComplement}{}
\mbox{}
\begin{longtable}{{@{\extracolsep{\fill}}p{\linewidth}}}
\centering $x \ \subseteq \ y\ \leftrightarrow \ \overline{y} \ \subseteq \ \overline{x}$
\end{longtable}

\end{prop}




\par
F{\"u}r das Komplement und die Teilklassenbeziehung gelten die folgenden {\"A}quivalenzen.

\begin{prop}
\label{theorem:subsetComplementEquations} \hypertarget{theorem:subsetComplementEquations}{}
\mbox{}
\begin{longtable}{{@{\extracolsep{\fill}}p{0.9\linewidth}l}}
\centering $x \ \subseteq \ y\ \leftrightarrow \ (x \cap \overline{y}) \ =  \ \emptyset$ & \label{theorem:subsetComplementEquations:a} \hypertarget{theorem:subsetComplementEquations:a}{} \mbox{\emph{(a)}} \\
\centering $x \ \subseteq \ y\ \leftrightarrow \ (\overline{x} \cup y) \ =  \ \mathfrak{V}$ & \label{theorem:subsetComplementEquations:b} \hypertarget{theorem:subsetComplementEquations:b}{} \mbox{\emph{(b)}} \\
\centering $x \ \subseteq \ \overline{y}\ \leftrightarrow \ (x \cap y) \ =  \ \emptyset$ & \label{theorem:subsetComplementEquations:c} \hypertarget{theorem:subsetComplementEquations:c}{} \mbox{\emph{(c)}} \\
\centering $(x \cap y) \ \subseteq \ z\ \leftrightarrow \ x \ \subseteq \ (\overline{y} \cup z)$ & \label{theorem:subsetComplementEquations:d} \hypertarget{theorem:subsetComplementEquations:d}{} \mbox{\emph{(d)}} 
\end{longtable}

\end{prop}




\section{Einerklassen und Klassenpaare} \label{chapter3_section3} \hypertarget{chapter3_section3}{}
Eine Klasse kann auch durch explizite Auflistung ihrer Elemente definiert werden.

\par
Insbesondere kann durch Angabe eines Elements die sogenannte \emph{Einerklasse} 
festgelegt werden.
  Wiederum mit Regel \ref{rule:classDefinition} k{\"o}nnen wir die Sprachsyntax erweitern und eine neue abk{\"u}rzende Schreibweise einf{\"u}hren.

\begin{defn}[Einerklasse\index{Einerklasse}\index{Klasse!Einer-}]
\label{definition:singleton} \hypertarget{definition:singleton}{}
$$\{ x \}\ := \ \{ y \ | \ (\mathfrak{M}(x)\ \rightarrow \ y \ =  \ x) \} $$

\end{defn}

Da der Ausdruck $\{x\}$ f{\"u}r jegliches $x$ definiert ist, kann er auch f{\"u}r den Fall, 
dass $x$ eine echte Klasse ist, gebildet werden. In diesem Fall erf{\"u}llen alle Mengen 
$y$ die Bedingung 
$\mathfrak{M}(y) \land (\mathfrak{M}(x) \rightarrow y = x)$ und die Einerklasse ist 
mit der Allklasse identisch. Das f{\"u}hrt zu einem technisch einfacheren Umgang mit der
Einerklasse.\footnote{Andere Autoren wie z.~B. auch K.~G{\"o}del, definieren $\{x\}$ durch
$\{y~|~y = x\}$.}


\par
F{\"u}r Mengen enth{\"a}lt die Einerklasse wie gew{\"u}nscht nur die Menge selbst.

\begin{prop}
\label{theorem:setSingletonHasSetAsOnlyElement} \hypertarget{theorem:setSingletonHasSetAsOnlyElement}{}
\mbox{}
\begin{longtable}{{@{\extracolsep{\fill}}p{\linewidth}}}
\centering $\mathfrak{M}(x)\ \rightarrow \ \forall z\ (z \in \{ x \}\ \leftrightarrow \ z \ =  \ x)$
\end{longtable}

\end{prop}




\par
F{\"u}r echte Mengen ist die Einerklasse mit der Allklasse identisch.

\begin{prop}
\label{theorem:properSingletonIsUniversalClass} \hypertarget{theorem:properSingletonIsUniversalClass}{}
\mbox{}
\begin{longtable}{{@{\extracolsep{\fill}}p{\linewidth}}}
\centering $\neg \mathfrak{M}(x)\ \rightarrow \ \{ x \} \ =  \ \mathfrak{V}$
\end{longtable}

\end{prop}




\par
Einerklasse einer Menge zu sein ist {\"a}quivalent dazu Element seiner
Einerklasse zu sein.

\begin{prop}
\label{theorem:setSingletonEqualHasItselfAsElement} \hypertarget{theorem:setSingletonEqualHasItselfAsElement}{}
\mbox{}
\begin{longtable}{{@{\extracolsep{\fill}}p{\linewidth}}}
\centering $\mathfrak{M}(x)\ \leftrightarrow \ x \in \{ x \}$
\end{longtable}

\end{prop}




\par
Nun kann einfach durch Vereinigung zweier Einerklassen das \emph{Paar}
zweier Klassen definiert werden.

\begin{defn}[Paar\index{Paar}\index{Klasse!Paar-}]
\label{definition:pair} \hypertarget{definition:pair}{}
$$\{ x, y \}\ := \ (\{ x \} \cup \{ y \})$$

\end{defn}




\par
Ein Klassenpaar kann auch direkt, d.~h. ohne Zuhilfenahme der Einerklassen
beschrieben werden.

\begin{prop}
\label{theorem:classPairIsEqual} \hypertarget{theorem:classPairIsEqual}{}
\mbox{}
\begin{longtable}{{@{\extracolsep{\fill}}p{\linewidth}}}
\centering $\{ x, y \} \ =  \ \{ z \ | \ ((\mathfrak{M}(x)\ \land \ \mathfrak{M}(y))\ \rightarrow \ (z \ =  \ x\ \lor \ z \ =  \ y)) \} $
\end{longtable}

\end{prop}




\par
F{\"u}r Klassenpaare die aus Mengen gebildet werden kann die Eigenschaft Element 
des Klassenpaares zu sein einfacher ausgedr{\"u}ckt werden.

\begin{prop}
\label{theorem:membershipOfClassPair} \hypertarget{theorem:membershipOfClassPair}{}
\mbox{}
\begin{longtable}{{@{\extracolsep{\fill}}p{\linewidth}}}
\centering $(\mathfrak{M}(x)\ \land \ \mathfrak{M}(y))\ \rightarrow \ \forall z\ (z \in \{ x, y \}\ \leftrightarrow \ (z \ =  \ x\ \lor \ z \ =  \ y))$
\end{longtable}

\end{prop}




\par
Falls bei der Klassenpaarbildung eine echte Klasse dabei ist, dann ist das
resultierende Klassenpaar mit der Allklasse identisch.

\begin{prop}
\label{theorem:properClassPairIsUniversalClass} \hypertarget{theorem:properClassPairIsUniversalClass}{}
\mbox{}
\begin{longtable}{{@{\extracolsep{\fill}}p{\linewidth}}}
\centering $(\neg \mathfrak{M}(x)\ \lor \ \neg \mathfrak{M}(y))\ \rightarrow \ \{ x, y \} \ =  \ \mathfrak{V}$
\end{longtable}

\end{prop}




\par
Wir notieren dass die Klassenpaarbildung kommutativ ist.

\begin{prop}
\label{theorem:classPairBuildingIsCommutative} \hypertarget{theorem:classPairBuildingIsCommutative}{}
\mbox{}
\begin{longtable}{{@{\extracolsep{\fill}}p{\linewidth}}}
\centering $\{ x, y \} \ =  \ \{ y, x \}$
\end{longtable}

\end{prop}




\par
Die Einerklasse ist ein Spezialfall des Klassenpaares.

\begin{prop}
\label{theorem:singletonIsClassPair} \hypertarget{theorem:singletonIsClassPair}{}
\mbox{}
\begin{longtable}{{@{\extracolsep{\fill}}p{\linewidth}}}
\centering $\{ x \} \ =  \ \{ x, x \}$
\end{longtable}

\end{prop}




\par
Menge zu sein ist equivalent dazu Element eines Klassenpaares zu sein.

\begin{prop}
\label{theorem:setEquiInClassPair} \hypertarget{theorem:setEquiInClassPair}{}
\mbox{}
\begin{longtable}{{@{\extracolsep{\fill}}p{\linewidth}}}
\centering $\mathfrak{M}(x)\ \leftrightarrow \ x \in \{ x, y \}$
\end{longtable}

\end{prop}




\par
F{\"u}r Mengen ist die Elementbeziehung equivalent zur Teilklassenbeziehung
f{\"u}r die zugeh{\"o}rige Einerklasse.

\begin{prop}
\label{theorem:elementEquiSingletonSubclass} \hypertarget{theorem:elementEquiSingletonSubclass}{}
\mbox{}
\begin{longtable}{{@{\extracolsep{\fill}}p{\linewidth}}}
\centering $\mathfrak{M}(x)\ \rightarrow \ (x \in y\ \leftrightarrow \ x \ \subseteq \ \{ y \})$
\end{longtable}

\end{prop}




\par
Die Gleichheit von aus Mengen gebildeten Klassenpaaren ist wie erwartet.

\begin{prop}
\label{theorem:pairIdentities} \hypertarget{theorem:pairIdentities}{}
\mbox{}
\begin{longtable}{{@{\extracolsep{\fill}}p{\linewidth}}}
\centering $(\mathfrak{M}(x)\ \land \ \mathfrak{M}(y)\ \land \ \mathfrak{M}(u)\ \land \ \mathfrak{M}(v))\ \rightarrow \ (\{ x, y \} \ =  \ \{ u, v \}\ \rightarrow \ ((x \ =  \ u\ \land \ y \ =  \ v)\ \lor \ (x \ =  \ v\ \land \ y \ =  \ u)))$
\end{longtable}

\end{prop}




\section{Unendliche boolsche Operatoren} \label{chapter3_section4} \hypertarget{chapter3_section4}{}
Es k{\"o}nnen auch beliebige Schnittklassen und Vereinigungsklassen gebildet werden. Dazu muss nur
festgelegt werden, {\"u}ber welche Klassen jeweils geschnitten bzw. vereinigt wird.

\par
F{\"u}r eine Klasse von Mengen wird ein \emph{Produkt} so definiert, dass genau die Elemente, 
die in allen Mengen enthalten sind, in dem Produkt liegen.

\begin{defn}[Mengenprodukt\index{Mengenprodukt}\index{Produkt!von Mengen}]
\label{setProduct} \hypertarget{setProduct}{}
$$\bigcap \ x\ := \ \{ z \ | \ \forall y\ (y \in x\ \rightarrow \ z \in y) \} $$

\end{defn}

Diese Funktion kann als Verallgemeinerung der Schnittklassenbildung angesehen werden.
Siehe auch Proposition~\ref{theorem:setPairSetSumProduct}.

\par
Wir sagen auch, dass die Klasse $x$ eine \emph{Mengenfamilie}\index{Familie}\index{Mengenfamilie}
festlegt. Jedes Element von $x$ ist ein Mitglied der Familie.


\par
Wie {\"u}blich k{\"o}nnen wir die Elementbeziehung zum Mengenprodukt wie folgt beschreiben.

\begin{prop}
\label{theorem:setProductMembership} \hypertarget{theorem:setProductMembership}{}
\mbox{}
\begin{longtable}{{@{\extracolsep{\fill}}p{\linewidth}}}
\centering $z \in \bigcap \ x\ \leftrightarrow \ (\mathfrak{M}(z)\ \land \ \forall y\ (y \in x\ \rightarrow \ z \in y))$
\end{longtable}

\end{prop}




\par
F{\"u}r den Speziallfall $x\,=\,\emptyset$ erhalten wir.

\begin{prop}
\label{theorem:emptySetProduct} \hypertarget{theorem:emptySetProduct}{}
\mbox{}
\begin{longtable}{{@{\extracolsep{\fill}}p{\linewidth}}}
\centering $\bigcap \ \emptyset \ =  \ \mathfrak{V}$
\end{longtable}

\end{prop}




\par
Falls wir das Mengenprodukt einer nichtleeren Klasse bilden
k{\"o}nnen wir die Mengenbedingung weglassen.

\begin{prop}
\label{theorem:nonEmptySetProductMembership} \hypertarget{theorem:nonEmptySetProductMembership}{}
\mbox{}
\begin{longtable}{{@{\extracolsep{\fill}}p{\linewidth}}}
\centering $x \ \neq  \ \emptyset\ \rightarrow \ (z \in \bigcap \ x\ \leftrightarrow \ \forall y\ (y \in x\ \rightarrow \ z \in y))$
\end{longtable}

\end{prop}




\par
Analog k{\"o}nnen wir die \emph{Mengensumme} definieren.
Genau die Elemente, die in irgend einer der Mengen vorkommen, sollen in der Summe liegen.

\begin{defn}[Mengensumme\index{Mengensumme}\index{Summe!von Mengen}]
\label{definition:setSum} \hypertarget{definition:setSum}{}
$$\bigcup \ x\ := \ \{ z \ | \ \exists y\ (y \in x\ \land \ z \in y) \} $$

\end{defn}




\par
Die Zugeh{\"o}rigkeit zur Mengensumme kann wie folgt ausgedr{\"u}ckt werden.

\begin{prop}
\label{theorem:setSumMembership} \hypertarget{theorem:setSumMembership}{}
\mbox{}
\begin{longtable}{{@{\extracolsep{\fill}}p{\linewidth}}}
\centering $z \in \bigcup \ x\ \leftrightarrow \ \exists y\ (y \in x\ \land \ z \in y)$
\end{longtable}

\end{prop}

Hier k{\"o}nnen wir die Mengenbedingung $\mathfrak{M}(z)$ weglassen.


\par
F{\"u}r die leere Klasse erhalten wir.

\begin{prop}
\label{theorem:emptySetSum} \hypertarget{theorem:emptySetSum}{}
\mbox{}
\begin{longtable}{{@{\extracolsep{\fill}}p{\linewidth}}}
\centering $\bigcup \ \emptyset \ =  \ \mathfrak{V}$
\end{longtable}

\end{prop}




\par
Die Teilklassenrelation verh{\"a}lt sich zu Mengenprodukt und Mengensumme wie folgt.

\begin{prop}
\label{theorem:subsetSumProductImplication} \hypertarget{theorem:subsetSumProductImplication}{}
\mbox{}
\begin{longtable}{{@{\extracolsep{\fill}}p{0.9\linewidth}l}}
\centering $x \ \subseteq \ y\ \rightarrow \ \bigcap \ y \ \subseteq \ \bigcap \ x$ & \label{theorem:subsetSumProductImplication:a} \hypertarget{theorem:subsetSumProductImplication:a}{} \mbox{\emph{(a)}} \\
\centering $x \ \subseteq \ y\ \rightarrow \ \bigcup \ x \ \subseteq \ \bigcup \ y$ & \label{theorem:subsetSumProductImplication:b} \hypertarget{theorem:subsetSumProductImplication:b}{} \mbox{\emph{(b)}} 
\end{longtable}

\end{prop}




\par
Die Elementbeziehung induziert Teilklassenbeziehungen in der folgenden Weise.

\begin{prop}
\label{theorem:membershipToSetProductAndSum} \hypertarget{theorem:membershipToSetProductAndSum}{}
\mbox{}
\begin{longtable}{{@{\extracolsep{\fill}}p{0.9\linewidth}l}}
\centering $x \in y\ \rightarrow \ x \ \subseteq \ \bigcup \ y$ & \label{theorem:membershipToSetProductAndSum:a} \hypertarget{theorem:membershipToSetProductAndSum:a}{} \mbox{\emph{(a)}} \\
\centering $x \in y\ \rightarrow \ \bigcap \ y \ \subseteq \ x$ & \label{theorem:membershipToSetProductAndSum:b} \hypertarget{theorem:membershipToSetProductAndSum:b}{} \mbox{\emph{(b)}} 
\end{longtable}

\end{prop}




\par
Die Vereinigungs- und Schnittklassenbildung passt zu Mengensumme und
Mengenprodukt wie nachfolgend beschrieben.

\begin{prop}
\label{theorem:unionIntersectionSetSumProduct} \hypertarget{theorem:unionIntersectionSetSumProduct}{}
\mbox{}
\begin{longtable}{{@{\extracolsep{\fill}}p{0.9\linewidth}l}}
\centering $\bigcap \ (x \cup y) \ =  \ (\bigcap \ x \cap \bigcap \ y)$ & \label{theorem:unionIntersectionSetSumProduct:a} \hypertarget{theorem:unionIntersectionSetSumProduct:a}{} \mbox{\emph{(a)}} \\
\centering $\bigcup \ (x \cup y) \ =  \ (\bigcup \ x \cup \bigcup \ y)$ & \label{theorem:unionIntersectionSetSumProduct:b} \hypertarget{theorem:unionIntersectionSetSumProduct:b}{} \mbox{\emph{(b)}} \\
\centering $\bigcup \ (x \cap y) \ \subseteq \ (\bigcup \ x \cap \bigcup \ y)$ & \label{theorem:unionIntersectionSetSumProduct:c} \hypertarget{theorem:unionIntersectionSetSumProduct:c}{} \mbox{\emph{(c)}} 
\end{longtable}

\end{prop}




\par
F{\"u}r den Fall einer nichtleeren Mengenfamile haben wir dieses.

\begin{prop}
\label{theorem:nonEmptySumProductSubSet} \hypertarget{theorem:nonEmptySumProductSubSet}{}
\mbox{}
\begin{longtable}{{@{\extracolsep{\fill}}p{\linewidth}}}
\centering $\forall x\ (x \ \neq  \ \emptyset\ \rightarrow \ \bigcap \ x \ \subseteq \ \bigcup \ x)$
\end{longtable}

\end{prop}




\par
F{\"u}r Mengenpaare erhalten wir die erwarteten Ergebnisse.

\begin{prop}
\label{theorem:setPairSetSumProduct} \hypertarget{theorem:setPairSetSumProduct}{}
\mbox{}
\begin{longtable}{{@{\extracolsep{\fill}}p{0.9\linewidth}l}}
\centering $(\mathfrak{M}(x)\ \land \ \mathfrak{M}(y))\ \rightarrow \ \bigcap \ \{ x, y \} \ =  \ (x \cap y)$ & \label{theorem:setPairSetSumProduct:a} \hypertarget{theorem:setPairSetSumProduct:a}{} \mbox{\emph{(a)}} \\
\centering $(\mathfrak{M}(x)\ \land \ \mathfrak{M}(y))\ \rightarrow \ \bigcup \ \{ x, y \} \ =  \ (x \cup y)$ & \label{theorem:setPairSetSumProduct:b} \hypertarget{theorem:setPairSetSumProduct:b}{} \mbox{\emph{(b)}} 
\end{longtable}

\end{prop}




\par
F{\"u}r Einermengen erhalten wir analoge Aussagen.

\begin{prop}
\label{theorem:singletonSetSumProduct} \hypertarget{theorem:singletonSetSumProduct}{}
\mbox{}
\begin{longtable}{{@{\extracolsep{\fill}}p{0.9\linewidth}l}}
\centering $\mathfrak{M}(x)\ \rightarrow \ \bigcap \ \{ x \} \ =  \ x$ & \label{theorem:singletonSetSumProduct:a} \hypertarget{theorem:singletonSetSumProduct:a}{} \mbox{\emph{(a)}} \\
\centering $\mathfrak{M}(x)\ \rightarrow \ \bigcup \ \{ x \} \ =  \ x$ & \label{theorem:singletonSetSumProduct:b} \hypertarget{theorem:singletonSetSumProduct:b}{} \mbox{\emph{(b)}} 
\end{longtable}

\end{prop}




\section{Potenzklassenbildung} \label{chapter3_section5} \hypertarget{chapter3_section5}{}
Nun k{\"o}nnen wir einen wichtigen neuen Klassenoperator einf{\"u}hren.

\par
Aus der Teilklassenrelation l{\"a}sst sich ein weiterer Klassenoperator gewinnen, die \emph{Potenzklassenbildung}.

\begin{defn}[Potenzklasse\index{Klasse!Potenz-}]
\label{definition:power} \hypertarget{definition:power}{}
$$\mathfrak{P}(x)\ := \ \{ z \ | \ z \ \subseteq \ x \} $$

\end{defn}




\par
F{\"u}r diesen neuen Operator gelten die folgenden Aussagen.

\begin{prop}
\label{theorem:powerPropositions} \hypertarget{theorem:powerPropositions}{}
\mbox{}
\begin{longtable}{{@{\extracolsep{\fill}}p{0.9\linewidth}l}}
\centering $z \in \mathfrak{P}(x)\ \leftrightarrow \ (\mathfrak{M}(x)\ \land \ z \ \subseteq \ x)$ & \label{theorem:powerPropositions:a} \hypertarget{theorem:powerPropositions:a}{} \mbox{\emph{(a)}} \\
\centering $\mathfrak{P}(\mathfrak{V}) \ =  \ \mathfrak{V}$ & \label{theorem:powerPropositions:b} \hypertarget{theorem:powerPropositions:b}{} \mbox{\emph{(b)}} \\
\centering $\mathfrak{P}(\emptyset) \ =  \ \{ \emptyset \}$ & \label{theorem:powerPropositions:c} \hypertarget{theorem:powerPropositions:c}{} \mbox{\emph{(c)}} \\
\centering $\mathfrak{M}(x)\ \leftrightarrow \ x \in \mathfrak{P}(x)$ & \label{theorem:powerPropositions:d} \hypertarget{theorem:powerPropositions:d}{} \mbox{\emph{(d)}} \\
\centering $x \ \subseteq \ y\ \rightarrow \ \mathfrak{P}(x) \ \subseteq \ \mathfrak{P}(y)$ & \label{theorem:powerPropositions:e} \hypertarget{theorem:powerPropositions:e}{} \mbox{\emph{(e)}} \\
\centering $(\mathfrak{M}(x)\ \land \ \mathfrak{P}(x) \ \subseteq \ \mathfrak{P}(y))\ \rightarrow \ x \ \subseteq \ y$ & \label{theorem:powerPropositions:f} \hypertarget{theorem:powerPropositions:f}{} \mbox{\emph{(f)}} \\
\centering $\mathfrak{P}((x \cap y)) \ =  \ (\mathfrak{P}(x) \cap \mathfrak{P}(y))$ & \label{theorem:powerPropositions:g} \hypertarget{theorem:powerPropositions:g}{} \mbox{\emph{(g)}} \\
\centering $(\mathfrak{P}(x) \cup \mathfrak{P}(y)) \ \subseteq \ \mathfrak{P}((x \cup y))$ & \label{theorem:powerPropositions:h} \hypertarget{theorem:powerPropositions:h}{} \mbox{\emph{(h)}} \\
\centering $x \ \subseteq \ \mathfrak{P}(\bigcup \ x)$ & \label{theorem:powerPropositions:i} \hypertarget{theorem:powerPropositions:i}{} \mbox{\emph{(i)}} \\
\centering $\bigcup \ \mathfrak{P}(x) \ \subseteq \ x$ & \label{theorem:powerPropositions:j} \hypertarget{theorem:powerPropositions:j}{} \mbox{\emph{(j)}} 
\end{longtable}

\end{prop}




\par
Speziell f{\"u}r Potenzmengen gilt die folgende Aussage.

\begin{prop}
\label{theorem:powerSetPropositions} \hypertarget{theorem:powerSetPropositions}{}
\mbox{}
\begin{longtable}{{@{\extracolsep{\fill}}p{\linewidth}}}
\centering $\mathfrak{M}(x)\ \rightarrow \ x \ =  \ \bigcup \ \mathfrak{P}(x)$
\end{longtable}

\end{prop}

Sp{\"a}ter k{\"o}nnen wir die Mengenbedingung fallenlassen, da wir dann {\"u}ber weitere Axiome
verf{\"u}gen.



%% end of chapter Boolesche Algebra der Klassen\index{Boolesche Algebra!der Klassen}

\chapter{Mengen, Relationen und Funktionen} \label{chapter4} \hypertarget{chapter4}{}

In diesem Kapitel wird noch einmal genauer auf die Mengeneigenschaft eingegangen und es werden neue Axiome angegeben um die Existens von Mengen abzusichern.

\par
Um Relationen definieren zu k{\"o}nnen, wird der Begriff des \emph{geordneten Klassenpaares} ben{\"o}tigt,
der es erm{\"o}glicht das \emph{cartesische Produkt} von Klassen zu definieren. Relationen sind
Teilklassen von cartesischen Produkten und bilden eine mit bestimmten Operationen eine
\emph{universelle Algebra}\index{Algebra!universelle}. 

\par
Spezielle Relationen sind die \emph{{\"A}quivalenzrelationen}, die einen etwas
weiter gefassten Gleichheitsbegriff erm{\"o}glichen. Funktionen sind ebenfalls spezielle Relationen,
Das Fraenkelsche Ersetzungsaxiom garantiert das Mengen auf Mengen abgebildet werden.

\section{Mengen} \label{chapter4_section0} \hypertarget{chapter4_section0}{}
Zur Darstellung der Booleschen Klassenalgebra wurden noch keine mengentheoretischen Axiome ben{\"o}tigt
Im Folgenden werden weitere Axiome vorgestellt, die Bedingungen daf{\"u}r angeben, wann eine Klasse
eine Menge ist.

\par
Die leere Klasse soll auch eine Menge sein.

\begin{ax}[Axiom der leeren Menge\index{Axiom!der leeren Menge}]
\label{axiom:emptySet} \hypertarget{axiom:emptySet}{}
\mbox{}
\begin{longtable}{{@{\extracolsep{\fill}}p{\linewidth}}}
\centering $\mathfrak{M}(\emptyset)$
\end{longtable}

\end{ax}

Damit haben wir zum ersten Mal Kenntnis {\"u}ber die Existenz einer Menge.


\par
Um die Mengeneigenschaft f{\"u}r Paare von Mengen zu erhalten, haben wir das
folgende Axiom.

\begin{ax}[Axiom der Paarmenge\index{Axiom!Paarmengen-}]
\label{axiom:pairingSet} \hypertarget{axiom:pairingSet}{}
\mbox{}
\begin{longtable}{{@{\extracolsep{\fill}}p{\linewidth}}}
\centering $(\mathfrak{M}(x)\ \land \ \mathfrak{M}(y))\ \rightarrow \ \mathfrak{M}(\{ x, y \})$
\end{longtable}

\end{ax}




\par
Auch die Mengensumme einer Menge soll wieder eine Menge sein.

\begin{ax}[Summenmengenaxiom\index{Axiom!Summenmengen-}]
\label{axiom:setSumSet} \hypertarget{axiom:setSumSet}{}
\mbox{}
\begin{longtable}{{@{\extracolsep{\fill}}p{\linewidth}}}
\centering $\mathfrak{M}(x)\ \rightarrow \ \mathfrak{M}(\bigcup \ x)$
\end{longtable}

\end{ax}




\par
Die Potenzklasse einer Menge soll auch wieder eine Menge sein.

\begin{ax}[Axiom der Potenzmenge\index{Axiom!Potenzmengen-}]
\label{axiom:powerSet} \hypertarget{axiom:powerSet}{}
\mbox{}
\begin{longtable}{{@{\extracolsep{\fill}}p{\linewidth}}}
\centering $\mathfrak{M}(x)\ \rightarrow \ \mathfrak{M}(\mathfrak{P}(x))$
\end{longtable}

\end{ax}




\par
Die Teilklasse einer Menge soll wieder eine Menge sein.

\begin{ax}[Teilmengenaxiom\index{Axiom!Teilmengen-}]
\label{axiom:subset} \hypertarget{axiom:subset}{}
\mbox{}
\begin{longtable}{{@{\extracolsep{\fill}}p{\linewidth}}}
\centering $(\mathfrak{M}(x)\ \land \ y \ \subseteq \ x)\ \rightarrow \ \mathfrak{M}(y)$
\end{longtable}

\end{ax}




Die obigen Mengenaxiome erm{\"o}glichen es uns Mengen zu konstruieren.
Durch das Axiom~\ref{axiom:emptySet} haben wir eine erste Menge $\emptyset$. 
Durch die Anwendung von Axiom~\ref{axiom:powerSet} erhalten wir die Menge
$\{ \emptyset \}$. Die erneute Bildung der Potenzmenge erzeugt die Menge
$\{ \emptyset, \{ \emptyset \} \}$. Durch wiederholtes Anwendung der Prozedur
bekommen wir eine beliebige Anzahl von Mengen.\footnote{Dass die Mengen alle
paarweise voneinander verschieden sind, ist leicht zu zeigen.}

\par
Weiterhin stellen wir fest, dass wir mit unseren bisherigen Axiomen nur die 
Existenz von Mengen mit einer endlichen Elementanzahl nachweisen k{\"o}nnen.
Diese englichen Mengen sind {\glqq sicher\grqq} in dem Sinne, dass sie nicht zu
den Widerspr{\"u}chen f{\"u}hren, wie sie in der uneingeschr{\"a}nkten Mengenlehre 
Zermelos auftreten,


\par
Mit Hilfe der neuen Axiome k{\"o}nnen weitere Folgerungen gezogen werden.

\begin{prop}
\label{theorem:isSet} \hypertarget{theorem:isSet}{}
\mbox{}
\begin{longtable}{{@{\extracolsep{\fill}}p{0.9\linewidth}l}}
\centering $(\neg \mathfrak{M}(y)\ \land \ y \ \subseteq \ x)\ \rightarrow \ \neg \mathfrak{M}(x)$ & \label{theorem:isSet:a} \hypertarget{theorem:isSet:a}{} \mbox{\emph{(a)}} \\
\centering $\neg \mathfrak{M}(\mathfrak{V})$ & \label{theorem:isSet:b} \hypertarget{theorem:isSet:b}{} \mbox{\emph{(b)}} \\
\centering $(\mathfrak{M}(x)\ \land \ \mathfrak{M}(y))\ \rightarrow \ \mathfrak{M}((x \cup y))$ & \label{theorem:isSet:c} \hypertarget{theorem:isSet:c}{} \mbox{\emph{(c)}} \\
\centering $(\mathfrak{M}(x)\ \land \ \mathfrak{M}(y))\ \rightarrow \ \mathfrak{M}((x \cap y))$ & \label{theorem:isSet:d} \hypertarget{theorem:isSet:d}{} \mbox{\emph{(d)}} \\
\centering $\mathfrak{M}(x)\ \rightarrow \ \mathfrak{M}(\{ x \})$ & \label{theorem:isSet:e} \hypertarget{theorem:isSet:e}{} \mbox{\emph{(e)}} \\
\centering $\mathfrak{M}(x)\ \rightarrow \ \neg \mathfrak{M}(\overline{x})$ & \label{theorem:isSet:f} \hypertarget{theorem:isSet:f}{} \mbox{\emph{(f)}} \\
\centering $x \ =  \ \bigcup \ \mathfrak{P}(x)$ & \label{theorem:isSet:g} \hypertarget{theorem:isSet:g}{} \mbox{\emph{(g)}} \\
\centering $\mathfrak{M}(x)\ \leftrightarrow \ \mathfrak{M}(\bigcup \ x)$ & \label{theorem:isSet:h} \hypertarget{theorem:isSet:h}{} \mbox{\emph{(h)}} \\
\centering $\bigcap \ \mathfrak{V} \ =  \ \emptyset$ & \label{theorem:isSet:i} \hypertarget{theorem:isSet:i}{} \mbox{\emph{(i)}} \\
\centering $\bigcup \ \mathfrak{V} \ =  \ \mathfrak{V}$ & \label{theorem:isSet:j} \hypertarget{theorem:isSet:j}{} \mbox{\emph{(j)}} \\
\centering $x \ \neq  \ \emptyset\ \rightarrow \ \mathfrak{M}(\bigcap \ x)$ & \label{theorem:isSet:k} \hypertarget{theorem:isSet:k}{} \mbox{\emph{(k)}} 
\end{longtable}

\end{prop}




\section{Geordnetes Klassenpaar} \label{chapter4_section1} \hypertarget{chapter4_section1}{}
Das Konzept eines geordneten Paars ist f{\"u}r die weitere Entwicklung unserer Theorie wichtig.
Es erm{\"o}glicht uns die Objekte anzuordnen. Bisher hingen unsere Objektzusammenfassungen nicht
von der Reihenfolge der Sammlung ab. Wir wollen nun aber auch nach der Zusammenfassung
herausfinden k{\"o}nnnen, welches das \emph{erste} Element und welches das \emph{zweite} Element
war.

\par
Die Definition eines geordneten Paares $\langle x, y\rangle$ erfolgt nach \emph{N.~Wiener} (1914) bzw. \emph{K.~Kuratowski} (1921).

\begin{defn}[Geordnetes Paar\index{geordnetes Paar}\index{Paar!geordnetes}]
\label{definition:orderedPair} \hypertarget{definition:orderedPair}{}
$$\langle x, y \rangle\ := \ \{ \{ x \}, \{ x, y \} \}$$

\end{defn}




\par
F{\"u}r geordnete Paare von Mengen spielt die Reihe der angegebenen Elemente
eine Rolle. Geordnete Paare sollten nur dann identisch sein, wenn
ihre ersten Elemente und ihre zweiten Elemente identisch sind.

\begin{prop}
\label{theorem:orderedPairEquality} \hypertarget{theorem:orderedPairEquality}{}
\mbox{}
\begin{longtable}{{@{\extracolsep{\fill}}p{\linewidth}}}
\centering $(\mathfrak{M}(x)\ \land \ \mathfrak{M}(y)\ \land \ \mathfrak{M}(u)\ \land \ \mathfrak{M}(v))\ \rightarrow \ (\langle x, y \rangle \ =  \ \langle u, v \rangle\ \rightarrow \ (x \ =  \ u\ \land \ y \ =  \ v))$
\end{longtable}

\end{prop}




\par
Ein aus Mengen gebildetes geordnetes Paar ist auch eine Menge. 
Die Umkehrung gilt auch.

\begin{prop}
\label{theorem:orderedPairOfSets} \hypertarget{theorem:orderedPairOfSets}{}
\mbox{}
\begin{longtable}{{@{\extracolsep{\fill}}p{\linewidth}}}
\centering $(\mathfrak{M}(x)\ \land \ \mathfrak{M}(y))\ \leftrightarrow \ \mathfrak{M}(\langle x, y \rangle)$
\end{longtable}

\end{prop}




\par
Falls eine der Klassen keine Menge ist, dann ist das geordnete Paar
mit der Allklasse identisch.

\begin{prop}
\label{theorem:orderedPairWithNonSet} \hypertarget{theorem:orderedPairWithNonSet}{}
\mbox{}
\begin{longtable}{{@{\extracolsep{\fill}}p{\linewidth}}}
\centering $(\neg \mathfrak{M}(x)\ \lor \ \neg \mathfrak{M}(y))\ \rightarrow \ \langle x, y \rangle \ =  \ \mathfrak{V}$
\end{longtable}

\end{prop}




\par
Um {\"u}ber geordnete Paare sprechen zu k{\"o}nnen ben{\"o}tigen wir ein neues Pr{\"a}dikat
{\glqq ist ein geordnetes Paar \grqq}.

\begin{defn}[Eigenschaft geordnetes Paar\index{Paar!geordnetes}]
\label{definition:isOrderedPair} \hypertarget{definition:isOrderedPair}{}
$$\mbox{isOrderedPair}(x)\ :\leftrightarrow \ \exists u\ \exists v\ x \ =  \ \langle u, v \rangle$$

\end{defn}

Wir betonen noch einmal, dass auch $\mathfrak{V}$ ein geordnetes Paar ist. Aber da
wir meistens {\"u}ber Elemente von Klassen sprechen, haben wir nur mit Mengen zu tun,
die eventuell auch geordnete Paare sind.


\section{Kartesisches Produkt} \label{chapter4_section2} \hypertarget{chapter4_section2}{}
F{\"u}r die geordenten Klassenpaare brauchen wir eine Metastruktur. Daf{\"u}r fassen wir einfach geordete
Paare in einer Klasse zusammen.

\par
Das Kartesische Produkt\footnote{Kartesisch oder kartesianisch nach der 
lateinischen Namensform \emph{Cartesius} des Philosophen und Mathematikers 
\emph{R.~Descartes}.}, auch \emph{Kreuzprodukt} genannt, ist die Klasse 
aller geordneter Paare, deren Elemente aus den Ausgangsklassen stammen.

\begin{defn}[Kartesisches Produkt\index{Produkt!kartesisches}\index{Produkt!Kreuz-}\index{Kartesisches Produkt}\index{Kreuzprodukt}]
\label{definition:cartesianProduct} \hypertarget{definition:cartesianProduct}{}
$$( x \times y)\ := \ \{ z \ | \ \exists u\ \exists v\ (u \in x\ \land \ v \in y\ \land \ z \ =  \ \langle u, v \rangle) \} $$

\end{defn}




\section{Relationen} \label{chapter4_section3} \hypertarget{chapter4_section3}{}
Es ist wichtig Relationen zwischen mathematischen Objekten
ausdr{\"u}cken zu k{\"o}nnen und sie auch als Objekte behandeln
zu k{\"o}nnen. Es stellt sich heraus, dass wir keine neuen
Objektarten ben{\"o}tigen. Unsere bisherigen Strukturen reichen aus.

\par
Num k{\"o}nnen wir den Begriff der \emph{Relation} auch innerhalb unserer Mengenlehre definieren.

\begin{defn}[Relation\index{relation}]
\label{definition:relation} \hypertarget{definition:relation}{}
$$\mathfrak{Rel}(x)\ :\leftrightarrow \ \forall y\ (y \in x\ \rightarrow \ \mbox{isOrderedPair}(y))$$

\end{defn}




\par
Ein paar Aussagen {\"u}ber Relationen.

\begin{prop}
\label{theorem:relationProperties} \hypertarget{theorem:relationProperties}{}
\mbox{}
\begin{longtable}{{@{\extracolsep{\fill}}p{0.9\linewidth}l}}
\centering $\mathfrak{Rel}(\emptyset)$ & \label{theorem:relationProperties:a} \hypertarget{theorem:relationProperties:a}{} \mbox{\emph{(a)}} \\
\centering $\mathfrak{Rel}(( \mathfrak{V} \times \mathfrak{V}))$ & \label{theorem:relationProperties:b} \hypertarget{theorem:relationProperties:b}{} \mbox{\emph{(b)}} \\
\centering $(\mathfrak{Rel}(x)\ \land \ \mathfrak{Rel}(y))\ \rightarrow \ \mathfrak{Rel}((x \cap y))$ & \label{theorem:relationProperties:c} \hypertarget{theorem:relationProperties:c}{} \mbox{\emph{(c)}} \\
\centering $(\mathfrak{Rel}(x)\ \land \ \mathfrak{Rel}(y))\ \rightarrow \ \mathfrak{Rel}((x \cup y))$ & \label{theorem:relationProperties:d} \hypertarget{theorem:relationProperties:d}{} \mbox{\emph{(d)}} 
\end{longtable}

\end{prop}




\par
Wie geben nun eine allgemeine Definitin des Begriffs \emph{Definitionsbereich} an.

\begin{defn}[Definitionsbereich\index{Definitionsbereich}]
\label{definition:domain} \hypertarget{definition:domain}{}
$$\mathfrak{Dom}(x)\ := \ \{ y \ | \ \exists z\ \langle y, z \rangle \in x \} $$

\end{defn}




\par
Analog zu dem Definitionsbereich legen wir den \emph{Wertebereich} einer Klasse fest.

\begin{defn}[Wertebereich\index{Wertebereich}]
\label{definition:range} \hypertarget{definition:range}{}
$$\mathfrak{Rng}(x)\ := \ \{ y \ | \ \exists z\ \langle z, y \rangle \in x \} $$

\end{defn}




\section{Relationenalgebra} \label{chapter4_section4} \hypertarget{chapter4_section4}{}
MISSING! OTHER: +++

\section{{\"A}quivalenzrelationen} \label{chapter4_section5} \hypertarget{chapter4_section5}{}
MISSING! OTHER: +++

\section{Abbildungen und Funktionen} \label{chapter4_section6} \hypertarget{chapter4_section6}{}
MISSING! OTHER: +++

\par
Eine Funktion ist einfach eine spezielle Art von Relation.

\begin{defn}[Funktion\index{Funktion}]
\label{definition:function} \hypertarget{definition:function}{}
$$\mathfrak{Funct}(x)\ :\leftrightarrow \ \mathfrak{Rel}(x)\ \land \ \forall y\ (y \in \mathfrak{Dom}(x)\ \rightarrow \ \exists! z\ \langle y, z \rangle \in x)$$

\end{defn}




\par
Falls der Definitionsbereich einer Funktion eine Menge ist, dann sollte auch ihr Wertebereich eine Menge sein.

\begin{ax}[Fraenkelsches Ersetzungsaxiom\index{Axiom!Fraenkelsches Ersetzungs-}\index{Substitutionsaxiom}\index{Fraenkel}]
\label{axiom:FraenkelsReplacement} \hypertarget{axiom:FraenkelsReplacement}{}
\mbox{}
\begin{longtable}{{@{\extracolsep{\fill}}p{\linewidth}}}
\centering $(\mathfrak{Funct}(f)\ \land \ \mathfrak{M}(\mathfrak{Dom}(f)))\ \rightarrow \ \mathfrak{M}(\mathfrak{Rng}(f))$
\end{longtable}

\end{ax}





%% end of chapter Mengen, Relationen und Funktionen

\chapter{Nat{\"u}rliche Zahlen} \label{chapter5} \hypertarget{chapter5}{}

MISSING! OTHER: +++

\section{Fundierung und Unendlichkeit} \label{chapter5_section0} \hypertarget{chapter5_section0}{}
MISSING! OTHER: +++

\par
Mengen $x$ sollten sich nicht selbst als Element enthalten oder ein Element besitzen das wiederum
$x$ als Element hat. Um diese und andere Enthaltenseinszirkel auszuschlie{\ss}en stellen wir das filgende
Axiom vor.

\begin{ax}[Fundierungsaxiom\index{Axiom!Fundierungs-}\index{Axiom!Regularit{\"a}ts-}]
\label{axiom:foundation} \hypertarget{axiom:foundation}{}
\mbox{}
\begin{longtable}{{@{\extracolsep{\fill}}p{\linewidth}}}
\centering $x \ \neq  \ \emptyset\ \rightarrow \ \exists y\ (y \in x\ \land \ (y \cap x) \ =  \ \emptyset)$
\end{longtable}

\end{ax}

Dieses Axiom hei{\ss}t auch Regularit{\"a}tsaxiom.


\par
Eine naheliegende Klassenerweiterung ist die Bildung der
Vereinigungsmenge mit der Einerklasse.

\begin{defn}[Nachfolger\index{Nachfolger}]
\label{definition:successor} \hypertarget{definition:successor}{}
$$x'\ := \ (x \cup \{ x \})$$

\end{defn}

Weil $x \notin x$ f{\"u}gt die Nachfolgerfunktion der orginalen Klasse
genau ein Element hinzu.


\par
Wir wollen eine Menge mit unendlich vielen Elementen haben. So fordern wir einfach ihre Existenz.

\begin{ax}[Unendlichkeits\index{Unendlichkeit!-saxiom}\index{Axiom!Unendlichkeits-}]
\label{axiom:infinity} \hypertarget{axiom:infinity}{}
\mbox{}
\begin{longtable}{{@{\extracolsep{\fill}}p{\linewidth}}}
\centering $\exists x\ (\mathfrak{M}(x)\ \land \ \emptyset \in x\ \land \ \forall y\ (y \in x\ \rightarrow \ y' \in x))$
\end{longtable}

\end{ax}




\section{Definition und Grundeigenschaften} \label{chapter5_section1} \hypertarget{chapter5_section1}{}
MISSING! OTHER: +++

\section{Induktion} \label{chapter5_section2} \hypertarget{chapter5_section2}{}
MISSING! OTHER: +++

\section{Folgen und normale Funktionen} \label{chapter5_section3} \hypertarget{chapter5_section3}{}
MISSING! OTHER: +++

\section{Rekursion} \label{chapter5_section4} \hypertarget{chapter5_section4}{}
MISSING! OTHER: +++


%% end of chapter Nat{\"u}rliche Zahlen

\chapter{Auswahlaxiom} \label{chapter6} \hypertarget{chapter6}{}

+++

\section{Wohlordnungen} \label{chapter6_section0} \hypertarget{chapter6_section0}{}
+++

\par
Nun kommen wir zu dem bekannten Auswahlaxiom. Wir formulieres es f{\"u}r Relationen.

\begin{ax}[Auswahlaxiom\index{Axiom!Auswahl-}]
\label{axiom:choice} \hypertarget{axiom:choice}{}
\mbox{}
\begin{longtable}{{@{\extracolsep{\fill}}p{\linewidth}}}
\centering $\mathfrak{Rel}(x)\ \rightarrow \ \exists y\ (\mathfrak{Funct}(y)\ \rightarrow \ (y \ \subseteq \ x\ \land \ \mathfrak{Dom}(x) \ =  \ \mathfrak{Dom}(y)))$
\end{longtable}

\end{ax}




\section{Anwendungen des Auswahlaxioms} \label{chapter6_section1} \hypertarget{chapter6_section1}{}
MISSING! OTHER: 


%% end of chapter Auswahlaxiom

\chapter{Kontinuum} \label{chapter7} \hypertarget{chapter7}{}



%% end of chapter Kontinuum

\begin{thebibliography}{99}
\addcontentsline{toc}{chapter}{Literaturverzeichnis}


%% Used other QEDEQ modules:
\bibitem{qedeq_logic_v1} qedeq\_logic\_v1 \url{http://qedeq.org/0_03_07/doc/math}


%% Other references:

\bibitem{lemmon} \emph{E. J. Lemmon}, Introduction to Axiomatic Set Theory, Routledge \& Kegan Paul Ltd, London 1968

\bibitem{monk} \emph{J. D. Monk}, Introduction to Set Theory, McGraw-Hill, New York 1996

\bibitem{schmidt} \emph{J. Schmidt}, Mengenlehre I, BI, Mannheim 1966

\end{thebibliography}
\backmatter

\addcontentsline{toc}{chapter}{\indexname} \printindex

\end{document}

