% -*- TeX:EN -*-
%%% ====================================================================
%%% @LaTeX-file    qedeq_logic_v1_en.tex
%%% Generated from http://qedeq.org/0_03_07/doc/math/qedeq_logic_v1.xml
%%% Generated at   2008-01-20 13:40:30,913
%%% ====================================================================

%%% Permission is granted to copy, distribute and/or modify this document
%%% under the terms of the GNU Free Documentation License, Version 1.2
%%% or any later version published by the Free Software Foundation;
%%% with no Invariant Sections, no Front-Cover Texts, and no Back-Cover Texts.

\documentclass[a4paper,german,10pt,twoside]{book}
\usepackage[english]{babel}
\usepackage{makeidx}
\usepackage{amsmath,amsthm,amssymb}
\usepackage{color}
\usepackage[bookmarks,bookmarksnumbered,bookmarksopen,
   colorlinks,linkcolor=webgreen,pagebackref]{hyperref}
\definecolor{webgreen}{rgb}{0,.5,0}
\usepackage{graphicx}
\usepackage{xr}
\usepackage{epsfig,longtable}
\usepackage{tabularx}

\newtheorem{thm}{Theorem}[chapter]
\newtheorem{cor}[thm]{Corollary}
\newtheorem{lem}[thm]{Lemma}
\newtheorem{prop}[thm]{Proposition}
\newtheorem{ax}{Axiom}
\newtheorem{rul}{Rule}

\theoremstyle{definition}
\newtheorem{defn}[thm]{Definition}
\newtheorem{idefn}[thm]{Initial Definition}

\theoremstyle{remark}
\newtheorem{rem}[thm]{Remark}
\newtheorem*{notation}{Notation}

\addtolength{\textheight}{7\baselineskip}
\addtolength{\topmargin}{-5\baselineskip}

\setlength{\parindent}{0pt}

\frenchspacing \sloppy

\makeindex


\title{Elements of Mathematical Logic}
\author{
Michael Meyling
}

\begin{document}

\maketitle

\setlength{\parskip}{5pt plus 2pt minus 1pt}
\mbox{}
\vfill

\par
The source for this document can be found here:
\par
\url{http://qedeq.org/0_03_07/doc/math/qedeq_logic_v1.xml}

\par
Copyright by the authors. All rights reserved.
\par
If you have any questions, suggestions or want to add something to the list of modules that use this one, please send an email to the address \url{mailto:mime@qedeq.org}

\setlength{\parskip}{0pt}
\tableofcontents

\setlength{\parskip}{5pt plus 2pt minus 1pt}

\chapter*{Summary\index{summary}} \label{chapter0} \hypertarget{chapter0}{}
\addcontentsline{toc}{chapter}{Summary\index{summary}}

The project \textbf{Hilbert II} deals with the formal presentation and documentation of mathematical knowledge. For this reason \textbf{Hilbert II} provides a program suite to accomplish that tasks. The concrete documentation of mathematical basics is also a purpose of this project.   
For further informations about the \textbf{Hilbert II} project see under \url{http://www.qedeq.org/index_de.html}.

\par
This document describes the logical axoims and the rules and meta rules that are used to derive new propositions.

\par
The presentation is axiomatic and in a formal form. A formal calculus is given that enables us to derive all true formulas. Additional derived rules, definitions, abbreviations and syntax extensions basically correspond with the mathematical practice.

\par
This document is also written in a formal language, the original text is a XML file with a syntax defined by the XSD \url{http://www.qedeq.org/current/xml/qedeq.xsd}.

\par
This document is work in progress and is updated from time to time. Especially at the locations marked by {\glqq+++\grqq} additions or changes will take place.

%% end of chapter Summary\index{summary}

\chapter*{Foreword} \label{chapter1} \hypertarget{chapter1}{}
\addcontentsline{toc}{chapter}{Foreword}

The whole mathematical universe can be unfolded by set--theoretic means. Beside the set--theoretic axioms only logical axioms and rules are required. These elementary basics are sufficient to define the most complex mathematical structures and enable us to prove propositions for those structures. This approach can be fully formalized and can be reduced to simple manipulations of character strings. The semantical interpretation of these character strings represent the mathematical universum.

\par
It is more than convenient to introduce abbreviations and use further derivation rules. But these comforts could be eliminated and replaced by the basic terms at any time\footnote{At least this is theoretically possible. This transformation is not in each case practically realizable due to restrictions in time and space. For example it is not possible to write down the
natural number $1,000,000,000$ completely in set notation.}.

\par
This project has its source in a childhood dream to undertake a formalization of mathematics. In the meantime the technical possibilities are highly developed so that a realization seems within reach.

\par
Special thanks go to the professors \emph{W.~Kerby} and \emph{V.~G{\"u}nther} of the university of Hamburg for their inspiring lectures about logic and axiomatic set theory. Without these important impulses this project would not exist.

\par
I am deeply grateful to my wife \emph{Gesine~Dr{\"a}ger} and our son \emph{Lennart} for their support and patience.          

\par
\vspace*{1cm} Hamburg, January, 2008 \\
\hspace*{\fill} Michael Meyling

%% end of chapter Foreword

\chapter*{Introduction} \label{chapter2} \hypertarget{chapter2}{}
\addcontentsline{toc}{chapter}{Introduction}

At the beginning we quote \emph{D. Hilbert} from the lecture {\glqq The Logical Basis of Mathematics\grqq}, September 
1922\footnote{Lecture given at the Deutsche Naturforscher-Gesellschaft, September 1922.}.

\par
\begin{quote} {
\glqq The fundamental idea of my proof theory is the following:

\par
All the propositions that constitute in mathematics are converted into formulas, so that mathematics proper becomes all inventory of formulas. These differ from the ordinary formulas of mathematics only in that, besides the ordinary signs, 
the logical signs especially {\glqq implies\grqq} ($\rightarrow$) and for {\glqq not\grqq} ($\bar{\quad}$) occur in them. 
Certain formulas, which serve as building blocks for the formal edifice of mathematics, are called axioms. A proof is an array that must be given as such to our perceptual intuition of it of inferences according to the schema\\
\begin{eqnarray*}
& A & \\
& A \rightarrow B& \\
\cline{2-3}
 & B &
\end{eqnarray*}
where each of the premises, that is, the formulae, $A$ and $A \rightarrow B$ in the array either is an axiom or directly from an axiom by substitution, or else coincides with the end formula $B$ of an inference occurring earlier in the proof or results from it by substitution. A formula is said to be provable if it is either an axiom or the end formula of a proof.\grqq}
\end{quote}

\par
At the beginning there is logic. Logic is the analysis of methods of reasoning. It helps to derive new propositions from already given ones. Logic is universally applicable.

\par
In the 1928 published book \emph{Grundz{\"u}ge der theoretischen Logik} (Principles of Theoretical Logic) \emph{D.~Hilbert} and \emph{W.~Ackermann} formalized propositional calculus in a way that build the basis for the logical system used here. 1959 \emph{P.~S.~Novikov} specified a refined axiom and rule system for predicate calculus.

\par
In this text we present a first order predicate calculus with identity and functors that is the starting point for the development of the mathematical theory. Only the results without any proofs and in short form are given in the following.

%% end of chapter Introduction

\chapter{Language} \label{chapter3} \hypertarget{chapter3}{}

The logical foundation of \textbf{Hilbert II} will be introduced here. The language of our calculus bases on the formalizations of \emph{D.~Hilbert}, \emph{W.~Ackermann}, \emph{P.~Bernays} and \emph{P.~S.~Novikov}. New rules 
can be derived from the herein presented. Only these meta rules lead to a smooth flowing logical argumentation.

\section{Terms\index{term} and Formulas\index{formula}} \label{chapter3_section0} \hypertarget{chapter3_section0}{}
We use the \emph{logical symbols} 
$L = \{$ \mbox{`$\neg$'}, \mbox{`$\vee$'}, \mbox{`$\wedge$'}, \mbox{`$\leftrightarrow$'}, \mbox{`$\rightarrow$'}, \mbox{`$\forall$'}, \mbox{`$\exists$'} $\}$, the \emph{predicate constants}\index{predicate constant}\index{constant!predicate} $C = \{c^k_i~|~i, k \in \omega\}$, the \emph{function variables}\index{function variable}\index{variable!function}\footnote{Function variables are used for a shorter notation. For example writing an identity proposition $x = y \rightarrow f(x) = f(y)$. Also this introduction prepares for the syntax extension for functional classes.} $F = \{f^k_i~|~i, k \in \omega \land k > 0\}$, the \emph{function constants}\index{function constant}\index{constant!function}\footnote{Function constants are also introduced for convenience and are used for direct defined class functions. For example to define building of the power class operator, the union and intersection operator and the successor function. All these function constants can be interpreted as abbreviations.} $H = \{h^k_i~|~i, k \in \omega\}$, the \emph{subject variables}\index{subject variable}\index{variable!subject} $V = \{v_i~|~i \in \omega\}$, as well as
\emph{predicate variables}\index{predicate variable}\index{variable!predicate} $P = \{p^k_i~|~i, k \in \omega\}$.\footnote{By $\omega$ we understand the natural numbers including zero. All involved symbols are pairwise disjoint. 
Therefore we can conclude for example: $f^k_i = f^{k'}_{i'} \rightarrow (k = k' \land i = i')$ and $h^k_i \neq v_j$.} 
For the \emph{arity}\index{arity} or \emph{rank}\index{rank} of an operator we take the upper index. The set of predicate variables with zero arity is also called set of \emph{proposition variables}\index{proposition variable}\index{variable!proposition} or \emph{sentence letters}\index{sentence letters}: $A := \{p_i^0~|~i \in \omega \}$.
For subject variables we write short hand certain lower letters: \mbox{$v_1 = $ `$u$'}, \mbox{$v_2 = $ `$v$'}, 
\mbox{$v_3 = $ `$w$'}, \mbox{$v_4 = $ `$x$'}, \mbox{$v_5 = $ `$y$'}, \mbox{$v_5 = $ `$z$'}.
Furthermore we use the following short notations: for the predicate variables $p^n_1 = $ `$\phi$' und $p^n_2 = $ `$\psi$', where the appropriate arity $n$ is calculated by counting the subsequent parameters, for the proposition variables
$a_1 = $ `$A$', $a_2 = $ `$B$' and $a_3 = $ `$C$', for the function variables: $f^n_1 = $ `$f$' und $f^n_2 = $ `$g$', where again the appropriate arity $n$ is calculated by counting the subsequent parameters.  All binary propositional operators are written in infix notation. Parentheses surrounding groups of operands and operators are necessary to indicate the 
intended order in which operations are to be performed. E.~g. for the operator $\land$ with the parameters $A$ and $B$ we write $(A \land B)$. 


In the absence of parentheses the usual precedence rules determine the order of operations. Especially outermost parentheses are omitted.

\par
The operators have the order of precedence described below (starting with the highest).
$$
\begin{array}{c}
  \neg, \forall, \exists  \\
  \land \\
  \lor \\
  \rightarrow, \leftrightarrow \\
\end{array}
$$

\par
The term \emph{term\index{term}} is defined recursively as follows:

\begin{enumerate}
\item Every subject variable is a term. 
\item Let $i, k \in \omega$ and let $t_1$, \ldots, $t_k$ be terms. Then $h^k_i(t_1, \ldots, t_k)$ is a term and if $k > 0$, so $f^k_i(t_1, \ldots, t_k)$ is a term too.
\end{enumerate}

Therefore all zero arity function constants $\{h^0_i~|~i \in \omega\}$ are terms. They are called \emph{individual constants}\index{individual constant}\index{constant!individual}.\footnote{In an analogous manner subject variables might be defined as function variables of zero arity. Because subject variables play an important role they have their own notation.}

\par
We define a \emph{formula\index{formula}} and the relations \emph{free}\index{bound subject variable}\index{subject variable!free} and \emph{bound}\index{bound subject variable}\index{subject variable!bound} subject variable recursivly as follows:

\begin{enumerate}

\item Every proposition variable is a formula. Such formulas contain no free or bound subject variables.

\item If $p^k$ is a predicate variable with arity $k$ and $c^k$ is a predicate constant with arity $k$ and $t_1, t_2, \ldots, t_k$ are terms, then $p^k(t_1, t_2, \ldots t_k)$ and $c^k(t_1, t_2, \ldots, t_k)$ are formulas. All subject variables that occur at least in one of $t_1, t_2, \ldots, t_k$ are free subject variables. Bound subject variables does not occur.\footnote{This second item includes the first one, which is only listed for clarity.} 

\item Let $\alpha, \beta$ be formulas in which no subject variables occur bound in one formula and free in the other. Then $\neg \alpha$, $(\alpha \land \beta)$, $(\alpha \lor \beta)$, $(\alpha \rightarrow \beta)$ and $(\alpha \leftrightarrow \beta)$ are also formulas. Subject variables which occur free (respectively bound) in $\alpha$ or $\beta$ stay free (respectively bound).

\item If in the formula $\alpha$ the subject variable $x_1$ occurs not bound\footnote{This means that $x_1$ is free in the formula or does not occur at all.}, then also $\forall x_1~\alpha$ and $\exists x_1~\alpha$ are formulas. The symbol $\forall$ is called \emph{universal quantifier}\index{universal quantifier}\index{quantifier!universal} and $\exists$ as
\emph{existential quantifier}\index{existential quantifier}\index{quantifier!existential}.

Except for $x_1$ all free subject variables of $\alpha$ stay free. All bound subject variables are still bound and additionally $x_1$ is bound too.

\end{enumerate}

All formulas that are only built by usage of 1. and 3. are called formulas of the \emph{propositional calculus}\index{propositional calculus}\index{calculus!propositional}.

\par
For each formula $\alpha$ the following proposition holds: the set of free subject variables is disjoint with the set of bound subject variables..\footnote{Other formalizations allow for example $\forall x_1~\alpha$ also if $x_1$ occurs already bound within $\alpha$. Also propositions like $\alpha(x) \land (\forall x_1 \beta)$ are allowed. In this formalizations
free and bound are defined for a single occurrence of a variable.}

\par
If a formula has the form $\forall x_1 ~ \alpha$ respectively $\exists x_1 ~ \alpha$ then the formula $\alpha$ is called the
\emph{scope}\index{scope}\index{quantifier!scope} of the quantifier $\forall$ respectively $\exists$.

\par
All formulas that are used to build up a formula by 1. to 4. are called \emph{part formulas}\index{part formula}\index{formula!part}.


%% end of chapter Language

\chapter{Axioms and Rules of Inference} \label{chapter4} \hypertarget{chapter4}{}

We now state the system of axioms for the predicate calculus and present the rules for obtaining new formulas from them.

\section{Axioms\index{axioms!of predicate calculus}\index{predicate calculus!axioms}} \label{chapter4_section0} \hypertarget{chapter4_section0}{}
The logical operators of propositional calculus \mbox{`$\neg$'}, \mbox{`$\vee$'}, \mbox{`$\wedge$'}, \mbox{`$\leftrightarrow$'} und \mbox{`$\rightarrow$'} combine arbitrary \emph{propositions} to new propositions.
A proposition is a statement that affirms or denies something and is either {\glqq true\grqq} or {\glqq false\grqq} (but not both).\footnote{Later on we will define the symbols $\top$ and $\bot$ as truth values.}

The new ingredient of predicate calculus is quantification.

The binary operator \mbox{`$\vee$'} (logical disjunction) combines the two propositions $\alpha$ and $\beta$ into the new proposition $\alpha \vee \beta$. It results in true if at least one of its operands is true.

\par
The unary operator \mbox{`$\neg$'} (logical negation) changes the truth value of a proposition $\alpha$. $\neg \alpha$ has a value of true when its operand is false and a value of false when its operand is true. 

\par
The \emph{logical implication} (\emph{if}) the, \emph{logical conjunction} (\emph{and}) and the \emph{logical equivalence} (\emph{biconditional}) are defined as abbreviations.\footnote{Actually the symbols $\wedge, \rightarrow, 
\leftrightarrow$ are defined later on and are a syntax extension. But for convenience these symbols are already part of the logical language.}


\par
The logical implication (`if') could be defined as follows.

\begin{defn}[Implication\index{definition!of implication}]
\label{definition:implication} \hypertarget{definition:implication}{}
$$\alpha \rightarrow \beta\ :\leftrightarrow \ \neg \alpha\ \lor \ \beta$$

\end{defn}




\par
The logical conjunction (`and') could be defined with de Morgan.

\begin{defn}[Conjunction\index{definition!of conjunction}]
\label{definition:conjunction} \hypertarget{definition:conjunction}{}
$$\alpha \land \beta\ :\leftrightarrow \ \neg (\neg \alpha\ \lor \ \neg \beta)$$

\end{defn}




\par
The logical equivalence (`iff') is defined as usual.

\begin{defn}[Equivalence\index{definition!of equivalence}]
\label{definition:equivalence} \hypertarget{definition:equivalence}{}
$$\alpha \land \beta\ :\leftrightarrow \ (\alpha\ \rightarrow \ \beta)\ \land \ (\beta\ \rightarrow \ \alpha)$$

\end{defn}




\par
Now we come to the first axiom of propositional calculus. This axiom 
enables us to get rid of an unnecessary disjunction.

\begin{ax}[Disjunction Idempotence\index{axiom!of disjunction idempotence}]
\label{axiom:disjunction_idempotence} \hypertarget{axiom:disjunction_idempotence}{}
\mbox{}
\begin{longtable}{{@{\extracolsep{\fill}}p{\linewidth}}}
\centering $(A\ \lor \ A)\ \rightarrow \ A$
\end{longtable}

\end{ax}




\par
If a proposition is true, any alternative may be added without making it false.

\begin{ax}[Axiom of Weakening\index{axiom!of weakening}]
\label{axiom:disjunction_weakening} \hypertarget{axiom:disjunction_weakening}{}
\mbox{}
\begin{longtable}{{@{\extracolsep{\fill}}p{\linewidth}}}
\centering $A\ \rightarrow \ (A\ \lor \ B)$
\end{longtable}

\end{ax}




\par
The disjunction should be commutative.

\begin{ax}[Commutativity of the Disjunction\index{axiom!of weakening}]
\label{axiom:disjunction_commutative} \hypertarget{axiom:disjunction_commutative}{}
\mbox{}
\begin{longtable}{{@{\extracolsep{\fill}}p{\linewidth}}}
\centering $(A\ \lor \ B)\ \rightarrow \ (B\ \lor \ A)$
\end{longtable}

\end{ax}




\par
An disjunction could be added at both side of an implication.

\begin{ax}[Disjunctive Addition\index{axiom!of disjunctive addition}]
\label{axiom:disjunction_addition} \hypertarget{axiom:disjunction_addition}{}
\mbox{}
\begin{longtable}{{@{\extracolsep{\fill}}p{\linewidth}}}
\centering $(A\ \rightarrow \ B)\ \rightarrow \ ((C\ \lor \ A)\ \rightarrow \ (C\ \lor \ B))$
\end{longtable}

\end{ax}




\par
If something is true for all $x$, it is true for any specific $y$.

\begin{ax}[Universal Instantiation\index{axiom!of universal instantiation}]
\label{axiom:universalInstantiation} \hypertarget{axiom:universalInstantiation}{}
\mbox{}
\begin{longtable}{{@{\extracolsep{\fill}}p{\linewidth}}}
\centering $\forall x\ \phi(x)\ \rightarrow \ \phi(y)$
\end{longtable}

\end{ax}




\par
If a predicate holds for some particular $y$, then there is an $x$ for which the predicate holds.

\begin{ax}[Existential Generalization\index{axiom!of existential generalization}]
\label{axiom:existencialGeneralization} \hypertarget{axiom:existencialGeneralization}{}
\mbox{}
\begin{longtable}{{@{\extracolsep{\fill}}p{\linewidth}}}
\centering $\phi(y)\ \rightarrow \ \exists x\ \phi(x)$
\end{longtable}

\end{ax}




\section{Rules of Inference\index{rules!of predicate calculus}\index{rules!of inference}} \label{chapter4_section1} \hypertarget{chapter4_section1}{}
The following rules of inference enable us to obtain new true formulas from the axioms that are assumed to be true. From these new formulas we derive further formulas. So we can successively extend the set of true formulas.

\par


\begin{rul}[Modus Ponens\index{Modus Ponens}]
\label{rule:modusPonens} \hypertarget{rule:modusPonens}{}
If both formulas $\alpha$ and $\alpha \rightarrow \beta$ are true, then we can conclude that $\beta$ is true as well.
\end{rul}




\par


\begin{rul}[Replace Free Subject Variable]
\label{rule:replaceFree} \hypertarget{rule:replaceFree}{}
We start with a true formula.
A free subject variable may be replaced by an arbitrary term, provided that the substituted term contains no subject variable that have a bound occurrence in the original formula. All occurrences of the free variable must be simultaneously replaced.
\end{rul}

The prohibition to use subject variables within the term that occur bound in the original formula has two reasons. First it ensures that the resulting formula is well-formed. Secondly it preserves the validity of the formula. Let us look at the following derivation.

\par
\begin{tabularx}{\linewidth}{rclX}
  $\forall x \ \exists y \ \phi(x, y)$ & $\rightarrow$ & $\exists y \ \phi(z,y)$ 
    & with axiom~\ref{axiom:universalInstantiation} \\
  $\forall x \ \exists y \ \phi(x, y)$ & $\rightarrow$ & $\exists y \ \phi(y,y)$ 
    & forbidden replacement: $z$ in $y$, despite $y$ is already bound \\
  $\forall x \ \exists y \ x \neq y$ & $\rightarrow$ & $\exists y \ \neq y$ 
    & replace $\neq$ for $\phi$
\end{tabularx}

\par
This last proposition is not valid in many models.


\par


\begin{rul}[Rename Bound Subject Variable]
\label{rule:renameBound} \hypertarget{rule:renameBound}{}
We may replace a bound subject variable occurring in a formula by any other subject variable, provided that the new variable occurs not free in the original formula. If the variable to be replaced occurs in more than one scope, then the replacement needs to be made in one scope only.
\end{rul}




\par


\begin{rul}[Replace Predicate Variable]
\label{rule:replacePred} \hypertarget{rule:replacePred}{}
Let $\alpha$ be a true formula that contains a predicate variable $p$ of arity $n$, let $x_1$, \ldots, $x_n$ be subject variables and let $\beta(x_1, \ldots, x_n)$ be a formula where $x_1$, \ldots, $x_n$ are not bound. The formula $\beta(x_1, \ldots, x_n)$ must not contain all $x_1$, \ldots, $x_n$ as free subject variables. Furthermore it can also have other subject variables either free or bound.

If the following conditions are fulfilled, then a replacement of all occurrences of $p(t_1, \ldots, t_n)$ each with appropriate terms $t_1$, \ldots, $t_n$ in $\alpha$ by $\beta(t_1, \ldots, t_n)$ results in another true formula.

\begin{itemize}

\item
the free variables of $\beta(x_1, \ldots, x_n)$ without $x_1$, \ldots, $x_n$ do not occur as bound variables in $\alpha$ 

\item 
each occurrence of $p(t_1, \ldots, t_n)$ in $\alpha$ contains no bound variable of $\beta(x_1, \ldots, x_n)$ 

\item
the result of the substitution is a well-formed formula

\end{itemize}
\end{rul}

See III \S 5 in \cite{hilback}.

\par
The prohibition to use additional subject variables within the replacement formula that occur bound in the original formula assurs that the resulting formula is well-formed. Furthermore it preserves the validity of the formla. Take a look at the following derivation.

\par
\begin{tabularx}{\linewidth}{rclX}
  $ \phi(x)$                             & $\rightarrow$ & $\exists y \ \phi(y)$  
    & with axiom~\ref{axiom:existencialGeneralization} \\
  $ (\exists y \ y = y) \land \phi(x)$   & $\rightarrow$ & $\exists y \ \phi(y)$  
    &  \\
  $ \exists y \ (y = y \land \phi(x))$   & $\rightarrow$ & $\exists y \ \phi(y)$  
    &  \\
  $ \exists y \ (y = y \land x \neq y)$  & $\rightarrow$ & $\exists y \ y \neq y$  
    & forbidden replacment: $\phi(x)$ by $x \neq y$, despite $y$ is already bound \\
  $ \exists y \  x \neq y$  & $\rightarrow$ & $\exists y \ y \neq y$  
    &
\end{tabularx}

\par
The last proposition is not valid in many models.


\par
Analogous to rule~\ref{rule:replacePred} we can replace function variables too.

\begin{rul}[Replace Function Variable]
\label{rule:replaceFunct} \hypertarget{rule:replaceFunct}{}
Let $\alpha$ be an already proved formula that contains a function variable $\sigma$ of arity $n$, let $x_1$, \ldots, $x_n$ be subject variables and let $\tau(x_1, \ldots, x_n)$ be an arbitrary term where $x_1$, \ldots, $x_n$ are not bound. 
The term $\tau(x_1, \ldots, x_n)$ must not contain all $x_1$, \ldots, $x_n$. as free subject variables. Furthermore it can also have other subject variables either free or bound. 

If the following conditions are fulfilled we can obtain a new true formula by replacing each occurrence of $\sigma(t_1, \ldots, t_n)$ with appropriate terms $t_1$, \ldots, $t_n$ in $\alpha$ by $\tau(t_1, \ldots, t_n)$.

\begin{itemize}

\item 
the free variables of $\tau(x_1, \ldots, x_n)$ without $x_1$, \ldots, $x_n$ do not occur as bound variables in $\alpha$

\item
each occurrence of $\sigma(x_1, \ldots, x_n)$ in $\alpha$ contains no bound variable of $\tau(x_1, \ldots, x_n)$

\item
the result of the substitution is a well-formed formula
                  
\end{itemize}
\end{rul}




\par


\begin{rul}[Universal Quantifier Introduction]
\label{rule:universalIntroduction} \hypertarget{rule:universalIntroduction}{}
If $\alpha \rightarrow \beta(x_1)$ is a true formula and $\alpha$ does not contain the subject variable $x_1$, then $\alpha \rightarrow (\forall x_1~(\beta(x_1)))$ is a true 
formula too.
\end{rul}




\par


\begin{rul}[Existential Quantifier Introduction]
\label{rule:existentialIntroduction} \hypertarget{rule:existentialIntroduction}{}
If $\alpha(x_1) \rightarrow \beta$ is already proved to be true and $\beta$ does not contain the subject variable $x_1$, then $(\exists x_1~\alpha(x_1)) \rightarrow \beta$ is also a true formula.
\end{rul}




The usage and elimination of abbreviations and constants is also an inference rule. In many texts about mathematical logic these rules are not explicitly stated and this text is no exception. But in the exact QEDEQ format corresponding rules exist.



%% end of chapter Axioms and Rules of Inference

\chapter{Derived Propositions} \label{chapter5} \hypertarget{chapter5}{}

Now we derive elementary propositions with the axioms and rules of inference of chapter~\ref{chapter4}.

\section{Propositional Calculus} \label{chapter5_section0} \hypertarget{chapter5_section0}{}
At first we look at the propositional calculus.

\par
To define the predicate \emph{true} we just combine a predicate and its negation.

\begin{defn}[True\index{true}]
\label{definition:True} \hypertarget{definition:True}{}
$$\top\ :\leftrightarrow \ A\ \lor \ \neg A$$

\end{defn}




\par
For the predicate \emph{false} we just negate \emph{true}.

\begin{defn}[False\index{false}]
\label{definition:False} \hypertarget{definition:False}{}
$$\bot\ :\leftrightarrow \ \neg \top$$

\end{defn}




\par
We have the following basic propositions.

\begin{prop}[Basic Propositions\index{propositions!of propositional calculus}]
\label{theorem:propositionalCalculus} \hypertarget{theorem:propositionalCalculus}{}
\mbox{}
\begin{longtable}{{@{\extracolsep{\fill}}p{0.9\linewidth}l}}
\centering $\top$ & \label{theorem:propositionalCalculus:aa} \hypertarget{theorem:propositionalCalculus:aa}{} \mbox{\emph{(aa)}} \\
\centering $\neg \bot$ & \label{theorem:propositionalCalculus:ab} \hypertarget{theorem:propositionalCalculus:ab}{} \mbox{\emph{(ab)}} \\
\centering $A\ \rightarrow \ A$ & \label{theorem:propositionalCalculus:ac} \hypertarget{theorem:propositionalCalculus:ac}{} \mbox{\emph{(ac)}} \\
\centering $A\ \leftrightarrow \ A$ & \label{theorem:propositionalCalculus:ad} \hypertarget{theorem:propositionalCalculus:ad}{} \mbox{\emph{(ad)}} \\
\centering $(A\ \lor \ B)\ \leftrightarrow \ (B\ \lor \ A)$ & \label{theorem:propositionalCalculus:ae} \hypertarget{theorem:propositionalCalculus:ae}{} \mbox{\emph{(ae)}} \\
\centering $(A\ \land \ B)\ \leftrightarrow \ (B\ \land \ A)$ & \label{theorem:propositionalCalculus:af} \hypertarget{theorem:propositionalCalculus:af}{} \mbox{\emph{(af)}} \\
\centering $(A\ \land \ B)\ \rightarrow \ A$ & \label{theorem:propositionalCalculus:ag} \hypertarget{theorem:propositionalCalculus:ag}{} \mbox{\emph{(ag)}} \\
\centering $(A\ \leftrightarrow \ B)\ \leftrightarrow \ (B\ \leftrightarrow \ A)$ & \label{theorem:propositionalCalculus:ah} \hypertarget{theorem:propositionalCalculus:ah}{} \mbox{\emph{(ah)}} \\
\centering $(A\ \lor \ (B\ \lor \ C))\ \leftrightarrow \ ((A\ \lor \ B)\ \lor \ C)$ & \label{theorem:propositionalCalculus:ai} \hypertarget{theorem:propositionalCalculus:ai}{} \mbox{\emph{(ai)}} \\
\centering $(A\ \land \ (B\ \land \ C))\ \leftrightarrow \ ((A\ \land \ B)\ \land \ C)$ & \label{theorem:propositionalCalculus:aj} \hypertarget{theorem:propositionalCalculus:aj}{} \mbox{\emph{(aj)}} \\
\centering $A\ \leftrightarrow \ (A\ \lor \ A)$ & \label{theorem:propositionalCalculus:ak} \hypertarget{theorem:propositionalCalculus:ak}{} \mbox{\emph{(ak)}} \\
\centering $A\ \leftrightarrow \ (A\ \land \ A)$ & \label{theorem:propositionalCalculus:al} \hypertarget{theorem:propositionalCalculus:al}{} \mbox{\emph{(al)}} \\
\centering $A\ \leftrightarrow \ \neg \neg A$ & \label{theorem:propositionalCalculus:am} \hypertarget{theorem:propositionalCalculus:am}{} \mbox{\emph{(am)}} \\
\centering $(A\ \rightarrow \ B)\ \leftrightarrow \ (\neg B\ \rightarrow \ \neg A)$ & \label{theorem:propositionalCalculus:an} \hypertarget{theorem:propositionalCalculus:an}{} \mbox{\emph{(an)}} \\
\centering $(A\ \leftrightarrow \ B)\ \leftrightarrow \ (\neg A\ \leftrightarrow \ \neg B)$ & \label{theorem:propositionalCalculus:ao} \hypertarget{theorem:propositionalCalculus:ao}{} \mbox{\emph{(ao)}} \\
\centering $(A\ \rightarrow \ (B\ \rightarrow \ C))\ \leftrightarrow \ (B\ \rightarrow \ (A\ \rightarrow \ C))$ & \label{theorem:propositionalCalculus:ap} \hypertarget{theorem:propositionalCalculus:ap}{} \mbox{\emph{(ap)}} \\
\centering $\neg (A\ \lor \ B)\ \leftrightarrow \ (\neg A\ \land \ \neg B)$ & \label{theorem:propositionalCalculus:aq} \hypertarget{theorem:propositionalCalculus:aq}{} \mbox{\emph{(aq)}} \\
\centering $\neg (A\ \land \ B)\ \leftrightarrow \ (\neg A\ \lor \ \neg B)$ & \label{theorem:propositionalCalculus:ar} \hypertarget{theorem:propositionalCalculus:ar}{} \mbox{\emph{(ar)}} \\
\centering $(A\ \lor \ (B\ \land \ C))\ \leftrightarrow \ ((A\ \lor \ B)\ \land \ (A\ \lor \ C))$ & \label{theorem:propositionalCalculus:as} \hypertarget{theorem:propositionalCalculus:as}{} \mbox{\emph{(as)}} \\
\centering $(A\ \land \ (B\ \lor \ C))\ \leftrightarrow \ ((A\ \land \ B)\ \lor \ (A\ \land \ C))$ & \label{theorem:propositionalCalculus:at} \hypertarget{theorem:propositionalCalculus:at}{} \mbox{\emph{(at)}} \\
\centering $(A\ \land \ \top)\ \leftrightarrow \ A$ & \label{theorem:propositionalCalculus:au} \hypertarget{theorem:propositionalCalculus:au}{} \mbox{\emph{(au)}} \\
\centering $(A\ \land \ \bot)\ \leftrightarrow \ \bot$ & \label{theorem:propositionalCalculus:av} \hypertarget{theorem:propositionalCalculus:av}{} \mbox{\emph{(av)}} \\
\centering $(A\ \lor \ \top)\ \leftrightarrow \ \top$ & \label{theorem:propositionalCalculus:aw} \hypertarget{theorem:propositionalCalculus:aw}{} \mbox{\emph{(aw)}} \\
\centering $(A\ \lor \ \bot)\ \leftrightarrow \ A$ & \label{theorem:propositionalCalculus:ax} \hypertarget{theorem:propositionalCalculus:ax}{} \mbox{\emph{(ax)}} \\
\centering $(A\ \lor \ \neg A)\ \leftrightarrow \ \top$ & \label{theorem:propositionalCalculus:ay} \hypertarget{theorem:propositionalCalculus:ay}{} \mbox{\emph{(ay)}} \\
\centering $(A\ \land \ \neg A)\ \leftrightarrow \ \bot$ & \label{theorem:propositionalCalculus:az} \hypertarget{theorem:propositionalCalculus:az}{} \mbox{\emph{(az)}} \\
\centering $(\top\ \rightarrow \ A)\ \leftrightarrow \ A$ & \label{theorem:propositionalCalculus:ba} \hypertarget{theorem:propositionalCalculus:ba}{} \mbox{\emph{(ba)}} \\
\centering $(\bot\ \rightarrow \ A)\ \leftrightarrow \ \top$ & \label{theorem:propositionalCalculus:bb} \hypertarget{theorem:propositionalCalculus:bb}{} \mbox{\emph{(bb)}} \\
\centering $(A\ \rightarrow \ \bot)\ \leftrightarrow \ \neg A$ & \label{theorem:propositionalCalculus:bc} \hypertarget{theorem:propositionalCalculus:bc}{} \mbox{\emph{(bc)}} \\
\centering $(A\ \rightarrow \ \top)\ \leftrightarrow \ \top$ & \label{theorem:propositionalCalculus:bd} \hypertarget{theorem:propositionalCalculus:bd}{} \mbox{\emph{(bd)}} \\
\centering $(A\ \leftrightarrow \ \top)\ \leftrightarrow \ A$ & \label{theorem:propositionalCalculus:be} \hypertarget{theorem:propositionalCalculus:be}{} \mbox{\emph{(be)}} \\
\centering $((A\ \rightarrow \ B)\ \land \ (B\ \rightarrow \ C))\ \rightarrow \ (A\ \rightarrow \ C)$ & \label{theorem:propositionalCalculus:bf} \hypertarget{theorem:propositionalCalculus:bf}{} \mbox{\emph{(bf)}} \\
\centering $((A\ \leftrightarrow \ B)\ \land \ (B\ \leftrightarrow \ C))\ \rightarrow \ (A\ \leftrightarrow \ C)$ & \label{theorem:propositionalCalculus:bg} \hypertarget{theorem:propositionalCalculus:bg}{} \mbox{\emph{(bg)}} \\
\centering $((A\ \land \ B)\ \leftrightarrow \ (A\ \land \ C))\ \leftrightarrow \ (A\ \rightarrow \ (B\ \leftrightarrow \ C))$ & \label{theorem:propositionalCalculus:bh} \hypertarget{theorem:propositionalCalculus:bh}{} \mbox{\emph{(bh)}} \\
\centering $((A\ \land \ B)\ \leftrightarrow \ (A\ \land \ \neg B))\ \leftrightarrow \ \neg A$ & \label{theorem:propositionalCalculus:bi} \hypertarget{theorem:propositionalCalculus:bi}{} \mbox{\emph{(bi)}} \\
\centering $(A\ \leftrightarrow \ (A\ \land \ B))\ \leftrightarrow \ (A\ \rightarrow \ B)$ & \label{theorem:propositionalCalculus:bj} \hypertarget{theorem:propositionalCalculus:bj}{} \mbox{\emph{(bj)}} 
\end{longtable}

\end{prop}




\section{Predicate Calculus} \label{chapter5_section1} \hypertarget{chapter5_section1}{}
For predicate calculus we achieve the following propositions.

\par
We have the following basic propositions.

\begin{prop}[Basic Propositions\index{propositions!of predicate calculus}]
\label{theorem:predicateCalculus} \hypertarget{theorem:predicateCalculus}{}
\mbox{}
\begin{longtable}{{@{\extracolsep{\fill}}p{0.9\linewidth}l}}
\centering $\forall x\ (\phi(x)\ \rightarrow \ \psi(x))\ \rightarrow \ (\forall x\ \phi(x)\ \rightarrow \ \forall x\ \psi(x))$ & \label{theorem:predicateCalculus:a} \hypertarget{theorem:predicateCalculus:a}{} \mbox{\emph{(a)}} \\
\centering $\forall x\ (\phi(x)\ \rightarrow \ \psi(x))\ \rightarrow \ (\exists x\ \phi(x)\ \rightarrow \ \exists x\ \psi(x))$ & \label{theorem:predicateCalculus:b} \hypertarget{theorem:predicateCalculus:b}{} \mbox{\emph{(b)}} \\
\centering $\exists x\ (\phi(x)\ \land \ \psi(x))\ \rightarrow \ (\exists x\ \phi(x)\ \land \ \exists x\ \psi(x))$ & \label{theorem:predicateCalculus:c} \hypertarget{theorem:predicateCalculus:c}{} \mbox{\emph{(c)}} \\
\centering $(\forall x\ \psi(x)\ \lor \ \forall x\ \psi(x))\ \rightarrow \ \forall x\ (\phi(x)\ \lor \ \psi(x))$ & \label{theorem:predicateCalculus:d} \hypertarget{theorem:predicateCalculus:d}{} \mbox{\emph{(d)}} \\
\centering $\exists x\ (\phi(x)\ \lor \ \psi(x))\ \leftrightarrow \ (\exists x\ \phi(x)\ \lor \ \exists x\ \psi(x))$ & \label{theorem:predicateCalculus:e} \hypertarget{theorem:predicateCalculus:e}{} \mbox{\emph{(e)}} \\
\centering $\forall x\ (\phi(x)\ \land \ \psi(x))\ \leftrightarrow \ (\forall x\ \phi(x)\ \land \ \forall x\ \psi(x))$ & \label{theorem:predicateCalculus:f} \hypertarget{theorem:predicateCalculus:f}{} \mbox{\emph{(f)}} \\
\centering $\forall x\ \forall y\ \phi(x, y)\ \leftrightarrow \ \forall y\ \forall x\ \phi(x, y)$ & \label{theorem:predicateCalculus:g} \hypertarget{theorem:predicateCalculus:g}{} \mbox{\emph{(g)}} \\
\centering $\exists x\ \exists y\ \phi(x, y)\ \leftrightarrow \ \exists y\ \exists x\ \phi(x, y)$ & \label{theorem:predicateCalculus:h} \hypertarget{theorem:predicateCalculus:h}{} \mbox{\emph{(h)}} \\
\centering $\forall x\ (\phi(x)\ \rightarrow \ A)\ \leftrightarrow \ (\forall x\ \phi(x)\ \rightarrow \ A)$ & \label{theorem:predicateCalculus:i} \hypertarget{theorem:predicateCalculus:i}{} \mbox{\emph{(i)}} \\
\centering $\forall x\ (A\ \rightarrow \ \phi(x))\ \leftrightarrow \ (A\ \rightarrow \ \forall x\ \phi(x))$ & \label{theorem:predicateCalculus:j} \hypertarget{theorem:predicateCalculus:j}{} \mbox{\emph{(j)}} \\
\centering $\forall x\ (\phi(x)\ \land \ A)\ \leftrightarrow \ (\forall x\ \phi(x)\ \land \ A)$ & \label{theorem:predicateCalculus:k} \hypertarget{theorem:predicateCalculus:k}{} \mbox{\emph{(k)}} \\
\centering $\forall x\ (\phi(x)\ \lor \ A)\ \leftrightarrow \ (\forall x\ \phi(x)\ \lor \ A)$ & \label{theorem:predicateCalculus:l} \hypertarget{theorem:predicateCalculus:l}{} \mbox{\emph{(l)}} \\
\centering $\forall x\ (\phi(x)\ \leftrightarrow \ A)\ \leftrightarrow \ (\forall x\ \phi(x)\ \leftrightarrow \ A)$ & \label{theorem:predicateCalculus:m} \hypertarget{theorem:predicateCalculus:m}{} \mbox{\emph{(m)}} 
\end{longtable}

\end{prop}




\section{Derived Rules} \label{chapter5_section2} \hypertarget{chapter5_section2}{}
Beginning with the logical basis logical propositions and metarules can be derived an enable a convenient argumentation. Only with these metarules and additional definitions and abbreviations the mathematical world is unfolded. Every additional syntax is \emph{conservative}\index{conservativ}. That means:            within extended system no formulas can be derived, that are written in the old syntax but can not be derived in the old system. In the following such conservative extensions are introduced.

\par


\begin{rul}[Replace by Logical Equivalent Formula]
\label{rule:replaceEquiFormula} \hypertarget{rule:replaceEquiFormula}{}
Let the formula $\alpha \leftrightarrow \beta$ be true. If in a formula $\delta$ we replace an arbitrary occurence of $\alpha$ by $\beta$ and the result $\gamma$ is also a formula\footnote{During that substitution it might be necessary to rename bound variables of $\beta$.} and contains all the free subject variables of $\delta$, then $\delta \leftrightarrow \gamma$ is a true formula.
\end{rul}




\par


\begin{rul}[General Associativity]
\label{rule:generalAssociativity} \hypertarget{rule:generalAssociativity}{}
If an operator of arity two fulfills the associative law it also fulfills the general associative law. The operator can be extended to an operator of arbitrary arity greater one. For example: instead of $(a + b) + (c + d)$ we simply write $a + b + c + d$.\footnote{The operator of arity $n$ is defined with a certain bracketing, but every other bracketing gives the same result.}
\end{rul}




\par


\begin{rul}[General Commutativity]
\label{rule:generalCommutativity} \hypertarget{rule:generalCommutativity}{}
If an operator fulfills the general associative law and is commutative then all permutations of parameters are equal or equivalent.\footnote{That depends on the operator type: term or formula operator.} For example we have: $a + b + c + d  = c + a + d + b$.
\end{rul}




\par


\begin{rul}[Deducible from Formula\index{deducible}]
\label{rule:definitionDeductionFromFormula} \hypertarget{rule:definitionDeductionFromFormula}{}
We shall say that the formula $\beta$ is \emph{deducible from the formula $\alpha$} if the formula $\beta$ from the totality of all true formulas of the predicate calculus and the formula $\alpha$ by means of application of all the rules of the predicate calculus, in which connection both rules for binding by a quantifier, the rules for substitution in place of predicate variables and in place of free subject variables must be applied only to predicate variables or subject variables which do not occur in the formula $\alpha$ and $\alpha \rightarrow \beta$ is a formula. 

\par
Notation: $\alpha \vdash \beta$.
\end{rul}

That a formula $\beta$ is deducible from th formula $\alpha$ must be strictly distinguished from the deduction of a true formula from the axioms of the predicate calculus. In the second case more derivation rules are available. For example if $A$ is added to the axioms then the formula $B$ can be derived. But $B$ is not deducible from $A$.


\par


\begin{rul}[Deduction Theorem\index{deduction theorem}]
\label{rule:deductionTheorem} \hypertarget{rule:deductionTheorem}{}
If the formula $\beta$ is deducible from the formula $\alpha$, then the formula $\alpha \rightarrow \beta$ can be derived from the predicate calculus.
\end{rul}





%% end of chapter Derived Propositions

\chapter{Identity} \label{chapter6} \hypertarget{chapter6}{}

Everything that exists has a specific nature. Each entity exists as something in particular and it has characteristics that are a part of what it is.
Identity is whatever makes an entity definable and recognizable, in terms of possessing a set of qualities or characteristics that distinguish it from entities of a different type. An entity can have more than one characteristic, but any characteristic it has is a part of its identity.

\section{Identity Axioms} \label{chapter6_section0} \hypertarget{chapter6_section0}{}
We start with the identy axioms.

\par
We define a predicate constant of arity two that shall stand for the identity of subjects.

\begin{idefn}[Identity\index{identity}\index{definition!of identity}]
\label{definition:identity} \hypertarget{definition:identity}{}
$$x \ =  \ y$$

\end{idefn}




\par
For convenience we also define the negation of the identity a predicate constant.

\begin{defn}[Not Identical\index{identical!is not}]
\label{definition:notEqual} \hypertarget{definition:notEqual}{}
$$x \ \neq \ y\ :\leftrightarrow \ \neg x \ =  \ y$$

\end{defn}




\par


\begin{ax}[Reflexivity of Identity\index{reflexivity!of identity}\index{identy!reflexivity of}]
\label{axiom:identityIsReflexive} \hypertarget{axiom:identityIsReflexive}{}
\mbox{}
\begin{longtable}{{@{\extracolsep{\fill}}p{\linewidth}}}
\centering $x \ =  \ y$
\end{longtable}

\end{ax}




\par


\begin{ax}[Leibniz' replacement\index{Leibniz' replacement}]
\label{axiom:leibnizReplacement} \hypertarget{axiom:leibnizReplacement}{}
\mbox{}
\begin{longtable}{{@{\extracolsep{\fill}}p{\linewidth}}}
\centering $x \ =  \ y\ \rightarrow \ (\phi(x)\ \rightarrow \ \phi(y))$
\end{longtable}

\end{ax}




\par


\begin{ax}[Symmetrie of identity\index{identity!symmetry of}]
\label{axiom:symmetryOfIdentity} \hypertarget{axiom:symmetryOfIdentity}{}
\mbox{}
\begin{longtable}{{@{\extracolsep{\fill}}p{\linewidth}}}
\centering $x \ =  \ y\ \rightarrow \ y \ =  \ x$
\end{longtable}

\end{ax}




\par


\begin{ax}[Transitivity of identity\index{identity!transetivity of}]
\label{axiom:transitivityOfIdentity} \hypertarget{axiom:transitivityOfIdentity}{}
\mbox{}
\begin{longtable}{{@{\extracolsep{\fill}}p{\linewidth}}}
\centering $(x \ =  \ y\ \land \ y \ =  \ z)\ \rightarrow \ x \ =  \ z$
\end{longtable}

\end{ax}




\par
We can reverse the second implication in the Leibniz replacement.

\begin{prop}
\label{theorem:leibnizEquivalence} \hypertarget{theorem:leibnizEquivalence}{}
\mbox{}
\begin{longtable}{{@{\extracolsep{\fill}}p{\linewidth}}}
\centering $x \ =  \ y\ \rightarrow \ (\phi(x)\ \leftrightarrow \ \phi(y))$
\end{longtable}

\end{prop}




\par


\begin{prop}
\label{theorem:identyImpliesFunctionalEquality} \hypertarget{theorem:identyImpliesFunctionalEquality}{}
\mbox{}
\begin{longtable}{{@{\extracolsep{\fill}}p{\linewidth}}}
\centering $x \ =  \ y\ \rightarrow \ f(x) \ =  \ f(y)$
\end{longtable}

\end{prop}




\section{Restricted Quantifiers\index{quantifiers!restriced}} \label{chapter6_section1} \hypertarget{chapter6_section1}{}
Every quantification involves one specific subject variable and a domain of discourse or range of quantification of that variable. Until now we assumed a fixed domain of discourse for every quantification. Specification of the range of quantification allows us to express that a predicate holds only for a restricted domain.

\par
At the following definition the replacement formula for $\alpha(x)$ must {\glqq reveal\grqq} its quantification subject variable. This is usually the first following subject variable.\footnote{For example: in the following formula we identify the subject variable $m$ for the second quantification: $\forall \ n \in \mathbb{N} \ \forall \ m \in n \ m < n $.} In the exakt syntax of the QEDEQ format\footnote{See \url{http://www.qedeq.org/current/xml/qedeq/}.} the quantification subject variable is always given.

\begin{defn}[Restricted Universal Quantifier\index{universal quantifier, restricted}]
\label{definition:restrictedUniversalQuantifier} \hypertarget{definition:restrictedUniversalQuantifier}{}
$$\forall \ \alpha(x) \ (\beta(x))\ :\leftrightarrow \ \forall x\ (\phi(x)\ \rightarrow \ \psi(x))$$

\end{defn}




\par
A matching definiton for the restricted existential quantifier is the following.\footnote{Matching because of $\neg \forall \ \psi(x) \ (\phi(x)) \leftrightarrow \exists \ x \ \neg (\psi(x) \rightarrow \phi(x)) \leftrightarrow \exists \ x \ (\psi(x) \land \neg\phi(x)) \leftrightarrow \exists \ \psi(x) \ (\neg\phi(x))$.}

\begin{defn}[Restricted Existential Quantifier\index{existential quantifier, restricted}]
\label{definition:restrictedExistentialQuantifier} \hypertarget{definition:restrictedExistentialQuantifier}{}
$$\exists \ \alpha(x) \ (\beta(x))\ :\leftrightarrow \ \exists x\ (\alpha(x)\ \land \ \beta(x))$$

\end{defn}




\par
To express the existence of only one individuum with a certain property we introduce a new Quantor.

\begin{defn}[Restricted Uniqueness Quantifier\index{uniqueness quantifier, restricted}]
\label{definition:restrictedUniquenessQuantifier} \hypertarget{definition:restrictedUniquenessQuantifier}{}
$$\exists! \ \alpha(x) \ (\beta(x))\ :\leftrightarrow \ \exists x\ (\alpha(x)\ \land \ \beta(x))$$

\end{defn}




MISSING! OTHER: F{\"u}r die Existenz genau eines Individuums mit einer bestimmten Eigenschaft wird nun ein gesonderter
Quantor eingef{\"u}hrt.
\begin{defn}[Eingeschr{\"a}nkter Existenzquantor f{\"u}r genau ein Individuum]
$$ \exists! \ \psi(x) \ (\phi(x)) \ \Leftrightarrow \ \exists \ \psi(x) \ (\phi(x) \land \forall \ \psi(y) \ (\phi(y) \rightarrow x = y))$$
\end{defn}
Durch die G{\"u}ltigkeit von $\exists! \ \psi(x) (\phi(x))$ kann daher eine Subjektkonstante definiert
werden, wenn $\phi(x)$ und $\psi(x)$ durch Ausdr{\"u}cke ersetzt werden, die ausser $x$ keine freien
Variablen, keine Pr{\"a}dikatsvariablen und keine Funktionsvariablen mehr enth{\"a}lt.


\begin{rul}[Termdefinition durch Formel]\hypertarget{termdef}{}
Falls die Formel $\exists! x \ \alpha(x)$ gilt, dann kann die Termsyntax durch $D(x, \alpha(x))$
erweitert werden. Die Formel $alpha(x)$ m{\"o}ge die Variable $y$ nicht enthalten und $\beta(y)$ sei
eine Formel, welche die Variable $x$ nicht enth{\"a}lt. Dann wird durch $\beta(D(x, \alpha(x)))$ eine
Formel definiert durch $\beta(y) \land \exists! x \ (\alpha(x) \land x = y)$. Auch in der
abk{\"u}rzenden Schreibweise gilt die Subjektvariable $x$ als gebunden, die Subjektvariable $y$ ist mit
den obigen Einschr{\"a}nkungen frei w{\"a}hlbar und wird in der Abk{\"u}rzung nicht weiter beachtet.
Ver{\"a}nderungen von $\alpha$ in eine andere Formel $\alpha'$, die eventuell erforderlich sind, damit
keine Variablenkollisionen mit Variablen aus $\beta$ entstehen, m{\"u}ssen jedoch auch in der Abk{\"u}rzung
durchgef{\"u}hrt werden. Alle Termbildungsregeln werden entsprechend erweitert. Der Ausdruck ist auch
ersetzbar durch $\exists! y \ (\beta(y) \land \alpha(y)$ oder durch $\beta(y) \land \alpha(y)$.
\end{rul}


F{\"u}r eingeschr{\"a}nkte Quantoren gelten analog zu \hyperref{elmpred} entsprechende Formeln.
\\
+++



%% end of chapter Identity

\begin{thebibliography}{99}
\addcontentsline{toc}{chapter}{Bibliography}
\bibitem{witheruss} \emph{A.N. Whitehead, B. Russell}, Principia Mathematica, Cambridge University Press, London 1910

\bibitem{bernays} \emph{P. Bernays}, Axiomatische Untersuchung des Aussagen-Kalkuls der {\glqq Principia Mathematica\grqq}, Math. Zeitschr. 25 (1926), 305-320

\bibitem{hilback} \emph{D. Hilbert, W. Ackermann}, Grundz{\"u}ge der theoretischen Logik, 2nd ed., Berlin: Springer, 1938. English version: Principles of Mathematical Logic, Chelsea, New York 1950, ed. by R.~E.~Luce.
See also \url{http://www.math.uwaterloo.ca/~snburris/htdocs/scav/hilbert/hilbert.html}

\bibitem{novikov} \emph{P.S. Novikov}, Elements of Mathematical Logic, Edinburgh: Oliver and Boyd, 1964.

\bibitem{mendelson} \emph{E. Mendelson}, Introduction to Mathematical Logic, 3rd. ed., Belmont, CA: Wadsworth, 1987.

\bibitem{guenter} \emph{V.~G{\"u}nther}, Lecture {\glqq Mathematik und Logik\grqq}, given at the University of Hamburg, 1994/1995.

\bibitem{meyling} \emph{M. Meyling}, Hilbert II, Presentation of Formal Correct Mathematical Knowledge, Basic Concept,
\url{http://www.qedeq.org/current/doc/project/qedeq_basic_concept_en.pdf}.



%% QEDEQ modules that use this one:
\bibitem{qedeq_set_theory_v1} qedeq\_set\_theory\_v1 \url{http://qedeq.org/0_03_07/doc/math/qedeq_set_theory_v1.xml}


\end{thebibliography}
\backmatter

\addcontentsline{toc}{chapter}{\indexname} \printindex

\end{document}

