% -*- TeX:DE -*-
%%% ====================================================================
%%% @LaTeX-file $RCSfile: mengenlehre_1.tex,v $
%%% $Revision: 1.1 $
%%% ====================================================================

%%% Permission is granted to copy, distribute and/or modify this document
%%% under the terms of the GNU Free Documentation License, Version 1.2
%%% or any later version published by the Free Software Foundation;
%%% with no Invariant Sections, no Front-Cover Texts, and no Back-Cover Texts.

\newif\ifpdf
\ifx\pdfoutput\undefined
   \pdffalse			   % we are not running PDFLaTeX
\else
   \pdftrue				   % we assume that we will run PDFLaTeX
\fi

\documentclass[a4paper,german,10pt,twoside]{book}

\usepackage[german]{babel}
\usepackage{makeidx}
\usepackage{amsmath,amsthm,amssymb}
\usepackage{color}
\usepackage[bookmarks,bookmarksnumbered,bookmarksopen,
	   colorlinks,linkcolor=webgreen,pagebackref]{hyperref}
%%		 colorlinks,linkcolor=webgreen,filecolor=webbrown,citecolor=webgreen,
\definecolor{webgreen}{rgb}{0,.5,0}
%%\definecolor{webbrown}{rgb}{.6,0,0}
%%\definecolor{webyellow}{rgb}{0.98,0.92,0.73}
%%\definecolor{webgray}{rgb}{.753,.753,.753}
%%\definecolor{webblue}{rgb}{0,0,.8}

\ifpdf % pdf generation
  \usepackage[pdftex]{graphicx}
  \pdfcompresslevel=9
  \pdfinfo{
	/Author		  (Michael Meyling)
	/Title		  (Mengenlehre I)
	/Subject	  (Das Hilbert II Projekt)
	/CreationDate (D:20031015230000)
	/ModDate	  (D:20051228230000)
	/Creator	  (TeX)						% default: "TeX"
	/Producer	  (pdfTeX)					% default: "pdfTeX" + pdftex version
	/Keywords	  (Qedeq, Hilbert II, Principia Mathematica II, Grobkonzept, Mathematik, Grundlagen,
				   Axiome, Logik, Mengenlehre, Dokumentation)
  }
  \pdfcatalog{								% Catalog dictionary of PDF output.
	/PageMode /UseOutlines					%
	/URI (http://www.qedeq.org/)
% pdfscreen-like setting might look like:
%	  /PageMode /none
%	  /ViewerPreferences <<
%		  /HideToolbar true
%		  /HideMenubar true
%		  /HideWindowUI true
%		  /FitWindow true
%		  /CenterWindow true
%	  >>
}
\else % no pdf generation
  \usepackage{graphicx}
\fi

\usepackage{xr}
\usepackage{epsfig,longtable}

\addtolength{\textheight}{7\baselineskip}
\addtolength{\topmargin}{-5\baselineskip}

\setlength{\parindent}{0pt}					% Einrueckung 1. Zeile eines Absatzes
\frenchspacing \sloppy


% Theorem environments
\newtheorem{thm}{Satz}[section]
\newtheorem{cor}[thm]{Korollar}
\newtheorem{lem}[thm]{Lemma}
\newtheorem{rem}[thm]{Bemerkung}
\newtheorem{prop}[thm]{Proposition}
\newtheorem{ax}{Axiom}

%\theoremstyle{definition}
\newtheorem{idefn}{Initiale Definition}
\newtheorem{defn}{Definition}
\newtheorem{rul}{Regel}
\newtheorem*{notation}{Schreibweise}
\newtheorem*{prf}{Beweis}

% Math definitions
\newcommand{:=}{:=}
\newcommand{\Leftrightarrow}{:\leftrightarrow}
\newcommand{\rightarrow}{\rightarrow}
\newcommand{\leftrightarrow}{\leftrightarrow}
%\newcommand{\land}{\ \wedge \ }
%\newcommand{\lor}{\ \vee \ }

% Widerspruchszeichen
\newcommand{\err}{
  \begin{minipage}{\linewidth}
  %$\underset{\swarrow}{\angle}$
  $\hspace*{2.5pt} \angle \vspace*{-5.5pt} \\ \swarrow$
  \end{minipage}
}
\newcommand{\stimes}{\stackrel{\rm S} \times}
\newcommand{\mathbb{N}}{\mathbb{N}}
\newcommand{\Z}{\mathbb{Z}}
\newcommand{\class}[2]{\left\lfloor #1 \right\rfloor_{#2}}
\newcommand{\successor}[1]{\successor(#1)}
\newcommand{\isSet}[1]{\isSet(#1)}
\newcommand{\russell}{\russell}
\newcommand{\allClass}{\allClass}

\newcommand{\lref}[1]{(\ref{#1})}

\DeclareMathOperator{\id}{id} 
\DeclareMathOperator{\power}{\mathfrak P}
\DeclareMathOperator{\rel}{Rel} 
\DeclareMathOperator{\fkt}{Fkt} 
\DeclareMathOperator{\dmn}{Dmn}
\DeclareMathOperator{\rng}{Rng}
%\DeclareMathOperator{\isSet}{\isSet}


\title{
\textbf{Mengenlehre I} \\
\vspace*{1cm} Formal korrektes mathematisches Wissen\\
\vspace*{1cm} {\large Bestandteil des Projekts \\
\textbf{Hilbert II}} \\
\vspace*{0.1cm} {\large Dokumentversion 0.23}}
\author{Michael Meyling}
\date{28.~Dezember~2005}

\makeindex


\begin{document}

\maketitle

\setlength{\parskip}{5pt plus 2pt minus 1pt}  % Abstand zwischen Absaetzen
Homepage von \textbf{Hilbert II}: \url{http://www.qedeq.org/index_de.html}
\par
\vfill
\par
Dieses Dokument ist in elektronischer und aktualisierter Form an dem folgenden Ort zu finden:
\par
\url{http://www.qedeq.org/current/doc/math/mengenlehre_1.pdf}
\par
Die vorliegende Publikation ist urheberrechtlich gesch{\"u}tzt:
\par
Copyright \copyright{} 2005 Michael Meyling. Permission is granted to copy, distribute and/or
modify this document under the terms of the GNU Free Documentation License, Version 1.2 or any
later version published by the Free Software Foundation; with no Invariant Sections, no Front-Cover
Texts, and no Back-Cover Texts. A copy of the license is included in
appendix~\ref{GNU_Free_Documentation_License} entitled \emph{GNU Free Documentation License}.
\par
Die Wiedergabe von Gebrauchsnamen, Handelsnamen, Warenbezeichnungen usw. in diesem Werk berechtigt
auch ohne besondere Kennzeichnung nicht zu der Annahme, dass solche Namen im Sinne der
Warenzeichen- und Markenschutz-Gesetzgebung als frei zu betrachten w{\"a}ren. Die in diesem Text
erw{\"a}hnten Software- und Hardwarebezeichnungen sind in den meisten F{\"a}llen auch eingetragene
Warenzeichen und unterliegen als solche den gesetzlichen Bestimmungen.

\setlength{\parskip}{0pt}					  % Abstand zwischen Absaetzen
\tableofcontents
\setlength{\parskip}{5pt plus 2pt minus 1pt}  % Abstand zwischen Absaetzen


\chapter*{Zusammenfassung\index{Zusammenfassung}}
\addcontentsline{toc}{chapter}{Zusammenfassung} Das Projekt \textbf{Hilbert II}, besch{\"a}ftigt sich
mit der formalen Darstellung und Dokumentation von mathematischem Wissen. Dazu stellt
\textbf{Hilbert II} eine Programmsuite zur L{\"o}sung unterschiedlicher Aufgaben bereit. Auch die
konkrete Dokumentation mathematischen Grundlagenwissens mit diesen Hilfsmitteln geh{\"o}rt zum Ziel
dieses Projekts.
\par
Dieses Dokument beschreibt die mathematischen Grundlagen der Mengenlehre. Ziel ist dabei die
Bereitstellung von elementaren Ergebnissen der Mengenlehre, die in anderen mathematischen
Disziplinen ben{\"o}tigt werden. Zu Beginn werden die formal logischen Regeln aufgestellt, nach denen
vorgegangen wird. Nach den Grundlagen wird die Boolsche Algebra der Klassen betrachtet. Es
schliessen sich Betrachtungen {\"u}ber Relationen und Funktionen an. Ein weiteres wichtiges Ergebnis
sind die Definition der nat{\"u}rlichen Zahlen und die Erf{\"u}llung der Peano-Axiome durch diese, auch auf
den Begriff der Rekursion wird eingegangen.
\par
Die Darstellung erfolgt in axiomatischer Weise und in formaler Form. Dazu wird ein Kalk{\"u}l
angegeben, der es gestattet alle wahren Formeln abzuleiten. Weitere abgeleitete Regeln,
Definitionen, Abk{\"u}rzungen und Syntaxerweiterungen entsprechen im Wesentlichen der mathematischen
Praxis. Daher wird auch das Axiomensystem der Mengenlehre von \emph{A.~P.~Morse} und
\emph{J.~L.~Kelley} verwendet.
\par
W{\"a}hrend der Fertigstellung dieses Dokuments wird mit der Korrektur und Erg{\"a}nzung der formalen
Qedeq-Syntax von \textbf{Hilbert II} begonnen, um eine {\"U}bertragung dieses Texts in die formale
Sprache vorzunehmen. F{\"u}r weitere Information {\"u}ber das Projekt \textbf{Hilbert II} siehe auch unter
\url{http://www.qedeq.org/index_de.html}.
\par
Dieses Dokument ist noch sehr in Arbeit und wird von Zeit zu Zeit aktualisiert. Insbesondere werden
an den durch {\glqq+++\grqq} gekennzeichneten Stellen noch Erg{\"a}nzungen oder {\"A}nderungen vorgenommen.


\chapter*{Vorwort}
Das ganze Universium der Mathematik kann mit den Mitteln der Mengenlehre entfaltet werden. Au{\ss}er
den Axiomen der Mengenlehre werden dazu nur noch logische Axiome und Regeln ben{\"o}tigt. Diese
elementaren Grundlagen gen{\"u}gen, um die komplexesten mathematischen Strukturen zu definieren und
S{\"a}tze {\"u}ber solche Strukturen beweisen zu k{\"o}nnen. Dieses Vorgehen l{\"a}sst sich vollst{\"a}ndig
formalisieren und auf die einfache Manipulation von Zeichenketten zur{\"u}ckf{\"u}hren. Die inhaltliche
Deutung der Zeichenfolgen stellt dann das mathematische Universum dar.

Dabei ist es nat{\"u}rlich mehr als nur bequem, Abk{\"u}rzungen einzuf{\"u}hren und weitere abgeleitete Regeln
zu verwenden. Diese Bequemlichkeiten k{\"o}nnen aber jederzeit\footnote{Zumindest ist eine solche
R{\"u}ckf{\"u}hrung theoretisch immer m{\"o}glich. Praktisch kann sie jedoch an der Endlichkeit der zur
	Verf{\"u}gung stehenden Zeit und des nutzbaren Raums scheitern. So wird es sicherlich nicht m{\"o}glich
sein, die nat{\"u}rliche Zahl $1.000.000.000$ in Mengenschreibweise anzugeben.} eliminiert und durch
die grundlegenden Begrifflichkeiten ersetzt werden.
\par
Dieses Projekt entspringt meinem Kindheitstraum eine solche Formalisierung konkret vorzunehmen.
Inzwischen sind die technischen M{\"o}glichkeiten so weit entwickelt, dass eine Realisierung m{\"o}glich
ist.
\par
Dank geb{\"u}hrt den Professoren \emph{W.~Kerby} und \emph{V.~G{\"u}nther} der Hamburger Universit{\"a}t f{\"u}r
ihre inspirierenden Vorlesungen zu den Themen Logik und Axiomatische Mengenlehre. Ohne diese
entscheidenden Impulse h{\"a}tte es dieses Projekt nie gegeben.
\par
Vielen Dank auch an \emph{F.~Fritsche} f{\"u}r einige Korrekturvorschl{\"a}ge und inhaltliche Anregungen.
\par
Besondere Dank geht an meine Frau \emph{Gesine~Dr{\"a}ger} und unseren Sohn \emph{Lennart} f{\"u}r ihre
Unterst{\"u}tzung und ihr Verst{\"a}ndnis f{\"u}r ihnen fehlende Zeit -- wobei der Verst{\"a}ndnisgrad unseres
S{\"a}uglings vielleicht noch nicht so stark ausgepr{\"a}gt ist.
\par
\vspace*{1cm} Hamburg, im August 2005 \\
\hspace*{\fill} Michael Meyling


\chapter*{Einleitung}
An den Anfang sei ein Zitat aus einem von \emph{D. Hilbert} im September 1922 gehaltenen
Vortrag\footnote{Vortrag, gehalten in der Deutschen Naturforscher-Gesellschaft. September 1922.}
mit dem programmatischen Titel {\glqq Die logischen Grundlagen der Mathematik\grqq} gesetzt.
\par
\begin{quote} {
\glqq Der Grundgedanke meiner Beweistheorie ist folgender:
\par
Alles, was im bisherigen Sinne die Mathematik
ausmacht, wird streng formalisiert, so da{\ss} die eigentliche Mathematik oder die Mathematik in
engerem Sinne zu einem Bestande an Formeln wird. Diese unterscheiden sich von den gew{\"o}hnlichen
Formeln der Mathematik nur dadurch, da{\ss} au{\ss}er den gew{\"o}hnlichen Zeichen noch die logischen Zeichen,
insbesondere die f{\"u}r {\glqq folgt\grqq} ($\rightarrow$) und f{\"u}r {\glqq nicht\grqq} ($\bar{\quad}$)
darin vorkommen. Gewisse Formeln, die als Bausteine des formalen Geb{\"a}udes der Mathematik dienen,
werden Axiome genannt. Ein Beweis ist die Figur, die uns als solche anschaulich vorliegen mu{\ss}; er
besteht
aus Schl{\"u}ssen verm{\"o}ge des Schlu{\ss}schemas\\
\begin{eqnarray*}
& A & \\
& A \rightarrow B& \\
\cline{2-3}
 & B &
\end{eqnarray*}
wo jedesmal die Pr{\"a}missen, d.~h. die betreffenden Formeln $A$ und $A \rightarrow B$ jede entweder
ein Axiom ist bzw. direkt durch Einsetzung aus einem Axiom entsteht oder mit der Endformel $B$
eines Schlusses {\"u}bereinstimmt, der vorher im Beweise vorkommt bzw. durch Einsetzung aus einer
solchen Endformel entsteht. Eine Formel soll beweisbar hei{\ss}en, wenn sie entweder ein Axiom ist bzw.
durch Einsetzen aus einem Axiom entsteht oder die Endformel eines Beweises ist.\grqq}
\end{quote}
\par
In dem 1928 erschienenen Buch \emph{Grundz{\"u}ge der theoretischen Logik} formulierten
\emph{D.~Hilbert} und \emph{W.~Ackermann} ein axiomatisches System der Aussagenlogik, welches die
Ausgangsbasis f{\"u}r das hier verwendete bildet. Durch das von \emph{P.~S.~Novikov} 1959 angegebene
Axiomensystem und Regelwerk der Pr{\"a}dikatenlogik wird das System verfeinert.
\par
Die in diesem Dokument zur Darstellung verwendete formale Sprache erf{\"a}hrt bereits einige
Vereinfachungen\footnote{Z.~B. die Ber{\"u}cksichtigung von Operatorpriorit{\"a}ten bei der Weglassung von
Klammern.} und ist daher von der DV-technischen formalen Sprache verschieden. Das dort verwendete
Qedeq-Format besitzt jedoch eine sehr grosse Affinit{\"a}t zur Darstellung in diesem Text. Die
{\"U}berpr{\"u}fung der formalen Beweise erfolgt jedoch in dem elektronisch auswertbaren Qedeq-Format. F{\"u}r
weitere Information {\"u}ber dieses Format siehe auch unter
\url{http://www.qedeq.org/0_01_05/projektbeschreibung.pdf}.

\chapter{Logik} \label{Logik}
In diesem Kapitel wird der Pr{\"a}dikatenkalk{\"u}l erster Stufe mit Identit{\"a}t und Funktoren vorgestellt,
der die Ausgangsbasis f{\"u}r die Entwicklung der mathematischen Theorie darstellt. Da dieses Dokument
die Mengenlehre zum Schwerpunkt hat, werden hier nur die Ergebnisse ohne weitere Beweise und in
knapper Form pr{\"a}sentiert.

\section{Logische Axiome}\index{Logik}\index{Axiome!logische} \label{Logische_Grundlagen}
Die Grundlagen der bei \textbf{Hilbert II} verwendeten Logik werden hier zusammengestellt. Die
folgende Kalk{\"u}lsprache und ihre Axiome basieren auf den Formulierungen von \emph{D.~Hilbert},
\emph{W.~Ackermann}, \emph{P.~Bernays} und \emph{P.~S.~Novikov}. Aus den hier angegebenen logischen
Axiomen und den elementaren Schlussregeln k{\"o}nnen weitere Gesetzm{\"a}{\ss}igkeiten abgeleitet werden. Erst
diese neuen Metaregeln f{\"u}hren zu einer komfortablen logischen Argumentation.
\par
Als Symbole kommen die logischen Symbole $L = \{$ \mbox{`$\neg$'}, \mbox{`$\vee$'},
\mbox{`$\wedge$'}, \mbox{`$\leftrightarrow$'}, \mbox{`$\rightarrow$'}, \mbox{`$\forall$'},
\mbox{`$\exists$'} $\}$, die
Pr{\"a}dikatenkonstanten\index{Pr{\"a}dikatenkonstante}\index{Konstante!Pr{\"a}dikaten-} $C = \{c^k_i~|~i, k
\in \omega\}$, die
Funktionsvariablen\index{Funktionsvariablen}\index{Variable!Funktions-}\footnote{Funktionsvariablen
dienen der einfacheren Notation und werden beispielsweise zur Formulierung eines
identit{\"a}tslogischen Satzes ben{\"o}tigt: $x = y \rightarrow f(x) = f(y)$. Ausserdem bereitet ihre Einf{\"u}hrung
die sp{\"a}tere Syntaxerweiterung zur Anwendung von funktionalen Klassen vor.} $F = \{f^k_i~|~i, k \in
\omega \land k
> 0\}$, die
Funktionskonstanten\index{Funktionskonstanten}\index{Konstante!Funktions-}\footnote{Funktionskonstanten
dienen ebenfalls der Bequemlichkeit und werden sp{\"a}ter f{\"u}r direkt definierte Klassenfunktionen
verwendet. So zum Beispiel zur Potenzklassenbildung, zur Vereinigungsklassenbildung und f{\"u}r die
Nachfolgerfunktion. All diese Funktionskonstanten k{\"o}nnen auch als Abk{\"u}rzungen verstanden werden.}
$H = \{h^k_i~|~i, k \in \omega\}$, die
Subjektvariablen\index{Subjektvariable}\index{Variable!Subjekt-} V = $\{v_i~|~i \in \omega\}$,
sowie die Pr{\"a}dikatenvariablen\index{Pr{\"a}dikatenvariable}\index{Variable!Pr{\"a}dikaten-} $P =
\{p^k_i~|~i, k \in \omega\}$ vor.\footnote{Unter $\omega$ werden die nat{\"u}rlichen Zahlen, die Null
eingeschlossen, verstanden. Alle bei den Mengenbildungen beteiligten Symbole werden als paarweise
verschieden vorausgesetzt. Das bedeutet z.~B.: $f^k_i = f^{k'}_{i'} \rightarrow (k = k' \land i = i')$
und $h^k_i \neq v_j$.} Unter der \emph{Stellenzahl} eines Operators wird der obere Index
verstanden. Die Menge der nullstelligen Pr{\"a}dikatenvariablen wird auch als Menge der
Aussagenvariablen\index{Aussagenvariable}\index{Variable!Aussagen-} bezeichnet: $A := \{p_i^0~|~i
\in \omega \}$. F{\"u}r die Subjektvariablen werden abk{\"u}rzend auch bestimmte Kleinbuchstaben
geschrieben. Die Kleinbuchstaben stehen f{\"u}r verschiedene Subjektvariablen: \mbox{$v_1 = $ `$u$'},
\mbox{$v_2 = $ `$v$'}, \mbox{$v_3 = $ `$w$'}, \mbox{$v_4 = $ `$x$'}, \mbox{$v_5 = $ `$y$'},
\mbox{$v_5 = $ `$z$'}. Weiter werden als Abk{\"u}rzungen verwendet: f{\"u}r die Pr{\"a}dikatenvariablen $p^n_1
= $ `$\phi$' und $p^n_2 = $ `$\psi$', wobei die jeweilige Stellenanzahl $n$ aus der Anzahl der
nachfolgenden Parameter ermittelt wird, f{\"u}r die Aussagenvariablen $a_1 = $ `$A$', $a_2 = $ `$B$'
und $a_3 = $ `$C$'. Als Abk{\"u}rzungen f{\"u}r Funktionsvariablen wird festgelegt $f^n_1 = $ `$f$' und
$f^n_2 = $ `$g$', wobei wiederum die jeweilige Stellenanzahl $n$ aus der Anzahl der nachfolgenden
Parameter ermittelt wird. Bei allen aussagenlogischen zweistelligen Operatoren wird der leichteren
Lesbarkeit wegen die Infixschreibweise benutzt, dabei werden die Symbole `(' und `)' verwandt.
D.~h. f{\"u}r den Operator $\land$ mit den Argumenten $A$ und $B$ wird $(A \land B)$ geschrieben. Es
gelten die {\"u}blichen Operatorpriorit{\"a}ten und die dazugeh{\"o}rigen Klammerregeln. Insbesondere die
{\"a}u{\ss}eren Klammern werden in der Regel weggelassen.
\par
Nachfolgend werden die Operatoren mit absteigender Priorit{\"a}t aufgelistet.
$$
\begin{array}{c}
  \neg, \forall, \exists  \\
  \land \\
  \lor \\
  \rightarrow, \leftrightarrow \\
\end{array}
$$
\par
Der Begriff \emph{Term\index{Term}} wird im Folgenden rekursiv definiert:
\begin{enumerate}
\item Jede Subjektvariable ist ein Term. \item Seien $i, k \in \omega$ und $t_1$, \ldots, $t_k$
Terme. Dann ist auch $h^k_i(t_1, \ldots, t_k)$ und falls $k > 0$, so auch $f^k_i(t_1, \ldots, t_k)$
ein Term.
\end{enumerate}
Alle nullstelligen Funktionskonstanten $\{h^0_i~|~i, \in \omega\}$ sind demzufolge Terme, sie
werden auch \emph{Individuenkonstanten} genannt.\footnote{Analog dazu k{\"o}nnten Subjektvariablen auch
als nullstellige Funktionsvariablen definiert werden. Da die Subjektvariablen jedoch eine
hervorgehobene Rolle spielen, werden sie auch gesondert bezeichnet.}
\par
Die Begriffe \emph{Formel\index{Formel}}, \emph{freie\index{freie
Subjektvariable}\index{Subjektvariable!freie}} und \emph{gebundene\index{gebundene
Subjektvariable}\index{Subjektvariable!gebundene}} Subjektvariable werden rekursiv wie folgt
definiert:
\begin{enumerate}
\item Jede Aussagenvariable ist eine Formel, solche Formeln enthalten keine freien oder gebundenen
Subjektvariablen. \item Ist $p^k$ eine $k$-stellige Pr{\"a}dikatenvariable und $c^k$ eine $k$-stellige
Pr{\"a}dikatenkonstante und sind $t_1, t_2, \ldots, t_k$ Terme, so sind $p^k(t_1, t_2, \ldots t_k)$ und
$c^k(t_1, t_2, \ldots, t_k)$ Formeln. Dabei gelten alle in $t_1, t_2, \ldots, t_k$ vorkommenden
Subjektvariablen als freie Subjektvariablen, gebundene Subjektvariablen kommen nicht
vor.\footnote{Dieser zweite Punkt umfasst den ersten, welcher nur der Anschaulichkeit wegen extra
aufgef{\"u}hrt ist.} \item Falls die Formel $\alpha$ die freie Subjektvariable $x_1$ enth{\"a}lt (+++ ist
das tats{\"a}chlich erforderlich, dass $x_1$ in $\alpha$ vorkommt?), dann sind auch $\forall
x_1~\alpha$ und $\exists x_1~\alpha$ Formeln\footnote{Dabei wird $\forall$ als
\emph{Allquantor}\index{Allquantor}\index{Quantor!All-} und $\exists$ als
\emph{Existenzquantor}\index{Existenzquantor}\index{Quantor!Existenz-} bezeichnet}, und bis auf
$x_1$ bleiben alle freien Subjektvariablen von $\alpha$ auch frei, und zu den gebundenen
Subjektvariablen von $\alpha$ kommt $x_1$ hinzu. \item Es seien $\alpha, \beta$ Formeln, in denen
keine Subjektvariablen vorkommen, die in einer Formel gebunden und in der anderen frei sind. Dann
sind auch $\neg \alpha$, $(\alpha \land \beta)$, $(\alpha \lor \beta)$, $(\alpha \rightarrow \beta)$,
$(\alpha \leftrightarrow \beta)$ Formeln. Subjektvariablen, welche in $\alpha$, $\beta$ frei (bzw. gebunden)
vorkommen, bleiben frei (bzw. gebunden).
\end{enumerate}
\par
Die folgenden Axiome\index{Axiome!logische} werden an den Anfang gestellt.
\begin{ax}[Oder-K{\"u}rzung]
$$(A \lor A) \rightarrow A$$
\end{ax}

\begin{ax}[Oder-Verd{\"u}nnung]
$$A \rightarrow (A \lor B)$$
\end{ax}

\begin{ax}[Oder-Vertauschung]
$$(A \lor B) \rightarrow (B \lor A)$$
\end{ax}

\begin{ax}[Oder-Vorsehung]
$$(A \rightarrow B) \rightarrow ((C \lor A) \rightarrow (C \lor B))$$
\end{ax}

\begin{ax}
$$(\forall x~\phi(x)) \rightarrow \phi(y)$$
\end{ax}

\begin{ax}
$$\phi(y) \rightarrow (\exists x~\phi(x))$$
\end{ax}

Die Konjunktion, die Implikation und die {\"A}quivalenz werden als Abk{\"u}rzungen
definiert.\footnote{Eigentlich werden die Abk{\"u}rzungssymbole $\wedge, \rightarrow, \leftrightarrow$
erst an dieser Stelle definiert und erweitern die Sprachsyntax. Aus Bequemlichkeitsgr{\"u}nden wurden
diese Symbole bereits als logische Symbole angegeben. Bei den bisher angegebenen Axiomen muss also
strenggenommen die Implikation durch ihre Definition ersetzt werden.}

\begin{defn}[Konjunktion]
$$\alpha \land \beta \ \Leftrightarrow \ \neg(\neg\alpha \lor \neg\beta)$$
\end{defn}

\begin{defn}[Implikation]
$$\alpha \rightarrow \beta \ \Leftrightarrow \ \neg\alpha \lor \beta$$
\end{defn}

\begin{defn}[{\"A}quivalenz]
$$\alpha \leftrightarrow \beta \ \Leftrightarrow \ (\alpha \rightarrow \beta) \land (\beta \rightarrow \alpha)$$
\end{defn}

\begin{rul}[Abtrennung, Modus Ponens\index{Modus Ponens}\index{Abtrennungsregel}]
Wenn $\alpha$ und $\alpha \rightarrow \beta$ wahre Formeln sind, dann ist auch $\beta$ eine wahre Formel.
\end{rul}

\begin{rul}[Einsetzung f{\"u}r Pr{\"a}dikatenvariablen]\label{predreplace}
Es sei $\alpha$ eine Formel, die die $n$-stellige Pr{\"a}dikatenvariable $p$ enth{\"a}lt und $\beta(x_1,
\ldots, x_n)$ eine beliebige Formel mit den freien Variablen $x_1$, \ldots, $x_n$, welche nicht in
$\alpha$ vorkommen.\footnote{In der Formel $\beta(x_1, \ldots, x_n)$ k{\"o}nnen ausser den $n$
Subjektvariablen $x_1$, \ldots, $x_n$ noch weitere Variablen frei vorkommen.} Wenn die folgenden
Bedingungen erf{\"u}llt sind, dann kann durch die Ersetzung jedes Vorkommens von $p(t_1, \ldots, t_n)$
mit jeweils passenden Termen $t_1$, \ldots, $t_n$ in $\alpha$ durch $\beta(t_1, \ldots, t_n)$ eine
weitere wahre Formel gewonnen werden.
\begin{itemize}
\item $\alpha$ ist eine im Kalk{\"u}l wahre Formel \item die freien Variablen von $\alpha$ sind
disjunkt zu den gebundenen Variablen von $\beta(x_1, \ldots, x_n)$ und die gebundenen Variablen von
$\alpha$ disjunkt zu den freien Variablen von $\beta(x_1, \ldots, x_n)$ \item Liegt das zu
ersetzende $p(t_1, \ldots, t_n)$ in $\alpha$ im Wirkungsbereich eines Quantors, so kommt die
zugeh{\"o}rige Subjektvariable in $\beta(x_1, \ldots, x_n)$ nicht vor.\footnote{D.~h. nach dem
Einsetzen werden -- au{\ss}er den in den f{\"u}r $x_1$, \ldots, $x_n$ eingesetzten Termen vorkommenden --
keine weiteren Subjektvariablen gebunden.}
\end{itemize}
\end{rul}

\begin{rul}[Einsetzung f{\"u}r Funktionsvariablen]\label{functreplace}
Es sei $\alpha$ eine Formel, die die $n$-stellige Funktionsvariable $\sigma$ enth{\"a}lt und $\tau(x_1,
\ldots, x_n)$ eine beliebiger Term mit den Subjektvariablen $x_1$, \ldots, $x_n$, welche nicht in
$\alpha$ vorkommen.\footnote{In dem Term $\tau(x_1, \ldots, x_n)$ k{\"o}nnen ausser den $n$
Subjektvariablen $x_1$, \ldots, $x_n$ noch weitere Variablen vorkommen.} Wenn die folgenden
Bedingungen erf{\"u}llt sind, dann kann durch die Ersetzung jedes Vorkommens von $\sigma(t_1, \ldots,
t_n)$ mit jeweils passenden Termen $t_1$, \ldots, $t_n$ in $\alpha$ durch $\tau(t_1, \ldots, t_n)$
eine weitere wahre Formel gewonnen werden.
\begin{itemize}
\item $\alpha$ ist eine im Kalk{\"u}l wahre Formel \item die gebundenen Variablen von $\alpha$ sind
disjunkt zu den Subjektvariablen von $\tau(x_1, \ldots, x_n)$ \item Liegt das zu ersetzende
$\sigma(t_1, \ldots, t_n)$ in $\alpha$ im Wirkungsbereich eines Quantors, so kommt die zugeh{\"o}rige
Subjektvariable in $\beta(x_1, \ldots, x_n)$ nicht vor, d.~h. nach dem Einsetzen werden -- au{\ss}er
den f{\"u}r $x_1$, \ldots, $x_n$ eingesetzten -- keine weiteren Subjektvariablen gebunden.
\end{itemize}
\end{rul}

\begin{rul}[Ersetzung f{\"u}r freie Subjektvariablen]
Ausgehend von einer wahren Formel kann jede freie Subjektvariable durch einen Term ersetzt werden,
der keine in der Formel bereits gebundenen Subjektvariablen enth{\"a}lt. Die Ersetzung muss durchg{\"a}ngig
in der gesamten Formel erfolgen.
\end{rul}

\begin{rul}[Umbenennung f{\"u}r gebundene Subjektvariablen]
Aus jeder im Kalk{\"u}l bereits gewonnenen Formel k{\"o}nnen weitere Formeln abgeleitet werden: Jede
gebundene Subjektvariable kann in eine andere, nicht bereits frei vorkommende, Subjektvariable
umbenannt werden. Die Umbenennung braucht nur im Wirkungsbereich eines bestimmten Quantors zu
erfolgen.
\end{rul}

\begin{rul}[Generalisierung\index{Generalisierung}]
Wenn $\alpha \rightarrow \beta(x_1)$ eine wahre Formel ist und $\alpha$ die Subjektvariable $x_1$ nicht
enth{\"a}lt, dann ist auch $\alpha \rightarrow (\forall x_1~(\beta(x_1)))$ eine wahre Formel.
\end{rul}

\begin{rul}[Partikularisierung\index{Partikularisierung}]
Wenn $\alpha(x_1) \rightarrow \beta$ eine wahre Formel ist und $\beta$ die Subjektvariable $x_1$ nicht
enth{\"a}lt, dann ist auch $(\exists x_1~\alpha(x_1)) \rightarrow \beta$ eine wahre Formel.
\end{rul}

\par
Die Aufl{\"o}sung und der Einsatz von Abk{\"u}rzungen ist auch mit der Anwendung von Regeln verbunden. In
vielen Texten zur mathematischen Logik werden diese Regeln nicht explizit formuliert, auch dieser
Text geht darauf nicht weiter ein. In dem exakten Qedeq-Format gibt es jedoch entsprechende Regeln
zur Verwendung von Abk{\"u}rzungen.



\section{Elementare aussagenlogische S{\"a}tze}
Aus den logischen Axiomen und Regeln von Abschnitt~\ref{Logische_Grundlagen} lassen sich wichtige
elementare S{\"a}tze ableiten.

\begin{defn}
Die Pr{\"a}dikatskonstanten $\top$ f{\"u}r \emph{true} oder \emph{wahr} und $\bot$ f{\"u}r \emph{false} oder
\emph{falsch} werden wie folgt definiert:
\begin{eqnarray}
\top & \Leftrightarrow & A \lor \neg A \\
\bot & \Leftrightarrow & \neg \top
\end{eqnarray}
\end{defn}

Die folgenden Formeln lassen sich beweisen.
\begin{thm}
\begin{eqnarray}
& \top & \\
& \neg \bot & \\
A & \rightarrow & A \\
A & \leftrightarrow & A \\
A \lor B & \leftrightarrow & B  \lor A \\
A \land B & \leftrightarrow & B \land A \\
(A \leftrightarrow B) & \leftrightarrow & (B \leftrightarrow A) \\
A \lor (B \lor C) & \leftrightarrow & (A \lor B) \lor C \\
A \land (B \land C) & \leftrightarrow & (A \land B) \land C \\
A & \leftrightarrow & A \lor A \\
A & \leftrightarrow & A \land A \\
A & \leftrightarrow & \neg \neg A \\
(A \rightarrow B) & \leftrightarrow & (\neg B \rightarrow \neg A) \\
(A \rightarrow (B \rightarrow C)) & \leftrightarrow & (B \rightarrow (A \rightarrow C)) \\
\neg (A \lor B) & \leftrightarrow & \neg A \land \neg B \\
\neg (A \land B) & \leftrightarrow & \neg A \lor \neg B \\
A \lor (B \land C) & \leftrightarrow & (A \lor B) \land (A \lor C) \\
A \land (B \lor C) & \leftrightarrow & (A \land B) \lor (A \land C) \\
A \land \top & \leftrightarrow & A \\
A \land \bot & \leftrightarrow & \bot \\
A \lor \top & \leftrightarrow & \top \\
A \lor \bot & \leftrightarrow & A \\
A \lor \neg A & \leftrightarrow & \top \\
A \land \neg A & \leftrightarrow & \bot \\
(\top \rightarrow A) & \leftrightarrow & A \\
(\bot \rightarrow A) & \leftrightarrow & \top \\
(A \rightarrow B) \land (B \rightarrow C) & \rightarrow & A \rightarrow C \\
(A \leftrightarrow B) \land (B \leftrightarrow C) & \rightarrow & A \leftrightarrow C \label{andequi} \\
((A \land B) \leftrightarrow (A \land C)) & \leftrightarrow & (A \rightarrow (B \leftrightarrow C))
\end{eqnarray}
\end{thm}

\section{Elementare S{\"a}tze der Pr{\"a}dikatenlogik}
Aus den logischen Axiomen und Regeln von Abschnitt~\ref{Logische_Grundlagen} lassen sich auch f{\"u}r
die Pr{\"a}dikatenlogik elementare S{\"a}tze ableiten.

\begin{thm}\label{elmpred}
\begin{eqnarray}
\forall x \ (\phi(x) \rightarrow \psi(x)) & \rightarrow & \forall x \ (\phi(x)) \rightarrow \forall x \ (\psi(x)) \\
\forall x \ (\phi(x) \rightarrow \psi(x)) & \rightarrow & \exists x \ (\phi(x)) \rightarrow \exists x \ (\psi(x)) \\
\exists x \ (\phi(x) \land \psi(x)) & \rightarrow & \exists x \ (\phi(x)) \land \exists x \ (\psi(x)) \\
\forall x \ (\psi(x)) \lor \forall x \ (\psi(x)) & \rightarrow & \forall x \ (\phi(x) \lor \psi(x)) \\
\exists x \ (\phi(x) \lor \psi(x))) & \leftrightarrow & \exists x \ (\phi(x)) \lor \exists x \ (\psi(x)) \\
\forall x \ (\phi(x) \land \psi(x))) & \leftrightarrow & \forall x \ (\phi(x)) \land \forall x \ (\psi(x)) \label{allandpp}\\
\forall x \ \forall y \ (\phi(x, y)) & \leftrightarrow & \forall y \ \forall x \ (\phi(x, y)) \\
\exists x \ \exists y \ (\phi(x, y)) & \leftrightarrow & \exists y \ \exists x \ (\phi(x, y)) \\
\forall x \ (\phi(x) \rightarrow A) & \leftrightarrow & (\forall x \ (\phi(x)) \rightarrow A) \\
\forall x \ (A \rightarrow \phi(x)) & \leftrightarrow & (A \rightarrow \forall x \ (\phi(x))) \\
\forall x \ (\phi(x) \land A) & \leftrightarrow & \forall x \ (\phi(x)) \land A \\
\forall x \ (\phi(x) \lor A) & \leftrightarrow & (\forall x \ (\phi(x)) \lor A) \\
\forall x \ (\phi(x) \leftrightarrow A) & \leftrightarrow & (\forall x \ (\phi(x)) \leftrightarrow A)
\end{eqnarray}
+++ erg{\"a}nzen
\end{thm}

\section{Abgeleitete Regeln}
Aus den logischen Grundlagen lassen sich logische S{\"a}tze und Metaregeln ableiten, die eine bequemere
Argumentation erm{\"o}glichen. Erst mit diesem Regelwerk und zus{\"a}tzlichen Definitionen und Abk{\"u}rzungen
wird die restliche Mathematik entwickelt. Dabei wird stets nur eine
\emph{konservative}\index{konservativ} Erweiterung der bisherigen Syntax vorgenommen. D.~h. in dem
erweiterten System lassen sich keine Formeln ableiten, die in der alten Syntax formuliert, aber
dort nicht ableitbar sind. Im Folgenden werden solche Erweiterungen vorgestellt.

\begin{rul}[Ersetzung durch logisch {\"a}quivalente Formeln]
Sei die Aussage $\alpha \leftrightarrow \beta$ bereits bewiesen. Wird dann aus der Formel $\delta$ eine neue
Formel $\gamma$ dadurch gewonnen, dass ein beliebiges Vorkommen von $\alpha$ durch $\beta$
ersetzt\footnote{Bei dieser Ersetzung kann es erforderlich sein, dass gebundene Variablen von
$\beta$ umbenannt werden m{\"u}ssen, damit sich wieder eine Formel ergibt.} wird und besitzt $\gamma$
zumindest die freien Variablen (+++ unklar!) von $\delta$, dann gilt $\delta \leftrightarrow \gamma$.
\end{rul}

\begin{rul}[Allgemeine Assoziativit{\"a}t]
Falls ein zweistelliger Operator das Assoziativit{\"a}tsgesetz erf{\"u}llt, so erf{\"u}llt er auch das
allgemeine Assoziativit{\"a}tsgesetz. Dem Operator kann dann eine beliebige Stellenanzahl gr{\"o}{\ss}er eins
zugeschrieben werden. So wird beispielsweise anstelle f{\"u}r $(a + b) + (c + d)$ einfach $a + b + c +
d$ geschrieben.\footnote{Der $n$-stellig Operator wird mit einer bestimmten Klammerung definiert,
jede andere Klammerreihenfolge liefert jedoch dasselbe Ergebnis.}
\end{rul}

\begin{rul}[Allgemeine Kommutativit{\"a}t]
Falls ein Operator das allgemeine Assoziativit{\"a}tsgesetz erf{\"u}llt und kommutativ ist, so sind alle
Permutationen von Parameterreihenfolgen einander gleich oder {\"a}quivalent.\footnote{Je nachdem ob es
sich um einen Termoperator oder einen Formeloperator handelt.} So gilt beispielsweise $a + b + c +
d  = c + a + d + b$.
\end{rul}

\begin{defn}[Ableitbarkeit\index{Ableitbarkeit}]
Eine Formel $\beta$ hei{\ss}t \emph{aus der Formel $\alpha$ ableitbar}, wenn sich $\beta$ mit Hilfe
aller Regeln des Pr{\"a}dikatenkalk{\"u}ls und der um $\alpha$ vermehrten Gesamtheit aller wahren Formeln
des Pr{\"a}dikatenkalk{\"u}ls herleitbar und $\alpha \rightarrow \beta$ eine Formel ist. Dabei d{\"u}rfen die beiden
Quantifizierungsregeln, die Einsetzungsregel f{\"u}r Pr{\"a}dikatenvariable und die Umbenennungsregel f{\"u}r
freie Subjektvariable nur auf solche Variablen angewendet werden, die in der Formel $\alpha$ nicht
auftreten.
\par
Schreibweise: $\alpha \vdash \beta$.
\end{defn}
Die Ableitbarkeit einer Formel $\beta$ aus der Formel $\alpha$ ist streng zu unterscheiden von der
Ableitbarkeit einer wahren Formel aus den Axiomen des Kalk{\"u}ls, denn im zweiten Fall stehen mehr
Ableitungsregeln zur Verf{\"u}gung. Falls beispielsweise die Formel $A$ als Axiom aufgenommen wird, so
ist die Formel $A \rightarrow B$ herleitbar. Hingegen l{\"a}{\ss}t sich aus $A$ nicht $B$ ableiten.

\begin{rul}[Deduktionstheorem\index{Deduktionstheorem}]
Wenn eine Formel $\beta$ aus einer Formel $\alpha$ ableitbar ist, so ist die Formel $\alpha \rightarrow
\beta$ im Pr{\"a}dikatenkalk{\"u}l herleitbar.
\end{rul}

\section{Identit{\"a}t}\index{Identit{\"a}t}\label{gleichheit}
Es wird eine zweistellige Funktionskonstante festgelegt, welche in der Interpretation die Identit{\"a}t
von Subjekten ausdr{\"u}cken soll.
\begin{defn}[Gleichheit]
$$x = y \ \Leftrightarrow \ c_1^2(x, y)$$
\end{defn}

Dazu werden zwei weitere Axiome, auch
\emph{Gleichheitsaxiome}\index{Gleichheit}\index{Axiome!Gleichheits-} genannt, formuliert.
\begin{ax}[Reflexivit{\"a}t der Gleichheit]
$$x = x$$
\end{ax}

\begin{ax}[Leibnizsche Ersetzbarkeit]\index{Leibnizsche Ersetzbarkeit}\label{leibniz}
$$x = y	 \rightarrow (\phi(x) \rightarrow \phi(y))$$
\end{ax}

\begin{thm}[Symmetrie der Gleichheit]\index{Gleichheit!Symmetrie der}
\begin{equation}
x = y \leftrightarrow y = x
\end{equation}
\end{thm}

\begin{thm}[Transitivit{\"a}t der Gleichheit]\index{Gleichheit!Transitivit{\"a}t der}
\begin{equation}
x = y ~ \land ~ y = z \rightarrow x = z
\end{equation}
\end{thm}

\begin{thm}
\begin{equation}
x = y \rightarrow (\phi(x) \leftrightarrow \phi(y))
\end{equation}
\end{thm}

\begin{thm}
\begin{equation}
x = y \rightarrow f(x) = f(y)
\end{equation}
\end{thm}

\section{Eingeschr{\"a}nkte Quantoren\index{Quantor!eingeschr{\"a}nkter}}
Bei der folgenden Definition muss die f{\"u}r $\psi(x)$ eingesetzte Formel {\glqq erkennen
lassen\grqq}, {\"u}ber welche Subjektvariable quantifiziert wird. Das ist in der Regel dar{\"u}ber zu
entscheiden, welche freie Subjektvariable als erstes in der Formel
vorkommt.\footnote{Beispielsweise ist in der folgenden Formel erkennbar, dass die zweite
Quantifikation {\"u}ber die Subjektvariable $m$ l{\"a}uft: $\forall \ n \in \mathbb{N} \ \forall \ m \in n \ m < n
$.} In der exakten Syntax des Qedeq-Formats\footnote{Siehe unter
\url{http://www.qedeq.org/0_01_05/projektbeschreibung.pdf}.} ist die Subjektvariable immer
angegeben.
\begin{defn}[Eingeschr{\"a}nkter Allquantor]\index{Allquantor!eingeschr{\"a}nkter}
$$ \forall \ \psi(x) \ (\phi(x)) \ \Leftrightarrow \ \forall \ x \ (\psi(x) \rightarrow \phi(x))$$
\end{defn}

Dazu passt die folgende Definition f{\"u}r den eingeschr{\"a}nkten Existenzquantor.\footnote{Passend, da
$\neg \forall \ \psi(x) \ (\phi(x)) \leftrightarrow \exists \ x \ \neg (\psi(x) \rightarrow \phi(x)) \leftrightarrow \exists
\ x \ (\psi(x) \land \neg\phi(x)) \leftrightarrow \exists \ \psi(x) \ (\neg\phi(x))$.}
\begin{defn}[Eingeschr{\"a}nkter Existenzquantor]\index{Existenzquantor!eingeschr{\"a}nkter}
$$ \exists \ \psi(x) \ (\phi(x)) \ \Leftrightarrow \ \exists \ x \ (\psi(x) \land \phi(x))$$
\end{defn}

F{\"u}r die Existenz genau eines Individuums mit einer bestimmten Eigenschaft wird nun ein gesonderter
Quantor eingef{\"u}hrt.
\begin{defn}[Eingeschr{\"a}nkter Existenzquantor f{\"u}r genau ein Individuum]
$$ \exists! \ \psi(x) \ (\phi(x)) \ \Leftrightarrow \ \exists \ \psi(x) \ (\phi(x) \land \forall \ \psi(y) \ (\phi(y) \rightarrow x = y))$$
\end{defn}
Durch die G{\"u}ltigkeit von $\exists! \ \psi(x) (\phi(x))$ kann daher eine Subjektkonstante definiert
werden, wenn $\phi(x)$ und $\psi(x)$ durch Ausdr{\"u}cke ersetzt werden, die ausser $x$ keine freien
Variablen, keine Pr{\"a}dikatsvariablen und keine Funktionsvariablen mehr enth{\"a}lt.


\begin{rul}[Termdefinition durch Formel]\label{termdef}
Falls die Formel $\exists! x \ \alpha(x)$ gilt, dann kann die Termsyntax durch $D(x, \alpha(x))$
erweitert werden. Die Formel $alpha(x)$ m{\"o}ge die Variable $y$ nicht enthalten und $\beta(y)$ sei
eine Formel, welche die Variable $x$ nicht enth{\"a}lt. Dann wird durch $\beta(D(x, \alpha(x)))$ eine
Formel definiert durch $\beta(y) \land \exists! x \ (\alpha(x) \land x = y)$. Auch in der
abk{\"u}rzenden Schreibweise gilt die Subjektvariable $x$ als gebunden, die Subjektvariable $y$ ist mit
den obigen Einschr{\"a}nkungen frei w{\"a}hlbar und wird in der Abk{\"u}rzung nicht weiter beachtet.
Ver{\"a}nderungen von $\alpha$ in eine andere Formel $\alpha'$, die eventuell erforderlich sind, damit
keine Variablenkollisionen mit Variablen aus $\beta$ entstehen, m{\"u}ssen jedoch auch in der Abk{\"u}rzung
durchgef{\"u}hrt werden. Alle Termbildungsregeln werden entsprechend erweitert. Der Ausdruck ist auch
ersetzbar durch $\exists! y \ (\beta(y) \land \alpha(y)$ oder durch $\beta(y) \land \alpha(y)$.
\end{rul}


F{\"u}r eingeschr{\"a}nkte Quantoren gelten analog zu \ref{elmpred} entsprechende Formeln.
\\
+++


\chapter{Grundlagen}\index{Grundlagen}\index{Grundlagen!der Mengenlehre}
Nach der Entwicklung der \emph{naiven Mengenlehre} (1895 - 1897) durch \emph{G.~Cantor} und der
Entdeckung von Widerspr{\"u}chen durch \emph{B.~Russell} (1902) gab es verschiedene Versuche die
Mengenlehre zu axiomatisieren. Zum einen in der \emph{Typentheorie}\footnote{In der Typentheorie
k{\"o}nnen nur Mengen einer bestimmten Stufe zu Mengen der n{\"a}chsth{\"o}heren Stufe zusammengefa{\ss}t werden.}
(1910) von \emph{A.~N.~Whitehead} und \emph{B.~Russell} in ihrem epochalen Werk {\glqq Principia
Mathematica\grqq}. Zum anderen in der Zermeloschen axiomatischen Mengenlehre (1908) mit der
Fraenkischen Verbesserung\footnote{Verwendung unendlich vieler Axiome, Aussonderungsschema.} (1921)
und der skolemschen Pr{\"a}zisierung der mengentheoretischen Aussagen\footnote{Das Aussonderungsschema
wird durch das Ersetzungsaxiomschema ersetzt.} (1922). Diese sogenannte Zermelo-Fraenkelsche
Mengenlehre (\emph{ZF}) erfuhr durch \emph{K.~G{\"o}del} und \emph{P.~Bernays} eine Verbesserung, die
h{\"a}ufig mit \emph{ZFG}\index{ZFG} abgek{\"u}rzt wird. Die Verwendung dieser f{\"u}r mengentheoretische
Grundlagenforschung immer noch benutzten Theorie ist jedoch f{\"u}r die mathematische Praxis unbequem.
Auch die Erg{\"a}nzungen von \emph{J.~v.~Neumann} und \emph{P.~Bernays} f{\"u}hrten zu einer neuen
Formulierung, welche mit \emph{NBG}\index{NBG} bezeichnet wird. Eine Verallgemeinerung von NBG
wurde von \emph{J.~L.~Kelley} (1955) vorgestellt. In der Tradition dieses auch
\emph{MK}\index{MK}\footnote{Diese Abk{\"u}rzung ist nach \emph{A.~P.~Morse} und \emph{J.~L.~Kelley}
benannt, das Axiomensystem wurde jedoch zuvor auch von anderen Mathematikern vorgestellt.}
genannten Systems steht auch das hier verwendete Axiomensystem der Mengenlehre.
\par
In der naiven Mengenlehre wird der elementare Begriff von \emph{G.~Cantor} (1897) wie folgt erkl{\"a}rt: \\
\emph{\glqq Unter einer {\glqq Menge\grqq} verstehen wir jede Zusammenfassung $M$ von bestimmten
wohlunterscheidbaren Objekten unserer Anschaung oder unseres Denkens (welche die {\glqq
Elemente\grqq} von $M$ genannt werden) zu einem Ganzen.\grqq}
\par
Der bekannte \emph{Russellsche Widerspruch} l{\"a}sst sich nun wie folgt konstruieren. Sei $M$ die
Menge aller Mengen, die sich selbst als Element nicht enthalten: $M = \{X~|~X \not\in X\}$. Nun
stellt sich die Frage, ob $M \in M$ gilt. Angenommen, diese Aussage ist richtig, dann folgt $M
\not\in M$ aus der Eigenschaft die $M$ charakterisiert. Andererseits folgt aus $M \not\in M$ sofort
$M \in M$ aus demselben Grund. Also gilt $M \in M \leftrightarrow M \not\in M$ und diese naive Mengenbildung
f{\"u}hrt zu einem Widerspruch.
\par
Zur Aufl{\"o}sung dieses Widerspruchs muss die Mengenbildung eingeschr{\"a}nkt werden. Dazu wird in MK als
Grundbegriff die \emph{Klasse} gew{\"a}hlt. Bestimmte Klassen zeichnen sich dadurch aus, dass sie das
Pr{\"a}dikat \emph{ist eine Menge} erf{\"u}llen. Die Bildung neuer Klassen ist nur mit Mengen erlaubt:
jedes Element einer Klasse muss eine Menge sein. Durch diese Einschr{\"a}nkung werden alle bekannten
Widerspr{\"u}che vermieden. Die grundlegendsten Axiome und Definitionen der Mengenlehre werden in
diesem Kapitel vorgestellt.

\section{Komprehension und Extensionalit{\"a}t}
\par
Die Interpretationen der Subjektvariablen hei{\ss}en \emph{Klassen}\index{Klasse}.

\par
Als neue zweistellige Pr{\"a}dikatskonstante wird die Elementbeziehung $\in$ vorausgesetzt.
\begin{idefn}[Elementbeziehung]\index{Elementbeziehung}
$$x \in y \ \Leftrightarrow \ c_2^2(x, y)$$
\end{idefn}

Praktisch ist auch eine abk{\"u}rzende Schreibweise f{\"u}r die Negation dieser Relation.

\begin{defn}[Negation der Elementbeziehung]
$$x \not \in y \ \Leftrightarrow \ \neg(x \in y)$$
\end{defn}

\par
Ist eine Klasse Element einer anderen Klasse, dann wird sie auch als \emph{Menge}\index{Menge}
bezeichnet.
\begin{defn}[Mengendefinition\index{Definition!Menge}]
$$ \isSet(x) \ \Leftrightarrow \ \exists y \ ( x \in y)$$
\end{defn}

Der Gleichheitsbegriff\index{Gleichheit} wird durch das
Extentsionalit{\"a}tsaxiom\index{Axiom!Extensionalit{\"a}ts-}\index{Extensionalit{\"a}t}
festgeschrieben.\footnote{Die Umfangsgleichheit kann auch als Definition des Gleichheitspr{\"a}dikates
verwendet werden. Das ist jedoch zum Nachweis der Leibnizschen Ersetzbarkeit nicht ausreichend.
Daf{\"u}r muss die Extensionalit{\"a}t durch das folgende Axiom gesichert werden $(x = y ~ \land ~ x \in z
\rightarrow y \in z$. Damit gen{\"u}gt dann das definierte Gleichheitspr{\"a}dikat allen Anforderungen an den in
Abschnitt \ref{gleichheit} vorgestellten Identit{\"a}tsbegriff. Aus Gr{\"u}nden der formalen Vereinfachung
wird hier jedoch von einem Pr{\"a}dikatenkalk{\"u}l mit Identit{\"a}t ausgegangen.} Die Gleichheit zweier
Klassen soll dann gegeben sein, wenn sie dieselben Elemente besitzen.
\begin{ax}[Extensionalit{\"a}tsaxiom]
$$\forall z \ (z \in x \leftrightarrow z \in y) \rightarrow x = y$$
\end{ax}
\par
Aus der Leibnizschen Ersetzbarkeit Axiom~\ref{leibniz} folgt auch die umgekehrte Richtung, so dass
Folgendes gilt.
\begin{thm}[Extensionalit{\"a}t]
\begin{equation}
x = y \leftrightarrow \forall z \ (z \in x \leftrightarrow z \in y)) \label{extensional}
\end{equation}
\end{thm}

Zur Klassenbildung wird das Komprehensionsaxiom\index{Axiom!Komprehensions-}\index{Komprehension}
ben{\"o}tigt.
\begin{ax}[Komprehensionssaxiom]\label{kompax}
$$\exists x \ \forall y \ ( y \in x \leftrightarrow \isSet(y) \land \phi(y))$$
\end{ax}
Werden an $\phi(y)$ zus{\"a}tzliche Anforderungen gestellt, so bilden die hier aufgelisteten Axiome ein
NBG-System.\footnote{Dazu darf in $\phi(y)$ nur {\"u}ber Mengen quantifiziert werden, d.~h. alle in
$\phi(y)$ vorkommenden Quantoren haben die Form $\forall \ \isSet(z) \ (\psi(z)))$ oder $\exists \
\isSet(z) \ (\psi(z))$ mit entsprechenden $z$ und $\psi(z)$. Falls $\phi(y)$ diese Anforderung
erf{\"u}llt, wird auch von einer \emph{pr{\"a}dikativen} Formel gesprochen.} In den herk{\"o}mmlichen
Formulierungen dieses Axioms wird als zus{\"a}tzliche Einschr{\"a}nkung an $\phi(y)$ gefordert, dass in
dieser Formel die Subjektvariable $x$ nicht vorkommen darf. Diese Forderung ist aufgrund des
Verbots der mehrfachen Quantifizierung mit derselben Subjektvariablen und der
Regel~\ref{predreplace} erf{\"u}llt.

\par
Durch die Extensionalit{\"a}t und das Komprehensionsaxiom wird nun der Zusammenhang zwischen einer
Aussage $\phi(y)$ und der durch diese Aussage definierten Klasse festgelegt. Das
Komprehensionsaxiom behauptet die Existenz mindestens einer durch die Aussage
$\isSet(y)~\land~\phi(y)$ definierten Klasse. Das Extensionalit{\"a}tsaxiom und die Gleichheitsaxiome
sichern ab, dass es h{\"o}chstens eine solche Klasse gibt: irgend zwei Klassen, welche dieselben
Elemente besitzen, sind gleich im Sinne der Ersetzbarkeit in allen einschl{\"a}gigen Aussagen. Mit
anderen Worten: es gibt nur genau eine solche Klasse.
\begin{thm}
\begin{equation}
\exists! x \ \forall y \ (y \in x \leftrightarrow	 \isSet(y) \land \phi(y))
\end{equation}
\end{thm}
\begin{proof}
Zu zeigen ist:
$$
\begin{array}{rl}
\exists x \forall y & (y \in x \leftrightarrow  \isSet(y) \land \phi(y)) \\
\land \ \forall u \ \forall v  & (\forall y \ (y \in u \leftrightarrow \isSet(y) \land \phi(y)) \\
\land & \ \forall y \ ( y \in v \leftrightarrow \isSet(y) \land \phi(y))) \rightarrow u = v)
\end{array}
$$
Seien $u$ und $v$ beliebig. Es gelte weiterhin:\\
$\forall y \ (y \in u \leftrightarrow \isSet(y) \land \phi(y)) \land \ \forall y \ ( y \in v \leftrightarrow \isSet(y)
\land \phi(y)))$ \\
Dann folgt mit \ref{allandpp}: $\forall y \ ((y \in u \leftrightarrow \isSet(y) \land \phi(y)) \land (y \in v \leftrightarrow \isSet(y) \land \phi(y)))$ \\
Daraus erhalten wir mit \ref{andequi}: $\forall y \ ((y \in u \leftrightarrow y \in v ))$. Und mit
Satz~\ref{extensional} folgt nun $u = v$. Also haben wir gezeigt:\\
$\forall u \ \forall v \ (\forall y \ (y \in u \leftrightarrow \isSet(y) \land \phi(y)) \land \ \forall y \ (
y \in v \leftrightarrow \isSet(y) \land \phi(y))) \rightarrow u = v)$
\par
Zusammen mit Axiom~\ref{kompax} folgt nun die Behauptung.
\end{proof}

Deshalb k{\"o}nnen wir eine neue abk{\"u}rzende Schreibweise einf{\"u}hren. Siehe auch Regel~\ref{termdef}.

\begin{rul}[Klassenschreibweise]\label{classcomp}
Der Ausdruck $\{x~|~\phi(x) \}$ bezeichnet einen Term, alle Bildungsregeln werden entsprechend
erweitert. Sei $\beta(x)$ eine Formel und $\beta(\{ x~|~\phi(x) \})$ eine mit diesem Term gebildete
Formel, so wird diese Formel definiert durch $\beta(y) \land \forall x \ (x \in y \leftrightarrow \isSet(x)
\land \phi(x))$ mit einer noch nicht in $\beta$ vorkommenden Subjektvariablen $y$.\footnote{Auch
die Schreibweise $\exists! y \ (\beta(y) \land \forall x \ (x \in y \leftrightarrow \isSet(x) \land
\phi(x)))$ ist m{\"o}glich.} Auch in der abk{\"u}rzenden Schreibweise gilt die Subjektvariable $x$ als
gebunden, die Subjektvariable $y$ ist frei w{\"a}hlbar und wird in der Abk{\"u}rzung nicht weiter beachtet.
Bei der Ersetzung von $\phi$ durch eine andere Formel muss auch in $\{ x~|~\phi(x) \}$ ersetzt
werden.
\end{rul}
Im Folgenden kann diese abk{\"u}rzende Schreibweise in pr{\"a}dikativen Aussagen benutzt werden.

\begin{thm}
Die G{\"u}ltigkeit der Ausgangspr{\"a}dikate dr{\"u}ckt sich f{\"u}r diesen neuen Term wie folgt aus:
\begin{eqnarray}
y \in \{ x~|~\phi(x) \} & \leftrightarrow & \isSet(y) \land \phi(y) \label{inSetEqual} \\
y = \{ x~|~ \phi(x) \} & \leftrightarrow & \forall z \ (z \in y \leftrightarrow z \in \{ x~|~\phi(x) \}) \\
\{ x~|~\phi(x) \} = \{ x~|~\psi(x) \} & \leftrightarrow & \forall z \ (z \in \{ x~|~\phi(x) \} \nonumber \\
  & & \ \leftrightarrow z \in \{x~|~\psi(x) \}) \\
\{ x~|~\phi(x) \} \in \{ x~|~\psi(x) \} & \leftrightarrow & \forall u \ \forall
v \ (u	= \{ x~|~\phi(x) \} \nonumber \\
  & & \ \land \ v = \{ x~|~\psi(x) \}) \rightarrow u \in v \\
\{ x~|~\phi(x) \} \in y & \leftrightarrow & \forall u \ u  = \{ x~|~\phi(x) \}	\rightarrow u \in y
\end{eqnarray}
\end{thm}

%% Im Zweifel l{\"o}schen
%%\begin{thm} \label{inSetEqual}
%%\begin{equation}
%%x \in \{y~|~\phi(y) \} ~\leftrightarrow~ \phi(x)
%%\end{equation}
%%\end{thm}

\begin{thm}
\begin{equation}
\exists! x \ (x = \{y~|~\phi(y) \})
\end{equation}
\end{thm}

\begin{thm}
\begin{equation}
(\phi(x) \leftrightarrow \psi(x)) ~\rightarrow~ (\{y~|~\phi(y)\} = \{y~|~\psi(y)\})
\end{equation}
\end{thm}

Jede Klasse l{\"a}sst sich durch eine Aussage beschreiben, indem auf ihre Elemente Bezug genommen wird.

\begin{thm}
\begin{equation}
\forall x \ (x = \{y~|~y \in x\})
\end{equation}
\end{thm}

\section{Spezielle Klassen}
Die Russellsche Klasse kann nun einfach definiert werden.
\begin{defn}[Russellsche Klasse]
$$ \russell \ := \ \{x~|~x \not\in x\}$$
\end{defn}

Die Russellsche Klasse ist eine \emph{echte} Klasse, d.~h. sie ist keine Menge.
\begin{thm}
\begin{equation}
\neg \ \isSet(\russell)
\end{equation}
\end{thm}
\par
Die Allklasse soll alles m{\"o}gliche umfassen.
\begin{defn}[Allklasse]
$$ \allClass \ := \ \{x~|~x = x\}$$
\end{defn}
\par
Die Allklasse umfasst alle Mengen.
\begin{thm}
\begin{equation}
\allClass = \{x~|~\isSet(x)\}
\end{equation}
\end{thm}
Eine Umformulierung dazu ist der folgende Satz.
\begin{thm}
\begin{equation}
x \in \allClass \leftrightarrow \isSet(x)
\end{equation}
\end{thm}

Entsprechend definieren wir die leere Klasse. Sp{\"a}ter werden wir feststellen, dass die leere Klasse
eine Menge ist. Dazu ben{\"o}tigen wir jedoch weitere Mengenaxiome. Wir bezeichnen diese Klasse jedoch
schon jetzt mit den Worten \emph{leere Menge}.
\begin{defn}[Nullmenge]\index{Nullmenge}\index{Klasse!leere}\index{Menge!leere}
$$ \emptyset \ := \ \{x~|~x \neq x\}$$
\end{defn}

\begin{thm}
\begin{equation}
\forall z \ z \notin \emptyset
\end{equation}
\end{thm}

\begin{thm}
\begin{equation}
\forall z \ (z \notin x) \leftrightarrow	x = \emptyset
\end{equation}
\end{thm}

\par


\chapter{Boolesche Algebra der Klassen}
Die elementaren Operationen von Klassen und ihre Eigenschaften werden nun beschrieben.

Eine Boolesche Algebra, oft auch Boolescher Verband genannt, ist eine spezielle algebraische
Struktur, die die Eigenschaften der logischen Operatoren \emph{und}, \emph{oder}, \emph{nicht}
sowie die Eigenschaften der mengentheoretischen Verkn{\"u}pfungen \emph{Durchschnitt},
\emph{Vereinigung} und \emph{Komplement} abstrahiert.

Sie ist benannt nach \emph{G. Boole}, der sie in der Mitte des 19. Jahrhunderts definierte, um
algebraische Methoden in der Aussagenlogik anwenden zu k{\"o}nnen.

\section{Klassenoperatoren}
Die Schreibweise bzw. Regel \ref{classcomp} erm{\"o}glicht die Definition von Klassenperatoren mithilfe
der logischen Verkn{\"u}pfungen.
\par
Die Vereinigung zweier Klassen besteht aus den Elementen beider Klassen.
\begin{defn}[Vereinigung]\index{Vereinigung}
$$ x \cup y \ := \ \{z~|~z \in x \lor z \in y \}$$
\end{defn}

Entsprechend wird der Durchschnitt zweier Klassen definiert, als Klasse die aus den Elementen
besteht, die in beiden Klassen vorhanden sind.
\begin{defn}[Durchschnitt]\index{Durchschnitt}
$$ x \cap y \ := \ \{z~|~z \in x \land z \in y \}$$
\end{defn}

Auch das Komplement einer Klasse kann einfach definiert werden.
\begin{defn}[Komplement]\index{Komplement}
$$ \overline{x} \ := \ \{z~|~z \not\in x\}$$
\end{defn}

Das relative Komplement einer Klasse wird analog definiert.
\begin{defn}[relatives Komplement]\index{Komplement!relatives}
$$x \setminus y \ := \ x \cap \overline{y}$$
\end{defn}

Ob eine Menge ein Element der Vereinigung zweier Klassen ist, kann nat{\"u}rlich auch direkt angegeben
werden.
\begin{thm}
\begin{equation}
z \in (x \cup y) \leftrightarrow z \in x \lor z \in y
\end{equation}
\end{thm}

Entsprechendes gilt f{\"u}r den Durchschnitt zweier Klassen.
\begin{thm}
\begin{equation}
z \in (x \cap y) \leftrightarrow z \in x \land z \in y
\end{equation}
\end{thm}

Analoges gilt f{\"u}r das Komplement, dort muss jedoch die Mengeneigenschaft explizit abgepr{\"u}ft werden.
\begin{thm}
\begin{equation}
z \in \overline{x} \leftrightarrow \isSet(z) \land z \not\in x
\end{equation}
\end{thm}

F{\"u}r das relative Komplement gilt das Folgende.
\begin{thm}
\begin{equation}
z \in (x \setminus y) \leftrightarrow z \in x \land z \not\in x
\end{equation}
\end{thm}

\par
Die vorherigen S{\"a}tze  zeigen die {\"U}bertragbarkeit der Eigenschaften der logischen Verkn{\"u}pfungen
$\lor$, $\land$ und $\neg$ auf die Klassenverkn{\"u}pfungen $\cup$, $\cap$ und $\bar{~}$. Deshalb
lassen sich die entsprechenden logischen Gesetzm{\"a}ssigkeiten direkt auf die Klassenverkn{\"u}pfungen
{\"u}bertragen.

\begin{thm}
\begin{eqnarray}
x \cup y  & = & y \cup x \\
x \cap y  & = & y \cap x \\
(x \cup y) \cup z & = & x \cup (y \cup z) \\
(x \cap y) \cap z & = & x \cap (y \cap z) \\
x & = & x \cup x \\
x & = & x \cap x \\
\overline{\overline{x}} & = & x\\
\overline{(x \cup y)} & = & \overline{x} \cap \overline{y} \\
\overline{(x \cap y)} & = & \overline{x} \cup \overline{y} \\
x \cup (y \cap z) & = & (x \cup y) \cap (x \cup z) \\
x \cap (y \cup z) & = & (x \cap y) \cup (x \cap z) \\
\overline{\emptyset} = \allClass \\
\overline{\allClass} = \emptyset \\
x \cap \allClass & = & x \\
x \cap \emptyset & = & \emptyset \\
x \cup \allClass & = & \allClass \\
x \cup \emptyset & = & x \\
x \cup \overline{x} & = & \allClass \\
x \cap \overline{x} & = & \emptyset
\end{eqnarray}
\end{thm}

\section{Ordnung}

Wir definieren die Teilklassenrelation wie {\"u}blich.
\begin{defn}[Unterklasse]\index{Unterklasse}\index{Teilklasse}
$$ x \subset y \ \Leftrightarrow \ \forall z \ ( z \in x \rightarrow z \in y)$$
\end{defn}
\par
Diese Relation ist reflexiv, transitiv und antisymmetrisch, definiert also eine teilweise
(partielle) Ordnung mit $\emptyset$ als kleinstem und $\allClass$ als gr{\"o}{\ss}tem Element.
\begin{thm}
\begin{eqnarray}
x & \subset & x \\
x \subset y \ \land y \ \subset z & \rightarrow & x \subset z\\
x \subset y \ \land y \ \subset x & \rightarrow & x = y \\
\emptyset & \subset & x \\
x & \subset & \allClass \\
x \subset \emptyset & \rightarrow & x = 0 \\
\allClass \subset x & \rightarrow & x = \allClass
\end{eqnarray}
\end{thm}

\begin{thm}
\begin{eqnarray}
x \cap y & \subset & x \\
x \cap y & \subset & y
\end{eqnarray}
\end{thm}

\begin{thm}
\begin{eqnarray}
x & \subset & x \cup y \\
y & \subset & x \cup y
\end{eqnarray}
\end{thm}

\begin{thm}
\begin{eqnarray}
x \subset z \ \land \ y \subset z & \leftrightarrow & x \cup y \subset z \\
z \subset x \ \land \ z \subset y & \leftrightarrow & z \subset x \cap y
\end{eqnarray}
\end{thm}

\begin{thm}
\begin{eqnarray}
x \subset y & \rightarrow & x \cup z \subset y \cup z \\
x \subset y & \rightarrow & x \cap z \subset y \cap z
\end{eqnarray}
\end{thm}

\begin{thm}
\begin{equation}
x \subset y \ \leftrightarrow \ \overline{y} \subset \overline{x}
\end{equation}
\end{thm}

\begin{thm}
\begin{eqnarray}
x \subset y & \leftrightarrow & x \cap \overline{y} = \emptyset \\
x \subset y & \leftrightarrow & \overline{x} \cup y = \allClass \\
x \subset \overline{y} & \leftrightarrow & x \cap y = \emptyset \\
x \cap y \subset z & \leftrightarrow & x \subset \overline{y} \cup z
\end{eqnarray}
\end{thm}

\par
Aus der Teilklassenrelation l{\"a}sst sich ein weiterer Klassenoperator gewinnen, die
\emph{Potenzklassenbildung}.

\begin{defn}[Potenzklasse]\index{Potenzklasse}\index{Potenzmenge}
$$
\power(x) \ := \ \{ z~|~z \subset x \}
$$
\end{defn}

F{\"u}r diesen neuen Operator gelten die folgenden Aussagen.
\begin{thm}
\begin{eqnarray}
z \in \power(x) & \leftrightarrow & \isSet(x) \land z \subset x \\
\power(\emptyset) & = & \{\emptyset\} \\
\power(\allClass) & = & \allClass \\
\isSet(x) & \leftrightarrow & x \in \power(x) \\
x \subset y & \rightarrow & \power(x) \subset \power(y) \\
\isSet(x) \land \power(x) \subset \power(y) & \rightarrow & x \subset y \\
\power(x \cap y) & = & \power(x) \cap \power(y) \\
\power(x) \cup \power(y) & \subset & \power(x \cup y)
\end{eqnarray}
\end{thm}


\section{Boolsche Algebra}
Die Klassen bilden mit den Operatoren $\cap$, $\cup$, $\bar{~}$ und den Konstanten $\emptyset$,
$\allClass$ eine boolesche Algebra.
\par
+++ Kommutativit{\"a}t, Assoziativit{\"a}t, Distributivit{\"a}t Idempotenz.

\section{Unendliche boolsche Operatoren}
Es k{\"o}nnen auch beliebige Schnittklassen und Vereinigungsklassen gebildet werden. Dazu muss nur
festgelegt werden, {\"u}ber welche Klassen jeweils geschnitten bzw. vereinigt wird. Das kann im
allgemeinsten Fall durch die Angabe einer Eigenschaft $\phi$ geschehen. Alle Klassen, die diese
Eigenschaft besitzen, werden dann zur Klassenbildung herangezogen.

\begin{defn}[Allgemeiner Durchschnitt]\index{Durchschnitt!allgemeiner}\index{Schnitt!unendlicher}
$$ \bigcap_{\phi(x)} x \ := \ \{y~|~\forall \phi(x) \ (y \in x)\}$$
\end{defn}

\begin{defn}[Allgemeine Vereinigung]\index{Vereinigung!allgemeine}\index{Vereinigung!unendliche}
$$ \bigcup_{\phi(x)} x \ := \ \{y~|~\exists \phi(x) \ (y \in x) \}$$
\end{defn}

In Analogie zu \ref{inSetEqual} gilt nun:
\begin{thm}
\begin{eqnarray}
y \in \bigcap_{\phi(x)} x & \leftrightarrow & \isSet(y) \land \forall \phi(x) \ (y \in x) \label{inUnionEqual}\\
y \in \bigcup_{\phi(x)} x & \leftrightarrow & \exists \phi(x) \ (y \in x) \label{inIntersectionEqual}
\end{eqnarray}
\end{thm}
Die Einschr{\"a}nkung auf Mengen in (\ref{inUnionEqual}) ist notwendig, da sonst bei einem nicht
erf{\"u}llbaren $\phi$ auch alle Klassen $y$, die keine Mengen sind, die rechte Seite der {\"A}quivalenz
erf{\"u}llen w{\"u}rden.
\par
\begin{thm}
\begin{equation}
\phi(x) \rightarrow x \subset \bigcup_{\phi(y)} y
\end{equation}
\end{thm}
\begin{thm}
\begin{equation}
\forall \phi(x) \ (x \subset z) \leftrightarrow \bigcup_{\phi(y)} y \subset z
\end{equation}
\end{thm}
\begin{thm}
\begin{equation}
\forall \phi(x) \ (\bigcap_{\phi(y)} y \subset x)
\end{equation}
\end{thm}
\begin{thm}
\begin{equation}
\forall \phi(x) \ (z \subset x) \leftrightarrow z \subset \bigcap_{\phi(y)} y
\end{equation}
\end{thm}
\begin{thm}
\begin{eqnarray}
x \cup y & = & \bigcup_{z = x ~\lor~ z = y} z \\
x \cap y & = & \bigcap_{z = x ~\land~ z = y} z
\end{eqnarray}
\end{thm}
\begin{thm}
\begin{eqnarray}
x & = & \bigcup_{z = x} z \\
x & = & \bigcap_{z = x} z
\end{eqnarray}
\end{thm}

\begin{thm}
\begin{eqnarray}
\forall x \ (\phi(x) \leftrightarrow \psi(x) & \rightarrow & \bigcup_{\phi(x)} x = \bigcup_{\psi(x)} x \\
\forall x \ (\phi(x) \leftrightarrow \psi(x) & \rightarrow & \bigcap_{\phi(x)} x = \bigcap_{\psi(x)} x
\end{eqnarray}
\end{thm}

\begin{thm}
\begin{eqnarray}
\forall x \ (\phi(x) \leftrightarrow x = z) & \rightarrow & z = \bigcap_{\phi(x)} x \\
\forall x \ (\phi(x) \leftrightarrow x = z) & \rightarrow & z = \bigcup_{\phi(x)} x
\end{eqnarray}
\end{thm}

\begin{thm}
\begin{eqnarray}
\forall x \ (\phi(x) \rightarrow \psi(x) & \rightarrow & \bigcup_{\phi(x)} x \subset \bigcup_{\psi(x)} x \\
\forall x \ (\phi(x) \rightarrow \psi(x) & \rightarrow &\bigcap_{\phi(x)} x \subset \bigcap_{\psi(x)} x
\end{eqnarray}
\end{thm}

\begin{thm}
\begin{equation}
\exists \phi(x) \ (\bigcap_{\phi(x)} x ) \subset \bigcup_{\phi(x)} x
\end{equation}
\end{thm}

\begin{thm}
\begin{eqnarray}
x & \subset & \power(\bigcup_{z \in x} z) \\
\bigcup_{z \in \power(x)} z & \subset & x \\
\forall \, \isSet(x) \quad x & = & \bigcup_{z \in \power(x)} z
\end{eqnarray}
\end{thm}

+++ weitere Rechengesetze, de Morgan etc. \\
+++ muss eventuell in Relationen und Funktionen noch einmal f{\"u}r Indexmengen erg{\"a}nzt werden\\
\par

\section{Mengen}
Zur Darstellung der Booleschen Klassenalgebra wurden noch keine mengentheoretischen Axiome ben{\"o}tigt
Im Folgenden werden weitere Axiome vorgestellt, die Bedingungen daf{\"u}r angeben, wann eine Klasse
eine Menge ist.

\begin{ax}[Axiom der Potenzmenge\index{Axiom!Potenzmengen-}]
$$\forall \isSet(x) \ \exists \isSet(y) \ \forall z \ ( z \subset x \rightarrow z \in y)$$
\end{ax}

\begin{ax}[Axiom der Paarmenge\index{Axiom!Paarmengen-}]
$$\forall \isSet(x) \ \forall \isSet(y) \ \exists \isSet(z) \ ( x \in z \land y \in z)$$
\end{ax}

\begin{ax}[Axiom der Vereinigungsmenge\index{Axiom!Vereinigungsmengen-}]
$$\forall \isSet(x) \ \exists \isSet(y) \ \forall z \ ( z \in x \rightarrow z \subset y)$$
\end{ax}

\par
\begin{thm}
\begin{eqnarray}
\forall \, \isSet(x) & \isSet(\power(x)) \\
\forall \, \isSet(x) \	\forall \, \isSet(y) & \isSet(\{x, y\}) \\
\forall \, \isSet(x) & \isSet(\bigcup_{z \subset x} z) \\
\forall \, \isSet(x) & \isSet(\power( x)) \\
\forall \, \isSet(x) \	\forall \, y \subset x & \isSet(y)
\end{eqnarray}
\end{thm}

\par
\begin{thm}
\begin{eqnarray}
\neg \, \isSet(y) \land y \subset x & \rightarrow & \neg \, \isSet(x) \\			%%% T100
\isSet(x) & \rightarrow & \isSet(\{x\}) \\										%%% T103
\isSet(x) & \rightarrow & \isSet(x \cap y) \\										%%% T104
\isSet(x) & \rightarrow & \neg \, \isSet(\overline{x}) \\							%%% T105
x & = & \bigcup_{z \in \power(x)} z \\										%%% T106
\isSet(x) & \leftrightarrow & \isSet(\bigcup_{z \subset x} z) \\						%%% T107
\isSet(x) & \leftrightarrow & \isSet(\power(x)) \\									%%% T108
\isSet(x \cup y) & \rightarrow & \isSet(x) \\										%%% T109
\neg \, \isSet(x) \land \isSet(y) & \rightarrow & .\neg \, \isSet(x \setminus y)	%%% T110
\end{eqnarray}
\end{thm}


\begin{thm}
\begin{eqnarray}
\isSet(\emptyset)\\
\isSet(1) \\
\isSet(2)
\end{eqnarray}
\end{thm}

\section{Fundierung und Unendlichkeit}
\begin{ax}[Regularit{\"a}tsaxiom]\index{Axiom!Regularit{\"a}ts-}\index{Regularit{\"a}t}\index{Axiom!Fundierungs-}
$$\forall x \ (x \neq \emptyset \rightarrow \exists y \ (y \in x \land y \cap x = \emptyset))$$
\end{ax}

\begin{defn}[Nachfolger]\index{Nachfolger}
$$ \successor(x) \ := \ \{y~|~y \in x \lor y = x \}$$
\end{defn}

\begin{ax}[Unendlichkeitsaxiom]\index{Unendlichkeitsaxiom}\index{Axiom!Unendlichkeits-}
$$\exists \, \isSet(x) \ (\emptyset \in x \land \forall~y \ (y \in x \rightarrow \successor(y) \in x))$$
\end{ax}


\chapter{Relationen und Funktionen}
Um Relationen definieren zu k{\"o}nnen, wird der Begriff des \emph{geordneten Klassenpaares} ben{\"o}tigt,
der es erm{\"o}glicht das \emph{cartesische Produkt} von Klassen zu definieren. Relationen sind
Teilklassen von cartesischen Produkten und bilden eine mit bestimmten Operationen eine
\emph{universelle Algebra}. Spezielle Relationen sind die {\"A}quivalenzrelationen, die einen etwas
weiter gefassten Gleichheitsbegriff erm{\"o}glichen. Funktionen sind ebenfalls spezielle Relationen,
das Fraenkelsche Ersetzungsaxiom garantiert das Mengen auf Mengen abgebildet werden.

\section{Klassenangabe durch Aufz{\"a}hlung}
Eine Klasse kann auch durch Angabe ihrer Elemente definiert werden. Insbesondere kann durch Angabe
eines Elements die sogenannte \emph{Einerklasse} festgelegt werden.
\begin{defn}[Einerklasse]\index{Einerklasse}\index{Klasse!Einer-}
$$
\{x\} \ := \ \{u~|~\isSet(x) \rightarrow u = x \}
$$
\end{defn}
Da der Ausdruck $\{x\}$ f{\"u}r jegliches $x$ definiert ist, kann er auch f{\"u}r den Fall, dass $x$ eine
echte Klasse ist, gebildet werden. In diesem Fall erf{\"u}llen alle Mengen $u$ die Bedingung $\isSet(u)
\land (\isSet(x) \rightarrow u = x)$ und die Einerklasse ist mit der Allklasse identisch. Das f{\"u}hrt zu
einem technisch einfacheren Umgang mit der Einerklasse.\footnote{Andere Autoren wie z.~B. auch
K.~G{\"o}del, definieren $\{x\}$ durch $\{u~|~u = x\}$.}
\par
\begin{thm}
\begin{equation}
\neg \, \isSet(x) \rightarrow x = \allClass
\end{equation}
\end{thm}

\begin{thm}
\begin{equation}
\forall \, \isSet(x) \ \forall \, \isSet(y) \ \{x\} = \{y\} \leftrightarrow x = y
\end{equation}
\end{thm}

\par
\begin{defn}[Klassenangabe durch Aufz{\"a}hlung]\index{Klasse!Angabe durch Aufz{\"a}hlung}
\begin{eqnarray}
\{x, y\} & := & \{x\} \cup \{y\} \\
\{x, y, z\} & := & \{x\} \cup \{y\} \cup \{z\} \\
\end{eqnarray}
\end{defn}

\section{Geordnetes Klassenpaar}
Die Definition eines geordneten Paares $\langle x, y\rangle$ erfolgt nach \emph{N.~Wiener} (1914)
bzw. \emph{K.~Kuratowski} (1921).

\begin{defn}[Geordnetes Klassenpaar]\index{Klassenpaar!geordnetes}
$$
\langle x, y\rangle \ := \ \{\{x\}, \{x, y\}\}
$$
\end{defn}

\begin{thm}
\begin{equation}
\forall \, \isSet(x) \ \forall \, \isSet(y) \ \forall z \ z \in \{x, y\} \leftrightarrow z = x \lor z = y
\end{equation}
\end{thm}

\begin{thm}
\begin{equation}
(\neg \, \isSet(x) \lor \neg \, \isSet(y)) \rightarrow \{x, y\} = \allClass
\end{equation}
\end{thm}

\begin{thm}
\begin{eqnarray}
\{x, y\} & = & \{y, x\} \\
\{x, x\} & = & \{x\}
\end{eqnarray}
\end{thm}


\section{Kartesisches Produkt}
Das Kartesische Produkt\footnote{Kartesisch oder kartesianisch nach der lateinischen Namensform
\emph{Cartesius} des Philosophen und Mathematikers \emph{R. Descartes}.}, auch \emph{Kreuzprodukt}
genannt ist die Klasse aller geordneter Paare, deren Elemente aus diesen Klassen stammen.
\begin{defn}[Kartesisches Produkt]
$$
x \times y \ := \ \{z~|~\exists u \in x \ \exists v \in y \ (z = \langle u, v\rangle)\}
$$
\end{defn}

+++

\section{Relationen}
\begin{defn}[Relation]\index{Relation}
$$\rel(x) \ \Leftrightarrow \ \forall y \in x \ \exists \isSet(u) \ \exists \isSet(v) \ y = \langle u, v\rangle$$
\end{defn}

\begin{defn}[Definitionsbereich]\index{Definitionsbereich}
$$\dmn(x) \ := \ \{u~|~\isSet(u) \land \exists \isSet(v) \ \langle u, v\rangle \in x\}$$
\end{defn}

\begin{defn}[Wertebereich]\index{Wertebereich}
$$\rng(x) \ := \ \{v~|~\isSet(v) \land \exists \isSet(u) \ \langle u, v\rangle \in x\}$$
\end{defn}

+++

\section{Relationenalgebra}
+++
\section{{\"A}quivalenzrelationen}
+++
\section{Abbildungen und Funktionen}
+++
\par
\begin{defn}[Funktionale Klasse]\index{Klasse!funktionale}
$$\fkt(x) \ \Leftrightarrow \ \rel(x) \land \forall u \in \dmn(x) \ \exists! v \ \langle u, v\rangle \in x$$
\end{defn}

\begin{ax}[Fraenkelsches Ersetzungsaxiom]\index{Axiom!Fraenkelsches Ersetzungs-}\index{Substitutionsaxiom}
$$\fkt(x) \land \isSet(\dmn(x)) \rightarrow \isSet(\rng(x))$$
\end{ax}


\chapter{Nat{\"u}rliche Zahlen}
+++
\section{Definition und Grundeigenschaften}
+++
\section{Induktion}
+++
\section{Folgen und normale Funktionen}
+++
\section{Rekursion}
+++
\chapter{Auswahlaxiom}
+++
\par
\begin{ax}[Auswahlaxiom f{\"u}r Relationen]\index{Axiom!Auswahl-}
$$\rel(x) \rightarrow \exists \fkt(y) \ y \subset x \land \dmn(x) = \dmn(y) $$
\end{ax}

\section{Wohlordnungen}
+++
\section{Anwendungen des Auswahlaxioms}

\chapter{Kontinuum}
+++


%\backmatter	%% TODO what is that?

\begin{appendix}
\chapter{GNU Free Documentation License\index{GFDL}} \label{GNU_Free_Documentation_License}

{ \small \sffamily
\renewcommand{\LaTeX}{LaTeX}

 \begin{flushleft}
 Version 1.2, November 2002\\[.5ex]
 Copyright \copyright\ 2000,2001,2002  Free Software Foundation, Inc.\\
	 59~Temple Place, Suite~330, Boston, MA	 02111-1307	 USA\\[.5ex]
 Everyone is permitted to copy and distribute verbatim copies
 of this license document, but changing it is not allowed.
 \end{flushleft}


\subsection*{PREAMBLE}

The purpose of this License is to make a manual, textbook, or other functional and useful document
``free'' in the sense of freedom: to assure everyone the effective freedom to copy and redistribute
it, with or without modifying it, either commercially or noncommercially. Secondarily, this License
preserves for the author and publisher a way to get credit for their work, while not being
considered responsible for modifications made by others.

This License is a kind of ``copyleft'', which means that derivative works of the document must
themselves be free in the same sense.  It complements the GNU General Public License, which is a
copyleft license designed for free software.

We have designed this License in order to use it for manuals for free software, because free
software needs free documentation: a free program should come with manuals providing the same
freedoms that the software does.  But this License is not limited to software manuals; it can be
used for any textual work, regardless of subject matter or whether it is published as a printed
book.  We recommend this License principally for works whose purpose is instruction or reference.


\subsection*{APPLICABILITY AND DEFINITIONS}
\label{applicability}

This License applies to any manual or other work, in any medium, that contains a notice placed by
the copyright holder saying it can be distributed under the terms of this License.	Such a notice
grants a world-wide, royalty-free license, unlimited in duration, to use that work under the
conditions stated herein.  The ``Document'', below, refers to any such manual or work.	Any member
of the public is a licensee, and is addressed as ``you''.  You accept the license if you copy,
modify or distribute the work in a way requiring permission under copyright law.

A ``Modified Version'' of the Document means any work containing the Document or a portion of it,
either copied verbatim, or with modifications and/or translated into another language.

A ``Secondary Section'' is a named appendix or a front-matter section of the Document that deals
exclusively with the relationship of the publishers or authors of the Document to the Document's
overall subject (or to related matters) and contains nothing that could fall directly within that
overall subject.  (Thus, if the Document is in part a textbook of mathematics, a Secondary Section
may not explain any mathematics.)  The relationship could be a matter of historical connection with
the subject or with related matters, or of legal, commercial, philosophical, ethical or political
position regarding them.

The ``Invariant Sections'' are certain Secondary Sections whose titles are designated, as being
those of Invariant Sections, in the notice that says that the Document is released under this
License.  If a section does not fit the above definition of Secondary then it is not allowed to be
designated as Invariant.  The Document may contain zero Invariant Sections.	 If the Document does
not identify any Invariant Sections then there are none.

The ``Cover Texts'' are certain short passages of text that are listed, as Front-Cover Texts or
Back-Cover Texts, in the notice that says that the Document is released under this License.	 A
Front-Cover Text may be at most 5 words, and a Back-Cover Text may be at most 25 words.

A ``Transparent'' copy of the Document means a machine-readable copy, represented in a format whose
specification is available to the general public, that is suitable for revising the document
straightforwardly with generic text editors or (for images composed of pixels) generic paint
programs or (for drawings) some widely available drawing editor, and that is suitable for input to
text formatters or for automatic translation to a variety of formats suitable for input to text
formatters.	 A copy made in an otherwise Transparent file format whose markup, or absence of
markup, has been arranged to thwart or discourage subsequent modification by readers is not
Transparent. An image format is not Transparent if used for any substantial amount of text.	 A copy
that is not ``Transparent'' is called ``Opaque''.

Examples of suitable formats for Transparent copies include plain ASCII without markup, Texinfo
input format, \LaTeX\ input format, SGML or XML using a publicly available DTD, and
standard-conforming simple HTML, PostScript or PDF designed for human modification.	 Examples of
transparent image formats include PNG, XCF and JPG.	 Opaque formats include proprietary formats
that can be read and edited only by proprietary word processors, SGML or XML for which the DTD
and/or processing tools are not generally available, and the machine-generated HTML, PostScript or
PDF produced by some word processors for output purposes only.

The ``Title Page'' means, for a printed book, the title page itself, plus such following pages as
are needed to hold, legibly, the material this License requires to appear in the title page.  For
works in formats which do not have any title page as such, ``Title Page'' means the text near the
most prominent appearance of the work's title, preceding the beginning of the body of the text.

A section ``Entitled XYZ'' means a named subunit of the Document whose title either is precisely
XYZ or contains XYZ in parentheses following text that translates XYZ in another language.	(Here
XYZ stands for a specific section name mentioned below, such as ``Acknowledgements'',
``Dedications'', ``Endorsements'', or ``History''.)	 To ``Preserve the Title'' of such a section
when you modify the Document means that it remains a section ``Entitled XYZ'' according to this
definition.

The Document may include Warranty Disclaimers next to the notice which states that this License
applies to the Document.  These Warranty Disclaimers are considered to be included by reference in
this License, but only as regards disclaiming warranties: any other implication that these Warranty
Disclaimers may have is void and has no effect on the meaning of this License.


\subsection*{VERBATIM COPYING}
\label{verbatim}

You may copy and distribute the Document in any medium, either commercially or noncommercially,
provided that this License, the copyright notices, and the license notice saying this License
applies to the Document are reproduced in all copies, and that you add no other conditions
whatsoever to those of this License.  You may not use technical measures to obstruct or control the
reading or further copying of the copies you make or distribute.  However, you may accept
compensation in exchange for copies.  If you distribute a large enough number of copies you must
also follow the conditions in section ``COPYING IN QUANTITY''.

You may also lend copies, under the same conditions stated above, and you may publicly display
copies.


\subsection*{COPYING IN QUANTITY}
\label{copying}

If you publish printed copies (or copies in media that commonly have printed covers) of the
Document, numbering more than 100, and the Document's license notice requires Cover Texts, you must
enclose the copies in covers that carry, clearly and legibly, all these Cover Texts: Front-Cover
Texts on the front cover, and Back-Cover Texts on the back cover.  Both covers must also clearly
and legibly identify you as the publisher of these copies.	The front cover must present the full
title with all words of the title equally prominent and visible.  You may add other material on the
covers in addition. Copying with changes limited to the covers, as long as they preserve the title
of the Document and satisfy these conditions, can be treated as verbatim copying in other respects.

If the required texts for either cover are too voluminous to fit legibly, you should put the first
ones listed (as many as fit reasonably) on the actual cover, and continue the rest onto adjacent
pages.

If you publish or distribute Opaque copies of the Document numbering more than 100, you must either
include a machine-readable Transparent copy along with each Opaque copy, or state in or with each
Opaque copy a computer-network location from which the general network-using public has access to
download using public-standard network protocols a complete Transparent copy of the Document, free
of added material. If you use the latter option, you must take reasonably prudent steps, when you
begin distribution of Opaque copies in quantity, to ensure that this Transparent copy will remain
thus accessible at the stated location until at least one year after the last time you distribute
an Opaque copy (directly or through your agents or retailers) of that edition to the public.

It is requested, but not required, that you contact the authors of the Document well before
redistributing any large number of copies, to give them a chance to provide you with an updated
version of the Document.


\subsection*{MODIFICATIONS}
\label{modifications}

You may copy and distribute a Modified Version of the Document under the conditions of sections
``VERBATIM COPYING'' and ``COPYING IN QUANTITY'' above, provided that you release the Modified
Version under precisely this License, with the Modified Version filling the role of the Document,
thus licensing distribution and modification of the Modified Version to whoever possesses a copy of
it.	 In addition, you must do these things in the Modified Version:

\begin{enumerate}% [A.]
\item Use in the Title Page (and on the covers, if any) a title distinct
   from that of the Document, and from those of previous versions
   (which should, if there were any, be listed in the History section
   of the Document).  You may use the same title as a previous version
   if the original publisher of that version gives permission.
\item List on the Title Page, as authors, one or more persons or entities
   responsible for authorship of the modifications in the Modified
   Version, together with at least five of the principal authors of the
   Document (all of its principal authors, if it has fewer than five),
   unless they release you from this requirement.
\item State on the Title page the name of the publisher of the
   Modified Version, as the publisher.
\item Preserve all the copyright notices of the Document. \item Add an appropriate copyright notice
for your modifications
   adjacent to the other copyright notices.
\item Include, immediately after the copyright notices, a license notice
   giving the public permission to use the Modified Version under the
   terms of this License, in the form shown in the Addendum below.
\item Preserve in that license notice the full lists of Invariant Sections
   and required Cover Texts given in the Document's license notice.
\item Include an unaltered copy of this License. \item Preserve the section Entitled ``History'',
Preserve its Title, and add
   to it an item stating at least the title, year, new authors, and
   publisher of the Modified Version as given on the Title Page.  If
   there is no section Entitled ``History'' in the Document, create one
   stating the title, year, authors, and publisher of the Document as
   given on its Title Page, then add an item describing the Modified
   Version as stated in the previous sentence.
\item Preserve the network location, if any, given in the Document for
   public access to a Transparent copy of the Document, and likewise
   the network locations given in the Document for previous versions
   it was based on.	 These may be placed in the ``History'' section.
   You may omit a network location for a work that was published at
   least four years before the Document itself, or if the original
   publisher of the version it refers to gives permission.
\item For any section Entitled ``Acknowledgements'' or ``Dedications'',
   Preserve the Title of the section, and preserve in the section all
   the substance and tone of each of the contributor acknowledgements
   and/or dedications given therein.
\item Preserve all the Invariant Sections of the Document,
   unaltered in their text and in their titles.	 Section numbers
   or the equivalent are not considered part of the section titles.
\item Delete any section Entitled ``Endorsements''.	 Such a section
   may not be included in the Modified Version.
\item Do not retitle any existing section to be Entitled ``Endorsements''
   or to conflict in title with any Invariant Section.
\item Preserve any Warranty Disclaimers.

\end{enumerate}

If the Modified Version includes new front-matter sections or appendices that qualify as Secondary
Sections and contain no material copied from the Document, you may at your option designate some or
all of these sections as invariant.	 To do this, add their titles to the list of Invariant Sections
in the Modified Version's license notice. These titles must be distinct from any other section
titles.

You may add a section Entitled ``Endorsements'', provided it contains nothing but endorsements of
your Modified Version by various parties--for example, statements of peer review or that the text
has been approved by an organization as the authoritative definition of a standard.

You may add a passage of up to five words as a Front-Cover Text, and a passage of up to 25 words as
a Back-Cover Text, to the end of the list of Cover Texts in the Modified Version.  Only one passage
of Front-Cover Text and one of Back-Cover Text may be added by (or through arrangements made by)
any one entity.	 If the Document already includes a cover text for the same cover, previously added
by you or by arrangement made by the same entity you are acting on behalf of, you may not add
another; but you may replace the old one, on explicit permission from the previous publisher that
added the old one.

The author(s) and publisher(s) of the Document do not by this License give permission to use their
names for publicity for or to assert or imply endorsement of any Modified Version.


\subsection*{COMBINING DOCUMENTS}
\label{combining}

You may combine the Document with other documents released under this License, under the terms
defined in section ``MODIFICATIONS'' above for modified versions, provided that you include in the
combination all of the Invariant Sections of all of the original documents, unmodified, and list
them all as Invariant Sections of your combined work in its license notice, and that you preserve
all their Warranty Disclaimers.

The combined work need only contain one copy of this License, and multiple identical Invariant
Sections may be replaced with a single copy.  If there are multiple Invariant Sections with the
same name but different contents, make the title of each such section unique by adding at the end
of it, in parentheses, the name of the original author or publisher of that section if known, or
else a unique number. Make the same adjustment to the section titles in the list of Invariant
Sections in the license notice of the combined work.

In the combination, you must combine any sections Entitled ``History'' in the various original
documents, forming one section Entitled ``History''; likewise combine any sections Entitled
``Acknowledgements'', and any sections Entitled ``Dedications''.  You must delete all sections
Entitled ``Endorsements''.


\subsection*{COLLECTIONS OF DOCUMENTS}
\label{collections}

You may make a collection consisting of the Document and other documents released under this
License, and replace the individual copies of this License in the various documents with a single
copy that is included in the collection, provided that you follow the rules of this License for
verbatim copying of each of the documents in all other respects.

You may extract a single document from such a collection, and distribute it individually under this
License, provided you insert a copy of this License into the extracted document, and follow this
License in all other respects regarding verbatim copying of that document.


\subsection*{AGGREGATION WITH INDEPENDENT WORKS}
\label{aggregation}

A compilation of the Document or its derivatives with other separate and independent documents or
works, in or on a volume of a storage or distribution medium, is called an ``aggregate'' if the
copyright resulting from the compilation is not used to limit the legal rights of the compilation's
users beyond what the individual works permit. When the Document is included in an aggregate, this
License does not apply to the other works in the aggregate which are not themselves derivative
works of the Document.

If the Cover Text requirement of section ``COPYING IN QUANTITY'' is applicable to these copies of
the Document, then if the Document is less than one half of the entire aggregate, the Document's
Cover Texts may be placed on covers that bracket the Document within the aggregate, or the
electronic equivalent of covers if the Document is in electronic form. Otherwise they must appear
on printed covers that bracket the whole aggregate.


\subsection*{TRANSLATION}
\label{translation}

Translation is considered a kind of modification, so you may distribute translations of the
Document under the terms of section ``MODIFICATIONS''. Replacing Invariant Sections with
translations requires special permission from their copyright holders, but you may include
translations of some or all Invariant Sections in addition to the original versions of these
Invariant Sections.	 You may include a translation of this License, and all the license notices in
the Document, and any Warranty Disclaimers, provided that you also include the original English
version of this License and the original versions of those notices and disclaimers.	 In case of a
disagreement between the translation and the original version of this License or a notice or
disclaimer, the original version will prevail.

If a section in the Document is Entitled ``Acknowledgements'', ``Dedications'', or ``History'', the
requirement (section ``MODIFICATIONS'') to Preserve its Title (section ``APPLICABILITY AND
DEFINITIONS'') will typically require changing the actual title.


\subsection*{TERMINATION}
\label{termination}

You may not copy, modify, sublicense, or distribute the Document except as expressly provided for
under this License.	 Any other attempt to copy, modify, sublicense or distribute the Document is
void, and will automatically terminate your rights under this License.	However, parties who have
received copies, or rights, from you under this License will not have their licenses terminated so
long as such parties remain in full compliance.


\subsection*{FUTURE REVISIONS OF THIS LICENSE}
\label{future}

The Free Software Foundation may publish new, revised versions of the GNU Free Documentation
License from time to time.	Such new versions will be similar in spirit to the present version, but
may differ in detail to address new problems or concerns.  See \path{http://www.gnu.org/copyleft/}.

Each version of the License is given a distinguishing version number. If the Document specifies
that a particular numbered version of this License ``or any later version'' applies to it, you have
the option of following the terms and conditions either of that specified version or of any later
version that has been published (not as a draft) by the Free Software Foundation.  If the Document
does not specify a version number of this License, you may choose any version ever published (not
as a draft) by the Free Software Foundation.


\subsection*{ADDENDUM: How to use this License for your documents}

To use this License in a document you have written, include a copy of the License in the document
and put the following copyright and license notices just after the title page:
\begin{quote}
	Copyright \copyright{} \textit{year}\quad \textit{your name}.\\
	Permission is granted to copy, distribute and/or modify this document
	under the terms of the GNU Free Documentation License, Version 1.2
	or any later version published by the Free Software Foundation;
	with no Invariant Sections, no Front-Cover Texts, and no Back-Cover Texts.
	A copy of the license is included in the section entitled ``GNU
	Free Documentation License''.
\end{quote}
If you have Invariant Sections, Front-Cover Texts and Back-Cover Texts, replace the
``with...Texts.'' line with this:
\begin{quote}
	with the Invariant Sections being \textit{list their titles}, with the
	Front-Cover Texts being \textit{list}, and with the Back-Cover Texts being \textit{list}.
\end{quote}
If you have Invariant Sections without Cover Texts, or some other combination of the three, merge
those two alternatives to suit the situation.

If your document contains nontrivial examples of program code, we recommend releasing these
examples in parallel under your choice of free software license, such as the GNU General Public
License, to permit their use in free software.
\par
} % small sffamily

\begin{thebibliography}{99}

 % Logik
 
 \bibitem{witheruss} \emph{A.N. Whitehead, B. Russell}, Principia Mathematica, Cambridge University Press, London 1910
 
 \bibitem{bernays} \emph{P. Bernays}, Axiomatische Untersuchung des Aussagen-Kalkuls der {\glqq Principia Mathematica\grqq}, Math. Zeitschr. 25 (1926), 305-320
 
 \bibitem{hilback} \emph{D. Hilbert, W. Ackermann}, Grundz{\"u}ge der theoretischen Logik, Springer, Berlin 1928
 
 \bibitem{novikov} \emph{P.S. Novikov}, Grundz{\"u}ge der mathematischen Logik, VEB Deutscher Verlag der Wissenschaften, Berlin 1973
 
 % Mengenlehre
 
 \bibitem{monk} \emph{J.~D.~Monk}, Introduction to Set Theory, McGraw-Hill, New York 1969
 
 \bibitem{schmidt} \emph{J.~Schmidt}, Mengenlehre I, Bibliographisches Institut, Mannheim 1966
 
 \bibitem{friepres} \emph{U.~Friedrichsdorf, A.~Prestel}, Mengenlehre f{\"u}r den Mathematiker, Vieweg, Braunschweig 1985
 
 \bibitem{takzar} \emph{G.~Takenti, W.~M.~Zaring}, Introduction to Axiomatic Set Theory, McGraw-Hill, New York 1971
 
 \bibitem{kerby} \emph{W.~Kerby}, Vorlesung {\glqq Axiomatische Mengenlehre\grqq}, gehalten an der Universit{\"a}t Hamburg, Sommersemester 1992
 
 \bibitem{guenter} \emph{V.~G{\"u}nther}, Vorlesung {\glqq Mathematik und Logik\grqq}, gehalten an der Universit{\"a}t Hamburg, Wintersemester 1994/1995
 
 % Computer
 
 \bibitem{meyling}\emph{M. Meyling}, Hilbert II, Darstellung von formal korrektem mathematischen Wissen, \url{http://www.qedeq.org/0_01_05/projektbeschreibung.pdf}
 

\end{thebibliography}

\end{appendix}

\addcontentsline{toc}{chapter}{Literaturverzeichnis}

\addcontentsline{toc}{chapter}{\indexname} \printindex

\end{document}
